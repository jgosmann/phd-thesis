\chapter{Conclusion}
To summarize, the context-unified encoding model advances our understanding of human memory by matching human behavioural data using an implementation grounded in a spiking neural network.
The difficult task of building a spiking neuron model of this scale also led to a number advances in large-scale cognitive modeling in general, such as high-dimensional neural representations with reduced noise.
Despite this, the CUE model is only a first step in the integration of neural and behavioural data as well as the understanding the interaction of short- and long-term memory.
Much more data from memory experiments exists that could be matched, and can be used to highlight where the CUE model is wrong or needs to be extended.
One such aspect that was already evident is the strength of the forward recall bias in delayed free recall.
Future work could also focus on mapping the long-term memory components more precisely onto hippocampal structures and cellular properties in the relevant regions.
In particular, introducing sparsification with a dentate gyrus model could improve the biological plausibility of the AML\@.

The sequences that the CUE model can memorize are much simpler than the richness of actual human memory.
But they have been proven useful in psychology to investigate basic properties of memory.
Also, many forms of memory can be understood as a sequence:
episodic memory is essentially a sequence of events or a sequence of left and right turns might be necessary to navigate from one place to another.
Thus, the model is relevant to our understanding despite the relative simplicity of modelled tasks.

Finally, the CUE model, in the broader context of large scale cognitive modeling, could provide an excellent extension to the Spaun model.
A proper long-term memory component is missing from this model so far, despite being essential for cognition.
The CUE model itself could also benefit from such an integration, as Spaun's ability to perform multiple tasks would allow to model experimental delay phases with an actual distractor task.
