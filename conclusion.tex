\chapter{Conclusion}
To summarize, the context-unified encoding model advances our understanding of human memory by matching human behavioural data with an implementation grounded in a spiking neural network.
The difficult task of building a spiking neuron model of this scale also led to a number advances in large-scale cognitive modeling in general, such as high-dimensional neural representations with reduced noise.
Despite this, the CUE model is only a first step in the integration of neural and behavioural data as well as the understanding the interaction of short- and long-term memory.
Much more data from memory experiments exists that could be matched and might highlight where the CUE model is wrong or needs to be extended.
One such aspect that was already evident is the strength of the forward recall bias in delayed free recall.
Future work could also focus on mapping the long-term memory components more precisely on hippocampal structures and cellular properties in these regions.
In particular, introducing sparsification with a dentate gyrus model could improve the biological plausibility of the AML\@.

Finally, the CUE model, in the broader context of large scale cognitive modeling, could provide an excellent extension to the Spaun model.
A proper long-term memory component is missing from this model so far despite being essential for cognition.
But also the CUE model could benefit from such an integration, as Spaun's ability to perform multiple tasks would allow to model experimental delay phases with an actual distractor task.
