\chapter{Results}
To validate the CUE model, I matched it against human experimental data of serial and free recall experiments.
The same model architecture was used in all of these simulations, except for the Hebb repetition effect where the extensions discussed in the previous sections have been used.
Parameter values were kept constant across conditions as much as possible, but some small changes were necessary in some instances as noted when results are discussed.
The simulations were designed to replicate the experimental paradigms as closely as possible.
In particular, the list length, item presentation times, delay times, and recall times were matched to the actual experiments with human subjects.

To model the effect of distractor tasks during delay phases, non-list items where presented a rate of $\drate$ items per second.
These were allowed to influence the STM component, but learning in the LTM component was disabled as these items were irrelevant to the main memorization task.
This is similar to how \textcite{Howard2002} modeled the distractor interval, though in their case they did not define the distractor rate, but the effective length of the distractor interval.
As I was aiming to match the experimental paradigms as closely as possible, changing the length of the distractor interval was not an option.

Unless otherwise noted, \num{100} simulations with different seeds were run per experimental condition.
This number is sufficient to get clear results with reasonably small confidence intervals, but still small enough that the simulation and parameter matching is feasible on a high-performance computing cluster.
The values used for free parameters in the different settings are summarized in \cref{tbl:params}.
In addition to these, the context drift parameter was set to $\tcmbeta = 0.62676$ and the OSE short term decay was set to $\osestmdecay = 0.9775$ in all simulations.
These are the values reported as best fitting by \textcite{Sederberg2008} and \textcite{Choo2010} respectively.
Some other parameters, like the OSE scaling $\oseepisscale$ for episodic memory and the TCM ratio $\gamma$ for updating $\mft$, are not present in the CUE model.
The extensions for the Hebb repetition effect introduced two additional parameters.
The learning rate for the direct or forward associations was set \num{0.05} or \num{0.25} respectively.
The decay rate for the $\mtf$ and $\mft$ weights was set to \num{0.999973176} which corresponds to a decay to about $0.2$ of the original weight over a period of \SI{60}{\second} with a simulation timestep of \SI{1}{\milli\second}.
This decay required adjusting the bias of the null choice to $\minev = 0.015$.
\begin{table}
    \centering
    \caption[Summary of free parameter values.]{Summary of free parameters values for distractor rate $\drate$, probability $\psi$ of using the serial recall strategy, bias of the null choice $\minev$ in recall, standard deviation of the input noise $\recnoise$ in recall, and the AML learning rate $\eta$ for $\mtf$ and $\mft$. See text for discussion of the parameter choices and two additional parameters in the Hebb repetition condition not listed in the table.}\label{tbl:params}
    \begin{tabular}{lSSSSS}
        \toprule
        Experimental condition & $\drate / \si{\second^{-1}}$ & $\psi$ & $\minev$ & $\recnoise$ & $\eta$ \\
        \midrule
        \textbf{Serial recall} & & & & & \\
        \hspace{1em}Immediate & {\textemdash} & 1 & 0.04 & 0.009 & 10 \\
        \hspace{1em}w/o STM & {\textemdash} & 1 & 0.04 & 0.009 & 10 \\
        \hspace{1em}w/o LTM & {\textemdash} & 1 & 0.03 & 0.015 & 10 \\
        \textbf{Free recall} & & & & & \\
        \hspace{1em}Immediate & {\textemdash} & 0.1 & 0.04 & 0.015 & 10 \\
        \hspace{1em}Delayed & 0.4 & 0 & 0.0325 & 0.015 & 10 \\
        \hspace{1em}Continuous distractor & 0.3 & 0.1 & 0.03 & 0.009 & 10 \\
        \textbf{Scopolamine} & & & & \\
        \hspace{1em}Placebo & {\textemdash} & 0.1 & 0.02 & 0.015 & 10 \\
        \hspace{1em}Scopolamine & {\textemdash} & 0.1 & 0.02 & 0.015 & 0.1 \\
        \textbf{Hebb repetition} & {\textemdash} & 1 & 0.015 & 0.009 & 10 \\
        \bottomrule
    \end{tabular}
\end{table}

\section{Serial recall}
In serial recall participants are asked to recall items in the same order as they were presented.
In an experiment presented by \textcite{Jahnke1968}, lists of ten items were presented at the rate of one item per second and recalled immediately by the \num{96}~subjects.
\Cref{fig:results-serial} shows the serial position curve for the experimental and model data and the distribution of transposition errors averaged over all serial positions.
In both cases the serial position curve shows a clear primacy and recency effect.
The model predictions are statistically indistinguishable from the human data as all confidence intervals overlap.
While \textcite{Jahnke1968} did not provide the transposition data, the model qualitatively matches the transposition gradients reported elsewhere \parencite[e.g.,][]{Henson1996}.
In general, few transposition errors are made, but for those that occur, transpositions of nearby items are more likely than transpositions of distant items.
\begin{figure}
    \centering
    \includegraphics{figures/results/serial}
    \caption[Serial position curve and transpositions for serial recall with the CUE model.]{Serial position curve (left) and transpositions (right) for the serial recall of a \num{10}~item list with the CUE model. The experimental data from \textcite{Jahnke1968} is shown for comparison in the serial position curve (blue squares). The error bars show \SI{95}{\percent} confidence intervals.}\label{fig:results-serial}
\end{figure}

The model allows selective disabling of the recall from the STM or LTM component.
Doing so, with appropriate adjustment of the recall noise level to account for the reduced evidence input, shows that the primacy effect is mediated by the LTM, while the recency effect depends on the STM (\cref{fig:results-no_xtm}).
The recall performance without the LTM contribution is also much worse.
This might be the case either because the input to the recall network might need further adjustment or because no rehearsal mechanism is modelled, resulting in drift of the OSE integrator.
\begin{figure}
    \centering
    \includegraphics{figures/results/no_xtm}
    \caption[Serial position curves with disabled STM/LTM.]{Serial position curves when either (a) STM recall or (b) LTM recall is disabled in the model. The error bars show \SI{95}{\percent} confidence intervals.}\label{fig:results-no_xtm}
\end{figure}


\section{Free recall}
While the order of recall is predetermined in serial recall, in free recall list items may be recalled in any order.
Here, I provide the model match to the data from \textcite{Howard1999}, which has also been used in the original fits of the TCM model in \textcite{Howard2002} and \textcite{Sederberg2008}.
Three experimental conditions are matched: immediate recall, delayed recall, and continuous distractor recall.

In the immediate free recall condition, list items are presented at a rate of one item every second.
After the presentation phase, a recall phase of \SI{45}{\second} followed immediately.
This protocol is changed to a presentation rate of one item every \SI{1.2}{\second} and a recall phase of \SI{60}{\second} in the delayed and continuous distractor conditions.
In both of these latter conditions, the presentation and recall phase are separated by a \SI{16}{\second} distractor task.
In addition, in the continuous distractor conditions such a \SI{16}{\second} distractor phase is inserted in between every pair of items.
A list length of twelve items is used in all conditions.
The experimental data was obtained from \num{65}~subjects presented with \num{25}~lists each for the immediate recall condition, and from \num{16}~subjects presented with \num{15}~lists each for the remaining conditions.

Four resulting metrics from the model and experimental data are shown in \cref{fig:results-free}.
First, the distribution of the total number of successful recalls is shown.
To my knowledge, this data has not been analyzed for the original TCM model, even though it is arguably the most fundamental comparison.
In all conditions, the \SI{95}{\percent} confidence intervals of the mean, standard deviation, and kurtosis overlap with the exception of the kurtosis in the immediate recall condition.
Thus, no significant difference for the most essential moments is shown, which indicates that the model approximates the experimental distributions well, even though an equality of the distributions cannot be inferred.
\begin{figure}
    \centering
    \includegraphics[trim=0 6 0 6]{figures/results/free}
    \caption[Comparison of experimental and model free recall data.]{Comparison of experimental and model free recall data. The columns show the immediate, delayed, and continuous distractor conditions. The rows show from top to bottom: distribution of the number of successful recalls (mean marked by vertical line), serial positions curves, probability of first recall, and the conditional response probability. The error bars show \SI{95}{\percent} confidence intervals. Experimental data by \textcite{Howard1999}.}\label{fig:results-free}
\end{figure}

Second, the serial position curves are given.
The strong recency effect in immediate recall is attenuated in delayed recall, but reappears to some degree in continuous distractor recall.
Interestingly, the recency effect in immediate recall gives the curve an S-like shape that is missing in continuous distractor recall.

Third, these effects also show in the probability of first recall.
In immediate recall, the first recall is, with high probability, from the end of the list, whereas in delayed recall the probability is much more uniform.
In continuous distractor recall, the probability to start the recall at the end of the list is partially restored.

Finally, the conditional response probability (CRP) gives the probability of how much lag there is between the positions of two recalled items.
For example, the asymmetry in immediate recall shows the bias to do forward recall and the peak around zero that nearby items tend to be recalled together.
Both of these effects become attenuated in delayed and continuous distractor recall.
In delayed free recall, the model predicts a stronger forward bias than the experimental data shows.

The model provides an excellent fit on most measures.
The confidence intervals of \num{101} of the \num{108} data points overlap.
This amounts to less than \SI{7}{\percent} confidence intervals that do not overlap, while about \SI{5}{\percent} are expected to be non-overlapping by pure chance given a \SI{95}{\percent} confidence level.
The most salient deviation of the model and experimental data is observed in the CRP curve for the delayed recall condition.
The model predicts a slightly higher forward bias than is actually found.


\section{Scopolamine}
A spiking neural network model allows the investigation of the effects of drugs more readily than a pure mathematical model.
I demonstrate this here with the acetylcholine antagonist scopolamine.
Administered before the presentation phase in an immediate recall experiment, scopolamine is detrimental to recall performance \parencite{ghoneim1975}.
However, scopolamine administered in between the presentation and recall phase does not influence performance.
This indicates that scopolamine prevents encoding of new memories in LTM, but does not prevent recall of already encoded memories.
More precisely, scopolamine has been shown to attenuate long-term potentiation in hippocampus \parencite{leung2003,ito1988,hirotsu1989-1}.

To model the effect of scopolamine on LTP, I adjusted the AML learning rate for learning the $\mtf$ and $\mft$ matrices.
With this approach the experimental results obtained by \textcite{ghoneim1975} from \num{36}~subjects (\num{8} trials each) are reproduced.
I focus here on the immediate recall experiment with \num{16}~item word lists.
The simulation protocols were again modeled to replicate the experimental settings, with a presentation time of two seconds per item.
To obtain a similar effect, the normal AML learning rate was used before the time point of scopolamine injection and in placebo trials.
After the time point of scopolamine injection it was set to \num{0.1}.

\Textcite{ghoneim1975} reported a recall accuracy of \SI{77.92(494)}{\percent} (mean $\smash{\pm}$ standard error) in the placebo condition and a recall accuracy of \SI{31.67(228)}{\percent} in the scopolamine condition.
The model produces recall accuracies of \SI{74.18(108)}{\percent} and
\SI{37.94(99)}{\percent}
respectively.
Moreover, the serial position curve for the scopolamine condition shows no primacy effect, but a recency effect (\cref{fig:scopolamine-serial}).
\begin{figure}
    \centering
    \includegraphics{figures/results/scopolamine-serial}
    \caption[Serial position curve with a scopolamine injection predicted by the CUE model.]{Serial position curve with a scopolamine injection predicted by the CUE model. The error bars show \SI{95}{\percent} confidence intervals.}\label{fig:scopolamine-serial}
\end{figure} 


\section{Hebb repetition effect}
To model the Hebb repetition effect, the extensions described in \cref{sec:hebbext} where required.
The simulations for the effect were modelled after \textcite{Hebb1961}.
A total of \num{25} model instances, equivalent to the \num{25} experimental subjects, were run for \num{24} consecutive trials each.
Each trial consisted of a nine item list (of the digits from one to nine in random order) and starting with the third trial every third list was identical.

While the model performance seems slightly worse overall, the qualitive Hebb repetititon effect is reproduced (\cref{fig:hebb}).
\begin{figure}
    \centering
    \includegraphics{figures/hebb}
    \caption[Hebb repetition effect.]{Experimental and model data showing the Hebb repetition effect. Nine item lists were presented and one list was repeated on every third trial. From left to right: experimental data \parencite{Hebb1961}, model data with direct learning of position to item associations, and model data with learning of forward associations. The error bars show \SI{95}{\percent} confidence intervals (no confidence intervals were provided for the experimental data).}\label{fig:hebb}
\end{figure}
In both the experimental and all sets of model data, the number of correct recalls increases by about two to three items on the repeated list.


\section{Memory encoding}
As a spiking neural network model, the CUE model allows the recording of spikes and examination of changes in neural firing (\cref{fig:spikes}).
The distribution of active neurons changes for the STM neurons with each item, with a delay of about \SI{250}{\milli\second} to encode the new item.
When no new item is present, persistent neural firing preserves the firing pattern.
Similar behaviour is observed for neurons encoding the current context signal that gets updated with about a \SI{100}{\milli\second} delay.
The LTM for $\mtf$ (and $\mft$, but not shown) is encoded in neural weights that change for each list item to encode the newly learned associations.
\begin{figure}
    \centering
    \includegraphics{figures/spikes}
    \caption[Memory encoding in the CUE model.]{Memory encoding in the CUE model. From top to bottom: weights encoding $\mtf$ in LTM, spiking activity of a subset of STM neurons, and spiking activity of a subset of neurons encdoing the context signal $\ctx$. The colored bars at the top mark the presentation of four different list items.}\label{fig:spikes}
\end{figure}

\Textcite{ninokura2003} identified neurons with activity selective to the serial order of items in the lateral prefrontal cortex cortex of monkeys.
Using permutations of a three item list, analogous to \textcite{ninokura2003}, similar neurons can be found among the STM neurons of the CUE model (\cref{fig:seqsel}).
Besides neurons selective to a single permutation of items that fire persistently during the delay period, some neurons appear selective for one permutation and either increase or decrease their firing rate during the delay period.
Some neurons are selective for multiple sequences.
\Cref{fig:seqsel}d shows a neurons that responds to sequences starting with ZY, but also the sequence XYZ\@.
Finally, many neurons are much more complex in their responses and cannot easily be assigned a specific preferred stimulus sequence (no figure shown).
\begin{figure}
    \centering
    \includegraphics{figures/seqsel}
    \caption[Firirng rate of selected STM neurons.]{Firing rate of selected STM neurons, smoothed with a Gaussian filter with a standard deviation of \SI{25}{\milli\second}, in response to permutations of a three item list. Analogous to \textcite{ninokura2003}, each list item was presented for \SI{0.5}{\second} preceded by \SI{1}{\second} delays. The presentation of all items was followed by a \SI{1.5}{\second} delay period. The presentation intervals for the second and third item are marked with the vertical lines.}\label{fig:seqsel}
\end{figure}

Furthermore, \textcite{folkerts2018} recorded from the medial temporal lobe of epilepsy patients and found that the population activity vector exhibits a neural recency effect.
The similarity of the population vector decreases with lag between presented item.
A similar analysis can be done in the CUE model by recording the spikes from the context component of the model.
The population vector is given by the average firing rates during the presentation window of each item in a list.
A decline in similarity qualitatively similar to \textcite{folkerts2018} when comparing to less recent population vectors can be found as shown in \cref{fig:popsim}.
The absolute similarity values are higher in then reported from the experimental data which can be attributed to the possibility of specifically recording from the neurons responsible for representing the context, whereas experimentally many unrelated neurons will be included in the analysis.
\begin{figure}
    \centering
    \includegraphics{figures/popsim}
    \caption{Similarity of the context population vector with recency.}\label{fig:popsim}
\end{figure}
