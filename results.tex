\chapter{Results}
To validate the CUE model, I matched it against human experimental data of serial and free recall experiments.
The same model architecture was used in all of these simulations with only small changes is some parameter values across experimental conditions.
The simulations were designed to replicate the experimental paradigms as closely as possible.
In particular, the list length, item presentation times, delay times, and recall times where matched.

To model the effect of distractor tasks during delay phases, non-list items where presented a rate of $\drate$ items per second.
These were allowed to influence the STM component, but learning in the LTM component was disabled as these items were irrelevant to the main memorization task.
This is similar to how \textcite{Howard2002} modeled the distractor interval, though in their case they did not define the distractor rate, but the effective length of the distractor interval.
As I was aiming to match the experimental paradigms as closely as possible, changing the length of the distractor interval was not an option.

As long as not otherwise noted, 100 simulations with different seeds were run per experimental condition.
This number is sufficient to get clear results with reasonably small confidence intervals, but still small enough that the simulation and parameter matching is feasible on a high-performance computing cluster.


TODO parameter values


\section{Serial recall}
In serial recall participants are asked to recall items in the same order as they were presented.
In an experiment presented by \textcite{Jahnke1968} list of ten items was presented at the rate of one item per second and recalled immediately.
Figure TODO shows the serial position curve for the experimental and model data and the distribution of transposition errors averaged over all serial positions.
In both cases the serial position curve show a clear primacy and recency effect.
While \textcite{Jahnke1968} did not provide the transposition data, the model matches qualitatively the transposition gradients reported elsewhere (TODO ref).
In general, few transposition errors are made, but for those that occur transpositions of nearby items are more likely than transpositions of distant items.

TODO primacy by LTM, recency by STM\@?


\section{Free recall}
While the order of recall is predetermined in serial recall, in free recall list items may be recalled in any order.
Here, I provide the model match to the data from \textcite{Howard1999} which also has been used in original fits of the TCM model in \textcite{Howard2002} and \textcite{Sederberg2008}.
Three experimental conditions were attempted to match: immediate recall, delayed recall, and continuous distractor recall.

In the immediate free recall condition, list items were presented at a rate of one item every second.
After the presentation phase a recall phase of \SI{45}{\second} followed immediately.
This protocol was changed to a presentation rate of one item every \SI{1.2}{\second} and a recall phase of \SI{60}{\second} in the delayed and continuous distractor conditions.
In both of these conditions, the presentation and recall phase were separated by a \SI{16}{\second} distractor task.
In addition, in the continuous distractor conditions such a \SI{16}{\second} distractor phase would be inserted in between every pair of items.
A list length of 12 items was used in all conditions.

A number of resulting metrics from the model and experimental data is shown in Figure TODO\@.
First, the distribution of the total number of successful recalls is shown.
To my knowledge, this data has not been analyzed for the original TCM model, even though it is arguably the most fundamental comparison.
In all conditions, the \SI{95}{\percent} confidence intervals of the mean, standard deviation, and kurtosis overlap with the exception of the kurtosis in the immediate recall condition.
Thus, no significant difference for the most essential moments could be shown which indicates that the model approximates the experimental distributions well, even though an equality of the distributions cannot be inferred.
TODO KL-divergence?

Second, the serial position curves are given.
The strong recency effect in immediate recall is attenuated in delayed recall, but reappears to some degree in continuous distractor recall.
Interestingly, the recency effect in immediate recall give the curve an S-like shape that is missing in continuous distractor recall.

Third, these effects also show in the probability of first recall.
In immediate recall, the first recall is with high probability from the end of the list, whereas in delayed recall the probability is much more uniform.
In continuous distractor recall, the probability to start the recall at the end of the list is partially restored.

Finally, the conditional response probability (CRP) gives the probability of how much lag between the positions of to recalled items is.
For example, the asymmetry in immediate recall shows the bias to do forward recall and the peak around zero that nearby items tend to be recalled together.
Both of these effects become attenuated in delayed and continuous distractor recall.
In delayed free recall, the model predicts a stronger forward bias than the experimental data shows.


\section{Scopolamine}
A spiking neural network model allows to investigate the effects of drugs more readily than a pure mathematical model.
I demonstrate this here with the acetylcholine antagonist scopolamine.
Administered before presentation phase in an immediate recall experiment, it will be detrimental to the recall performance \parencite{ghoneim1975}.
However, scopolamine administered inbetween the presentation and recall phase does not influence performance.
This indicates that scopolamine prevents encoding of new memories in LTM, but does not prevent recall of already encoded memories.
More precisely, scopolamine has been shown to attenuate long-term potentiation in hippocampus \parencite{leung2003,ito1988,hirotsu1989}.

To model the effect of scopolamine on LTP, the AML learning rate for learning the $\mtf$ and $\mft$ matrices can be adjusted.
By reducing it to TODO, the experimental results by \textcite{ghoneim1975} can be reproduced (see Figure TODO).
The simulation protocols were again modeled to replicate the experimental settings with 16 items per list, and a presentation time of two seconds per item.


\section{Hebb repetition effect}


\section{Spiking behaviour}
