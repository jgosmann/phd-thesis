\chapter{The Ordinal Serial Encoding Model}
The Ordinal Serial Encoding (OSE) model (TODO ref) is an NEF and SPA based model of serial recall.
It was able to reproduce the various effects found in human recall data such as the primacy effect, recency effect, and transposition gradients in serial and delayed forward recall.
Within the integrated memory model it provides the basis for the short-term memory component.

In the OSE $m$ presented items $\vc v_i$ are bound to fixed position vectors $\vc p_i$ and stored in two memory traces
\begin{align}
    \osestm &= \sum_{i=1}^m \osestmdecay^{m - i} \bind(\vc v_i, \vc p_i) \\
    \oseepis &= \sum_{i=1}^m \oseepisscale^{m - i} \bind(\vc v_i, \vc p_i)
\end{align}
with decay factor $\osestmdecay < 1$ and scaling factor $\oseepisscale > 1$.
The memory traces $\osestm$ and $\oseepis$ represent the short-term and episodic memory store, respectively.
The binding operation $\bind$ used here is circular convolution, but it could be worth exploring the effect of other binding operations in the future.
The recall of an item is given by unbinding the corresponding position vector as
\begin{equation}
    \vc v_i \approx \bind^+(\osestm + \oseepis, \vc p_i) \text{.}
\end{equation}
These encoding equations produce the primacy and recency effect due to the differential effect of the decay and scaling factors $\osestmdecay$ and $\oseepisscale$.

For the neural implementation each memory trace can be stored in an integrator with some additional processes for updating and unbinding the recalled item.
For the integration with the integrated memory model, the episodic memory trace will be replaced by a version of the temporal context model presented in the next chapter.
This introduces a more plausible episodic memory storing experiences in actual synaptic weight changes than the activities of a neural population.
In addition the recall process will need to be adjusted.
