\section{Anatomical mapping}

Given that the CUE model is neural, it is of interest to consider how parts of the model map to brain areas.
\Textcite{howard2005} proposed a mapping of the TCM model to brain areas that applies to a large degree also to the TCM-based part of the CUE model.
There are however some details to be reconsidered and the OSE-based short-term part has obviously not been discussed.

The medial temporal lobe (MTL) is known to be essential for free recall.
Damage to the MTL is detrimental to free recall performance \parencite{graf1984}.
Thus, we can assume the TCM related parts of the model to reside in the MTL (TODO figure).

More precisely, the context storage network can be mapped  onto parahippocampal areas, in particular the entorhinal cortex (EC).
Its properties are consistent with the storage of non-spatial memories for tens of seconds.
In delay periods stimulus dependent persistent activity can be observed \parencite{suzuki1997-1,young1997}.
\Textcite{quirk1992} showed EC has a higher mean firing rate than hippocampus which is not caused by short bursts and is thus compatible with the sustained maintenance of neural firing.
Also, the electrophysiological properties of the EC support integration \parencite{egorov2002}.
These findings are, however, based on the intrinsic cell properties, whereas the CUE model uses recurrent connectivity for integration instead.
Note, that the context network does contain integration ensembles that maintain the context signal over the timespan of seconds.

While the EC firing is modulated to some degree by the position, this coding is more noisy than in the hippocampus \parencite{quirk1992}.
This indicates that EC codes for additional information.
\Textcite{Frank2000} have shown that superficial EC employs retrospective coding, that it differentiates visits of the same position by the history leading up to that visit.
This is consistent with a context signal encoding the history of items leading up to the current item.

The other major component to map to brain structures are the learning and retrieval of associations in the $\mft$ and $\mtf$ matrices.
\Textcite{howard2005} stated that the $\mft$ matrix might not be implemented by a single anatomical region due to its complicated structure.
However, the updating equation for $\mft$ in the CUE model has been simplified which makes the correspondence to a single region more plausible.
The learning of new associations in these matrices is attributed to the hippocampus by \textcite{howard2005}.
TODO more details.
Though, they do not consider the retrieval to be dependent on the hippocampus because TODO\@.

TODO because of striatal association learning?
Simple associations of same modality might be learned there, but cross-modality and integration (i.e.\ of item and context) might happen in hippocampus.
But striatum seems to be mostly for reward associations like in Pavlovian conditioning and other reward/punishment learning.

The CUE model provides a more detailed description of the updating of the association matrices due to the neural implementation by means of the AML\@.
In the learning network we get recurrent connectivity between TODO\@.
This is analogous to recurrent connectivity in the CA3 region of hippocampus.
TODO additional input via different pathways.
Moreover, the AML highlighted the need to incorporate the decoder matrix $\mdec\Tr$ into the connections.
While I have shown it might be plausible that this connectivity is learned, the matrix could be reduced to the identity if the input ensemble were to provide orthogonalized inputs.
The dentate gyrus, providing input to CA3 is commonly assumed to perform such orthogonalization given its large neuron count, sparse firing, and neurogenesis (TODO refs).
Thus, while the CUE model does not explicitly model the dentate gyrus (which is a significant research problem in itself), the learning rule used at least provides a principled reason for its existence.

Further evidence for this neuroanatomical mapping can be obtained from the connectivity between hippocampus and EC\@.
The superficial EC provides input to hippocampus, but does not receive direct input for hippocampus (TODO ref).
In contrast to that, the deep layers of EC receive input from hippocampus and might be relevant for recall, especially the recall of pre-experimental contex.
This is consistent with the connectivity in the model where the context network projects to the association matrix learning network attributed to hippocampus.
The learning network for $\mft$ also projects back to an ensemble recalling the prior context before it gets combined in a different ensemble.


TODO did they actually say anything about context-to-item matrix?

more going on in MTL, eg place cells
