\documentclass[
    paper=letter,
    12pt,
    titlepage,
    twoside,
    final,
    BCOR=10mm,
    DIV=9,
    listof=totoc]{scrbook}

\usepackage{fontspec}
\usepackage{microtype}
\usepackage{amsmath,amssymb,amstext,amsthm}
\usepackage{bm}
\usepackage{commath}
\usepackage[backend=biber,style=authoryear]{biblatex}
\usepackage{graphicx}
\usepackage{siunitx}
\usepackage{subcaption}
\usepackage{tikz}
\usepackage{upgreek}

\usepackage[pagebackref=false]{hyperref}
\hypersetup{
    plainpages=false,       % needed if Roman numbers in frontpages
    unicode=false,          % non-Latin characters in Acrobat’s bookmarks
    pdftitle={An Integrated Model of Context, Short-Term, and Long-Term Memory},
    pdfauthor={Jan Gosmann},
%    pdfsubject={Subject},  % subject: CHANGE THIS TEXT! and uncomment this line
%    pdfkeywords={keyword1} {key2} {key3}, % list of keywords, and uncomment this line if desired
    pdfnewwindow=true,      % links in new window
    hidelinks,
}
\usepackage[capitalise]{cleveref}
\usepackage[automake,toc,nomain,nogroupskip,nonumberlist]{glossaries} % Exception to the rule of hyperref being the last add-on package
\usepackage{glossary-mcols}
\glssetwidest{$\mat M^{\ped{epis}}$}
\setglossarystyle{mcolalttree}


\usetikzlibrary{graphs}
\usetikzlibrary{nef}
\usetikzlibrary{quotes}

\defaultfontfeatures[Garamond]{
    UprightFont={EB Garamond},
    FontFace={bx}{n}{Font={EB Garamond},Color=red}
}
\defaultfontfeatures[Lato]{
    UprightFont={Lato Regular},
    BoldFont={Lato Bold},
    Scale=MatchLowercase
}
\newfontfamily\lato{Lato}
\newfontfamily\gmdlin{Garamond}[Numbers=Lining]

\setmainfont{Garamond}
\setsansfont{Lato}
\setkomafont{chapterentry}{\normalfont}
\sisetup{mode=math,number-math-rm=\gmdlin}

\addbibresource{references.bib}
\addbibresource{proposal-references.bib}

\newcommand{\mat}[1]{\bm{#1}}
\newcommand{\vc}[1]{\bm{#1}}
\newcommand{\ped}[1]{{\mathrm{#1}}}
\newcommand{\Tr}{^{\top}}
\newcommand{\spc}[1]{\textsc{#1}}
\newcommand{\spv}[1]{\ensuremath{\bm{#1}}}

\DeclareMathOperator*{\argmax}{arg\,max}

\newcommand{\pop}[1]{{\lato #1}}
\newcommand{\nin}[1]{{\lato #1}}

\tikzset{net/.append style={font={\lato}},ext/.append style={font={\lato}}}

\newtheorem{defn}{Definition}
\newtheorem{corollary}{Corollary}

% Define Glossary terms (This is properly done here, in the preamble. Could be \input{} from a file...)

\newglossary*{symbols}{List of Symbols}
\makeglossaries

% List of Symbols
\newcommand{\addsym}[4]{\newglossaryentry{#2}{sort={#2},type=symbols,name={\ensuremath{#3}},description={#4}}\glsadd{#2}\newcommand{#1}{\ensuremath{#3}}}
\addsym{\tcmitem}{f}{\vc{f}}{TCM item vector}
\addsym{\tcmitemin}{fin}{\vc{f}^{\ped{IN}}}{recalled TCM item vector}
\addsym{\ctx}{c}{\vc{c}}{TCM context vector}
\addsym{\ctxin}{cin}{\vc{c}^\ped{IN}}{TCM input context vector}
\addsym{\mft}{Mfc}{\mat{M}^{\ped{FC}}}{TCM item to context matrix}
\addsym{\mtf}{Mcf}{\mat{M}^{\ped{CF}}}{TCM context to item matrix}
\addsym{\tcmbeta}{beta}{\beta}{TCM beta parameter}
\addsym{\Heavi}{heavi}{\Theta}{Heaviside function}
\addsym{\krond}{kronecker delta}{\updelta}{Kronecker delta}
\addsym{\enc}{e}{\vc{e}}{NEF encoding vector}
\addsym{\dec}{dec}{\vc{d}}{NEF decoding vector}
\addsym{\menc}{Enc}{\mat{E}}{NEF encoder matrix}
\addsym{\mdec}{D}{\mat{D}}{NEF decoder matrix}
\addsym{\weights}{W}{\mat{W}}{synaptic weight matrix}
\addsym{\act}{activity}{a}{neural spiking activity}
\addsym{\gain}{alpha}{\alpha}{NEF neuron gain}
\addsym{\jbias}{jbias}{J^{\mathrm{bias}}}{NEF bias input current}
\addsym{\nl}{G}{G}{neuron nonlinearity}
\addsym{\repspace}{Xc}{\mathcal{X}}{NEF representational space}
\addsym{\dims}{dim}{d}{dimensionality}
\addsym{\radius}{r}{r}{NEF representational radius}
\addsym{\syn}{h}{h}{synaptic filter}
\addsym{\syntau}{tausyn}{\tau_\ped{syn}}{synaptic time constant}
\addsym{\evalp}{x}{\vc{x}}{evaluation point}
\addsym{\actmat}{A}{\mat{A}}{activity matrix}
\addsym{\evalpmat}{X}{\mat{X}}{evaluation point matrix}
\addsym{\imat}{I}{\mat{I}}{identity matrix}
\addsym{\superpos}{S}{\mathcal{S}}{superposition operator}
\addsym{\simmeasure}{s}{s}{similarity measure}
\addsym{\bind}{B}{\mathcal{B}}{binding operator}
\addsym{\bid}{i}{\vc{i}}{identity under binding}
\addsym{\fourier}{F}{\mathcal{F}}{discrete Fourier transform}
\addsym{\fouriermat}{TF}{\mat{T}_{\mathcal{F}}}{discrete Fourier transform matrix}
\addsym{\bzero}{n}{\vc{n}}{absorbing element}
\addsym{\iu}{iu}{\symup{i}}{imaginary unit}
\addsym{\vtb}{BV}{\mathcal{B}_{\mathrm{V}}}{vector-derived transformation binding operator}
\addsym{\ndist}{N}{\mathcal{N}}{normal distribution}
\addsym{\expected}{Ex}{\symbb{E}}{expected value}
\addsym{\err}{Err}{E}{error}
\addsym{\errtotal}{Etot}{E_\ped{tot}}{total error}
\addsym{\errnoise}{En}{E_\ped{n}}{noise error}
\addsym{\errdist}{Ed}{E_\ped{d}}{distortion error}
\addsym{\bO}{O}{O}{big O notation}
\addsym{\sa}{Omega}{\Omega}{solid angle}
\addsym{\gammafn}{Gamma}{\Gamma}{gamma function}
\addsym{\betafn}{Beta}{\symup{B}}{beta function}
\addsym{\ballvol}{V}{V_{\!d}}{volume $d$-dimensional hyperball}
\addsym{\csdist}{CS}{\mathcal{CS}}{PDF of cosine similarity}
\addsym{\pcs}{pCS}{p_{\csdist}}{cosine similarity distribution pdf}
\addsym{\osestm}{mstm}{\vc{m^{\ped{stm}}}}{OSE short-term memory trace}
    \addsym{\oseepis}{mepis}{\vc{m^{\ped{epis}}}}{OSE episodic memory trace}
\addsym{\osestmdecay}{gamma}{\gamma}{OSE short term decay}
\addsym{\oseepisscale}{rho}{\rho}{OSE episodic decay}
\addsym{\tauref}{tauref}{\tau_{\ped{ref}}}{refractory period}
\addsym{\taurc}{taurc}{\tau_{RC}}{membrane time constant}
\addsym{\drate}{phi}{\phi}{distractor rate}
\addsym{\reg}{lambda}{\lambda}{regularization scale}
\addsym{\posnum}{natural numbers}{\mathbb{N}_{>0}}{positive (excluding zero) natural numbers}
\addsym{\minev}{mu}{\mu}{null choice bias in recall}
\addsym{\recnoise}{sigma}{\sigma}{standard deviation of noise in recall}


\begin{document}

\pagenumbering{roman}

% The contents of the title page are specified in the "titlepage"
% environment.
\begin{titlepage}
        \begin{center}
        \vspace*{1.0cm}

        \Huge
        {\textsc{An Integrated Model of Context, Short-Term, and Long-Term Memory}}

        \vspace*{1.0cm}

        \normalsize
        by \\

        \vspace*{1.0cm}

        \Large
        Jan Gosmann \\

        \vspace*{3.0cm}

        \normalsize
        A thesis \\
        presented to the University of Waterloo \\ 
        in fulfillment of the \\
        thesis requirement for the degree of \\
        Doctor of Philosophy \\
        in \\
        Systems Design Engineering \\

        \vspace*{2.0cm}

        Waterloo, Ontario, Canada, 2018 \\

        \vspace*{1.0cm}

        \copyright\ Jan Gosmann 2018 \\
        \end{center}
\end{titlepage}

% The rest of the front pages should contain no headers and be numbered using Roman numerals starting with `ii'
\pagestyle{plain}
\setcounter{page}{2}

\cleardoublepage % Ends the current page and causes all figures and tables that have so far appeared in the input to be printed.
% In a two-sided printing style, it also makes the next page a right-hand (odd-numbered) page, producing a blank page if necessary.
 


% D E C L A R A T I O N   P A G E
% -------------------------------
  % The following is a sample Delaration Page as provided by the GSO
  % December 13th, 2006.  It is designed for an electronic thesis.
  \noindent
  This thesis consists of material all of which I authored or co-authored: see Statement of Contributions included in the thesis.
  This is a true copy of the thesis, including any required final revisions, as accepted by my examiners.

  \bigskip
  
  \noindent
I understand that my thesis may be made electronically available to the public.

\cleardoublepage

\begin{center}\textsc{Statement of Contributions}\end{center}
\Cref{sec:recall} (excluding \cref{sec:recall-net}) paraphrases a conference submission that was co-authored by myself, a PhD student Aaron R.\ Voelker, and my supervisor, Dr.\ Chris Eliasmith \parencite{jangosmann2017}.
\Cref{sec:apdx-wta} is a verbatim copy from the supplementary material accompanying the same paper.
I implemented the network models, performed the benchmarks, and data analysis.
Mr.~Voelker contributed the mathematical analyses.

\cleardoublepage

% A B S T R A C T
% ---------------

\begin{center}\textsc{Abstract}\end{center}
I present the context-unified encoding (CUE) model, a large-scale spiking neural network model of human memory.
It combines and integrates activity-based short-term memory with weight-based long-term memory.
The implementation with spiking neurons ensures biological plausibility and allows for predicitions on the neural level.
At the same time, the model produces behavioural outputs that have been matched to human data from serial and free recall experiments.
In particular, well-known results such as primacy, recency, transposition error gradients, and forward recall bias have been reproduced with good quantitative matches.
Additionally, the model accounts for the effects of the acetylcholine antagonist scopolamine, and the Hebb repetition effect.

The CUE model combines and extends the ordinal serial encoding (OSE) model, a spiking neuron model of short-term memory, and the temporal context model (TCM), a mathematical model of free recall.
To the former, a neural mechanism for tracking the list position is added.
The latter is converted into a spiking neural network under considerations of the main features and simplification of equations where appropriate.
Previous models of the recall process in the TCM are replaced by a new independent accumulator recall process that is more suited to the integration into a large-scale network.
To implement the modification of the required association matrices, a novel learning rule, the association matrix learning rule (AML), is derived that allows for one-shot learning without catastrophic forgetting.
Its biological plausibility is discussed and it is shown that it accounts for changes in neural firing observed in human recordings from an association learning experiment.
Furthermore, I discuss a recent proposal of an optimal fuzzy temporal memory as replacement for the TCM context signal and show it to be likely to require more neurons than there are in the human brain.

To construct the CUE model, I have used the Neural Enineering Framework (NEF) and Semantic Pointer Architecture (SPA).
This thesis makes novel contributions to both.
I propose to distribute NEF intercepts according to the distribution of cosine similarities of random uniformly distributed unit vectors.
This leads to a uniform distribution of active neurons and reduces the error introduced by spiking noise considerably in high-dimensional neuronal representations.
It improves the asymptotic scaling of the noise error with dimensions $\dims$ from $\bO(\dims)$ to $\bO\big(\dims^{3/4}\big)$.
These results are applied to achieve improved Semantic Pointer representations in neural networks are on par with or better than previous methods of optimizing neural representations for the Semantic Pointer Architecture.
Furthermore, the vector-derived transformation binding (VTB) is investigated as an alternative to circular convolution in the SPA, with promising results.


\cleardoublepage

% A C K N O W L E D G E M E N T S
% -------------------------------

\begin{center}\textsc{Acknowledgements}\end{center}
I would like to thank my supervisor, Chris Eliasmith, for his continued support and guidance.
I am also grateful to my colleagues at the Centre for Theoretical Neuroscience, in particular Terrance Stewart and Aaron Voelker, for their invaluable knowledge and helpful discussions.
The research in this thesis would have not been possible without the excellent and continuously improving work of the Nengo development team, spearheaded by Trevor Bekolay.
This research was also enabled in part by support provided by SHARCNET\footnote{\href{https://www.sharcnet.ca}{www.sharcnet.ca}} and Compute Canada\footnote{\href{https://www.computecanada.ca}{www.computecanada.ca}}.
Many CPU cycles were burned on their hardware, simulating iterations of the CUE model.
I also would like to acknowledge the financial support for this research by the Canada Research Chairs program, the NSERC Discovery grant 261453, the Air Force Office of Scientific Research grant FA8655-13-1-3084, CFI, and OIT\@.  % chktex 8

A special mention is deserved by the Interdisciplinary College, a unique annual spring-school at the Lake Möhne in Germany.
It is the place where I met Chris Eliasmith for the first time and ultimately made me pursue this research direction.

Finally, I would like to thank all the people I climbed with over the years as they helped me to stay sane during the work on this thesis.


\cleardoublepage

% D E D I C A T I O N
% -------------------

\begin{center}\textsc{Dedication}\end{center}

This is dedicated to my parents, for everything they gave me.

\cleardoublepage

% T A B L E   O F   C O N T E N T S
% ---------------------------------
\renewcommand\contentsname{Table of Contents}
\tableofcontents
\cleardoublepage

% L I S T   O F   T A B L E S
% ---------------------------
\listoftables
\cleardoublepage

% L I S T   O F   F I G U R E S
% -----------------------------
\listoffigures
\cleardoublepage

% GLOSSARIES (Lists of definitions, abbreviations, symbols, etc. provided by the glossaries-extra package)
% -----------------------------
\printglossaries
\cleardoublepage

\pagestyle{headings}
% Change page numbering back to Arabic numerals
\pagenumbering{arabic}
 
\setchapterpreamble[u]{\dictum[Monkey in Kubo and the Two Strings]{If you have no memory, how can you be certain of anything?}\bigskip}
\chapter{Introduction}

Memory in its different forms is an important aspect of human and animal cognition.
It allows such agents to return to previously visited water and food locations \parencite{vorhees2014}, allows them to act more optimally in situations similar to prior experiences, or allows them to avoid dangerous situations.
In this way, memory allows for adaptation to a complex environment on a faster time scale than genetic selection.
This is especially important in unstable and changing environments.
In humans, memory is also important for forming social relationships, a shared culture, and even a functioning society.
For example, strategies in the repeated prisoner's
dilemma\footnote{The scenario of the prisoner's dilemma poses that two subjects are arrested, but the evidence is not sufficient to convict them on the principle charge. If both subjects stay silent, they each get a prison sentence of one year. Either subject can decide to betray the other in which case they get no prison sentence while the other subject gets a sentence of three years. However, if both subjects decide to betray each other, each gets a prison sentence of two years. The subjects are not allowed to communicate, but in the repeated prisoner's dilemma they can penalize each other for previous decisions.} depend on working memory capacity \parencite{milinski1998}.
Moreover, memory contributes to our individual sense of self \parencite{prebble2013}.

In addition to these functional reasons, an addressable memory is also important from a computational perspective.
It allows for a more compact implementation of many computational processes than a pure state machine could achieve \parencite{gallistel2009}.
Overall, the storage of information over time is a fundamental requirement in many cognitive systems.

Accordingly, there is a long history of memory research.
The well-known primacy and recency effect (discussed in more detail in \cref{sec:exp-findings}) has already been described by \textcite{Robinson1926}.
Nevertheless, there are still many open questions regarding memory function and implementation in the brain.
One important challenge that memory systems have to solve is the so-called \emph{stability-plasticity dilemma} \parencite{Abraham2005}.
On the one hand, there is a need to quickly form new memories, sometimes even with a single exposure (also known as \emph{one-shot learning}).
On the other hand, such high plasticity can easily lead to overwriting of old memories, rendering memory systems useless.
\Textcite{Buzsaki1989} and others have proposed multi-stage memory models to address this dilemma, where different memory systems are in place for different timescales with different levels of plasticity.

Despite this, much experimental research and modeling treat different memory systems as isolated.
This simplifies the analysis of results to an extent, but to get a general understanding of memory as a whole our characterization of memory phenomena needs to be integrated at some point.
Furthermore, many models of memory or learning focus on either small scale neural changes without any direct connection to behaviour \parencite[e.g.,][]{Levy2005}, or on idealized behaviour
described with mathematical equations, but no solid grounding in biological plausibility \parencite[e.g.,][]{Milford2004}.
So it is not only important to integrate our understanding of memory systems, but also to bridge the gap from neural mechanisms to behaviour.
With this thesis I attempt to advance our understanding of memory in these ways by proposing a model integrating short- and long-term memory, thus modeling their interaction.
Moreover, I implement the model as a spiking neural network to ensure biological plausibility, while at the same time matching behavioural data.

While this work is still limited with regard to the variety of memory systems covered, there are promising long-term prospects for a better understanding of human memory.
Many forms of memory loss are currently untreatable.
However, diseases associated with aging, like Alzheimer's, significantly impair the function of memory systems and are getting more common as our life-span increases due to medical and nutritional advances.
A better understanding of memory might allow us to devise better treatments, or even stop and reverse the memory detoriation.
A potential route to such treatments is through memory implants of the kind already demonstrated in rats by \textcite{Berger2011}.
For these sort of implants, an understanding of how memories are encoded is at least helpful, if not crucial.

These are certainly strong motivators for research into memory, but it is also worthwhile to advance our general understanding of how the human brain works.
An important step in testing our current understanding of the brain is the Spaun model \parencite{Eliasmith2012}.
Spaun is a spiking neural network of 2.5 million neurons, grounded in biology, that can perform 8 different tasks.
It gets sensory input (low resolution black and white symbols) and produces a behavioural motor output with a simulated arm.
The number of tasks it can perform demonstrates that it is not a specialized model for a single task, but can switch between different tasks.
Obviously, Spaun is still much simpler (and has many fewer neurons) than an actual brain, and is still far from capturing the complexity of a real brain.
But it incorporates a number of qualitative key aspects, such as the ability to switch task, making human-like errors, and being implemented in an anatomically constrained spiking neural network.
However, one key aspect is still missing.
While it can perform working memory dependent tasks and has simple reinforcement learning, it is missing a declarative and episodic long-term memory.
The work in this thesis can also be seen as a step toward implementing a model of such memory for the future integration in even larger scale models, combining further key aspects of cognition within a single model.


\section{Behavioural characterization of memory}
Memory systems have been characterized in different and sometimes contradictory ways.
However, one commonly used distinction is made along two orthogonal dimensions:
by timescale and by type of information stored.
On the timescale axis one can differentiate between \emph{short-term memory (STM)} lasting for up to tens of seconds and \emph{long-term memory (LTM)} which can last between hours to decades \parencite{chaudhuri2016}.
However, there is no single agreed upon definition of the exact boundaries.
Sometimes the term \emph{very long-term memory (vLTM)} is used in addition to STM and LTM for memory that exceeds LTM \parencite{solso1998}.

Another use of these terms defines STM as memory being maintained through sustained neural firing, while LTM and vLTM are realized by synaptic-weight changes.
The volatility of sustained neural firing explains why in this use of terms, STM usually corresponds to information maintained on shorter timescales than in LTM\@.
Moreover, the term working memory (WM) is often used to refer to active representations that allow direct mental manipulation.
In this thesis, I will use the terms STM and LTM primarily to distinguish between activity-based and weight-based forms of memory.

When classifying memory by type of information, a representation as a tree structure can be helpful (\cref{fig:memtypes}).
On the highest level, we have the distinction between implicit and explicit memory.
Implicit memory does not allow for conscious access.
A typical example is procedural or motor memory like the exact muscle contractions for keeping balance when riding a bike.
Explicit memory allows for conscious access and is further subdivided into declarative and episodic memory.
The declarative memory allows us to store and reproduce facts like the birthday of a friend, whereas the episodic memory provides a recollection of life experiences.
Here, I will focus on declarative memory as this type of memory has been well-studied in many memory experiments.
\begin{figure}
    \centering
    \begin{tikzpicture}
        [level 1/.style={sibling distance=7cm},
        level 2/.style={sibling distance=2.35cm}]
        \node {Memory}
        child {node {\strut Explict} child {node {\strut Episodic}} child {node 
                {\strut Declarative}} child {node {\strut Semantic}}}
        child {node {\strut Implict} child {node {\strut Priming}} child {node 
                {\strut Conditioning}} child {node {\strut Procedural}}};
    \end{tikzpicture}
    \caption{Categorization of memory by type of stored 
        information.}\label{fig:memtypes}
\end{figure}


\section{Experimental findings in memory research}\label{sec:exp-findings}
Many memory experiments require the participants to memorize lists of words and recall them afterwards.
Usually an experiment will fall into one of two categories depending on how the subjects are required to recall the list items: in free recall experiments, the subjects are free to recall the items in any order; in serial recall experiments, the subjects are required to recall the items in the order they were presented.

The hallmark finding in memory research, especially for serial recall experiments, are the \emph{primacy} and \emph{recency} effects.
Subjects tend to remember the start of list (primacy) and the end of list (recency) better (\cref{fig:exp-serial-pos}).
Furthermore, if subjects in a serial recall experiment recall an item at the wrong position, a so-called \emph{transposition}, it is more likely to be an item that was close in the list than an item several positions away.
\begin{figure}
    \centering
    \includegraphics{figures/exp-serial-pos}
    \caption[Immediate serial recall position curve.]{Serial position curve of a immediate serial recall experiment with a 10 item list. Data reproduced from \textcite{Jahnke1968} with \SI{95}{\percent} confidence intervals.}\label{fig:exp-serial-pos}
\end{figure}

In free recall experiments, a recency effect is observed as well.
This does not only show in the serial position curve, but participants often start out with recalling the last item first.
This can be measured by the \emph{probability of first recall} (\cref{fig:exp-free-recall}).
Another important aspect of free recall is captured by the conditional response probability (CRP).
It gives the probability for the difference (the \emph{lag}) in serial positions of two recalled items.
It is peaked around zero, indicating that items in proximity in the learned list tend to be recalled together, while jumps to remote items are rarer.
This is known as \emph{contiguity} or the \emph{lag-recency effect}.
The CRP curve also has a characteristic asymmetry which shows the bias for forward (opposed to backward) recall.
\begin{figure}
    \centering
    \includegraphics{figures/exp-free-recall}
    \caption[Free recall probablitiy of first recall and CRP.]{Data from free recall experiments with 12 item lists. (a) Probability of first recall. (b) Conditional response probability. All data from \textcite{Howard1999} with \SI{95}{\percent} confidence intervals.}\label{fig:exp-free-recall}
\end{figure}

By introducing a delay filled with a distractor task, referred to as \emph{delayed (free) recall}, both the probability of first recall and the CRP curve will become much flatter.
This effect cannot solely be attributed to (suppressed) rehearsal effects as introducing an equally long distractor interval in-between the list items, known as \emph{continuous distractor (free) recall}, partially restores the recency effect in the probability of first recall.

While there are many other experimental findings in memory research, there are two more findings that I will focus on in this thesis.
The first is the differential effects of the acetylcholine antagonist scopolamine on encoding and recall \parencite{ghoneim1975}.
Recall performance is normal when scopolamine is injected after the presentation phase, but the number of successful recalls is considerably lower when the injection is done before the presentation phase.
The number of successfully recalled items in the latter case is around the short-term memory span, which might show an effect purely on LTM, but not STM\@.
This experiment is of interest as a neural network model is more suited to modeling such drug effects than typical mathematical models.

Second, the \emph{Hebb repetition effect} is the observation that in repeated immediate recall experiments, the recall accuracy of a repeated list (typically every third list) will increase with repetitions \parencite{Hebb1961}.
Whether the test subject has to become consciously aware of the repetition is debated \parencite{Stadler1993}.
It is of interest in this thesis that the Hebb repetition effect has been regarded as useful in testing the interaction of STM and LTM, or as \textcite{Burgess2005} phrase it, this effect is  a ``powerful vehicle for developing and testing models of the relationship between STM and LTM''.


\section{Neuroanatomy of memory}
Short-term memory is attributed to cortical brain regions, but there is no single region that is the unique locus of STM\@.
Sensory short-term memory is distributed across the corresponding cortical sensory and related brain regions \parencite{zelano2009,todd2004,baldo2012}.
For example, auditory memory can be found in the temporal cortex close to speech processing areas, whereas visual short-term memory is located in parietal cortex.
Furthermore, multimodal integration and manipulation in working memory has been found to lead to increased activation in prefrontal cortex \parencite{rypma1999}.

Compared to short-term memory, the formation of new long-term memories is more localized.
In particular, the hippocampus (HC) has been implicated in the acquisition of new declarative and episodic memories \parencite{eichenbaum2001-1}.
A finding originally derived from patients with hippocampal lesions, most famously HM who got his hippocampus (but also other brain regions) removed due to severe epilepsy \parencite{penfield1958,scoville1957-1,squire2009}.
The hippocampus is named for its sea horse shaped structure.
It is located in the medial temporal lobe (\cref{fig:hc}) and lies near the subiculum and entorhinal cortex (EC).
The hippocampus is further divided into substructures CA3, CA1, and the dentate gyrus (DG) by its cytoarchitectural structure.
While the CA3 and CA1 regions consist mainly of pyramidal neurons and about \SI{10}{\percent} GABAergic, inhibitory interneurons \parencite{Freund1996}, granule cells are the main constituent of the dentate gyrus.
Furthermore, the dentate gyrus has a large number of cells that exhibit sparse activity compared to CA3.
In humans the cell count of the DG is about twice as high as in CA3, whereas in rats DG has about five times more cells than the CA3 region.
\begin{figure}
    \begin{addmargin*}[0mm]{-70.16pt}
        \hfill
        \subcaptionbox{}{\includegraphics[width=2.45in]{figures/Gray739-emphasizing-hippocampus}}
        \hfill
        \subcaptionbox{}{\includegraphics{tikz/hc}}
        \hfill
        \caption[Hippocampal anatomy.]{(a) Location of the hippocampus in the human brain. (b) Connectivity between entorhinal cortex (EC) and hippocampal subregions: dentate gyrus (DG), CA1, CA3.}\label{fig:hc}
    \end{addmargin*}
\end{figure}

The connectivity to, from, and within hippocampus is exceptionally well known.
There are two major pathways from entorhinal cortex.
The \emph{direct pathway} originates in layer III of the entorhinal cortex and targets the distal apical dendrites of CA1 neurons.
The \emph{trisynaptic pathway} originates in layer II and leads via the dentate gyrus, and CA3 also to CA1, but targeting the proximal dendrites (a signal thus crosses three synapses).
The connections from the dentate gyrus to the CA3 region is termed the \emph{mossy fiber pathway}.
Each of these mossy fibers innervates about 15 neurons \parencite{Claiborne1986} which results in each CA3 pyramidal neuron receiving input from about 50 to 90 granule cells \parencite[230]{Squire1989}.
The final bundle of connections from CA3 to CA1 is known as the \emph{Schaffer collateral pathway}.
Further major hippocampal connections exist from the entorhinal cortex to CA3, mostly targeting the same cells as the connection from dentate gyrus \parencite{Paxinos2014}, and recurrent connections in CA3.
However, a single cell in CA3 is not recurrently innervated by more than 5\% of all CA3 cells \parencite[231]{Squire1989}.

Given these neuroanatomical properties, some of the hippocampal regions have been assumed to be involved in specific tasks.
The sparse coding and large cell count of the dentate gyrus led to the belief that it is responsible for pattern separation \parencite{Rolls2013}.
The CA3 region with its recurrent connectivity could be involved in pattern completion or forward predictions \parencite{Guzowski2004,Leutgeb2007,Rolls2013}.
Additional evidence for this comes from unimpaired recognition memory after lesioning the CA3 to CA1 connections despite impairing selective recall \parencite{Brun2002}. 

A review of hippocampal structures would not be complete without mentioning hippocampal place cells and entorhinal grid cells \parencite{hafting2005}, first discovered in rats, but existent also in other animals \parencite{buzsaki2013}.
A grid cell fires when the animal is in locations arranged in a regular hexagonal grid in an environment.
Place cells are similar, but fire only for one specific location in an environment and remap between environments.
While these cells are involved in navigational tasks, a connection to memory is possible \parencite{buzsaki2013}.
However, paying further attention to these cells is out of the scope of this thesis.

Closely related to grid and place cells is the hippocampal theta oscillation of \SIrange{4}{8}{\hertz} found in rats.
The spiking of place cells during a theta cycle is timed according to the distance from corresponding land marks, a phenomen termed \emph{phase precession} \parencite{okeefe1993}.
The coupling of the theta and gamma bands has also been shown to be important in the learning of item-context associations \parencite{tort2009}.
Despite some debate, there is evidence for similar, but slower ($<\SI{4}{\hertz}$), hippocampal oscillations in humans related to episodic memory encoding \parencite{lega2012}.
However, their functional relevance is not entirely clear at this point.

Finally, most commonly during sleep, but also in immobile rats, sharp waves (SPWs) are observed in the rat hippocampus \parencite{chrobak1994,girardeau2009}.
These have been implicated to be involved in memory consolidation from hippocampus to the neocortex, a process not covered in this thesis.


\section{Memory models}
Different memory models can be roughly sorted into three classes.
Conceptual models describe different components and processes of memory and their interaction.
Often they are presented as box and arrow diagrams.
They do not allow for a precise mechanistic explanation or quantitative predictions.
Mathematical models address this by providing exact equations that can be evaluated to obtain quantitative predictions.
However, they do not explain the neural implementation and thus are not constrained to biologically plausible mechanisms.
Finally, connectionist models use neural networks with varying degrees of abstraction.
As such their biological plausibility also varies.
While those models come closer to explaining neural mechanisms of memory, they less often address the behavioural data from high-level cognitive experiments.

\subsection{Conceptual models}
\Textcite{Yntema1963} presented one of the first models of memory.
They conceptualized retrieval as a search process where each item in memory is associated with tags referencing further information.
For example, a time tag would encode the time an item was observed.
The whole description, however, is more based on how a memory system could be implemented on a classical von Neumann computer and does not consider if or how those operations could be neurally implemented.

To date, the most influential conceptual organization of working memory was proposed by \textcite{Baddeley1986}.
He proposed, based on experimental data, separate stores for visual and acoustic information, termed the \emph{visuospatial sketchpad} and \emph{phonological loop} respectively.
These are controlled by a \emph{central executive}.
Furthermore, in \textcite{Baddeley2000} the model was extended with an episodic buffer for the binding of multimodal information and transfer to episodic long-term memory.
The main relevance of the model is that it informs us about the organization of (working) memory by modality; but it does less to elucidate mechanisms.

In general, conceptual models can give a high-level account and they can be evaluated with respect to their qualitative agreement to behavioural data.
However, they cannot provide us with quantitative predictions which makes a more rigorous validation difficult.
They, also, do not explain the cognitive mechanisms in detail, and an account of the neural implementation is completely out of their reach.


\subsection{Mathematical models}
Some of the weaknesses of conceptual models are addressed by mathematical models.
These models make quantitative predictions about the behavioural data and describe underlying cognitive mechanisms to a varying degree, but do not provide a neural implementation.
There is a vast number of such models and not all can be discussed here.
Thus I will focus on some of the most influential ones.

Such a list certainly contains the perturbation model for serial order by \textcite{Estes1972}, the free recall model Search of Associative Memory (SAM) by \textcite{Raaijmakers1981}, the recognition memory model Retrieving Effectively from Memory (REM) by \textcite{Shiffrin1997}, and the episodic memory model MINERVA2 by \textcite{Hintzman1988}.
The perturbation model is an early attempt to provide a mathematical framework for how remembered item positions can drift over time.
In SAM, cues are assembled in a short-term memory to retrieve associations from a long-term associative memory (the search part of the model).
In the REM, model error prone copies of feature vectors derived from the study items are stored.
A recognition probe is matched to the stored feature vectors and a likelihood ratio of the match scores being generated by an old versus a new item is calculated.
The MINERVA2 model also uses feature vectors that are stored as traces and can be probed by cues.

All of these models, and most other mathematical models, either assume item-to-item or position-to-item associations.
The primacy model by \textcite{Page1998} is worth noting because it uses a different approach.
In that model, items are activated according to a primacy gradient, but the model is agnostic as to how this gradient is generated.
This, however, also leaves that aspect underspecified.

Common to all of these models is that they do not consider biological plausibility.
Thus it is unclear whether any of the models can be implemented in a neural substrate while preserving their predictions.
Or, similarly what the limitations with regard to noise and requirements for neural resources are.
Nevertheless, mathematical modelling is an important first step in figuring out what sort of processes are worth considering for a neural implementation.
Also, there are some reoccurring ideas in these models that are useful to consider in the context of a neural model.
For example, starting with \textcite{Anderson1973} many models have used random feature vectors to represent individual items.
That approach is similar to the Semantic Pointer Architecture (SPA) presented in \cref{sec:spa} which can be implemented neurally with the methods of the Neural Engineering Framework (\cref{sec:nef}).

An especially influential model based on such random feature vectors was \mbox{TODAM2} \parencite{Murdock1993}.
It was able to fit a large body of experimental data and, in that regard, aims to be a general theory for item recognition, serial order, and associative memory.
A neural implementation could very well be possible with the NEF, but has not been attempted so far.
However, \textcite{Choo2010} pointed out that the dimensionality of the vectors in the model increases with each stored item.
Thus, the requirement for neural resources grows in an unbounded manner.
It also worth noting, that \mbox{TODAM2} does not give an account of how responses are generated.

Another useful idea, that had its origin in mathematical models, is the idea of a randomly drifting context signal that items get associated to.
\Textcite{Estes1955} presented the first model of this type and \textcite{Murdock1997} extended the \mbox{TODAM2} model in this way to explain additional data.
Two open questions in these sort of models are, (1) how is the context at an earlier time re-instantiated to start the recall;
and (2) how the context is advanced in the same way during recall as in the study phase to recall the remaining items?
As the signal drifts randomly, a memory for the context signal itself would be needed.
The OSCAR model \parencite{Brown2000} solves part of this by using a deterministic context signal that is generated from multidimensional oscillators.
This allows replay of the exact same context signal once it has been reset.
It still does not describe how the context is re-instantiated to start the recall, as this still requires knowledge of the oscillator states at begin of the study phase.

All of these context-based models cannot explain the asymmetric CRP curves.
The temporal context model \parencite{Howard2002}, however, was specifically constructed to explain these free recall data.
It also uses a context signal, but this signal is updated by the studied (and recalled) items themselves.
Each item recalls a prior context associated with that item to partially update the current context, and the updated context gets associated with the studied item.
This solves the re-instantiation problem, as the right cue can set the context to be similar to the study context to retrieve an item.
That retrieved item in turn updates the context to retrieve more items related to the updated context.
Thus, it is also available to appropriately advance the context after each recall.
The TCM model is discussed in more detail in \cref{sec:tcm}.
But it is not perfect.
It did not capture immediate free recall data involving short-term memory, even though it was presented as a single-store framework, i.e.\ a single memory for both STM and LTM\@.
This single-store assumption was criticized by \textcite{Davelaar2008} and is in contradiction to neuroimaging data \parencite{talmi2005}.

Many of these models are also vague on the exact processes of recall.
Though, the ACT-R model of serial recall by \textcite{Anderson1997}, in which items are associated with their serial positions, describes detailed steps necessary for recall.
Unfortunately, it is vague on the exact storage mechanism of items.

Recently, \textcite{shankar2013} proposed an \emph{optimally fuzzy temporal memory} that is less motivated by behavioural data, but more by a mathematical optimal and scale-free storage of a time-varying signal.
Nevertheless, it has been proposed to model the coding in hippocampus \parencite{howard2014-1}.
In this framework, the input signal is stored by a bank of leaky accumulators with different time constants.
This corresponds to a Laplace transform of the signal.
To decode the history of the input signal, a linear operator approximating the inverse Laplace transform is used.
I discuss this model in more detail in \cref{sec:fuzzymem}, and demonstrate that it is highly sensitive to noise, which is prohibitive to a neural implementation.


\subsection{Connectionist models}
Compared to the vast number of mathematical models, there are many fewer connectionist models, although most of those that exist try to ensure more biological realism.
Many of these models focus on reproducing low-level findings in the hippocampus, such as sequence compression in replay \parencite{Levy2005} or place cells \parencite{Milford2004}.
The model presented by \textcite{Hasselmo2012}, might very well be the most comprehensive hippocampus model to date, describing the storage of episodic memories as a spatial trajectory.
It addresses experimental data on place cells and the theta rhythm.
A very recent model by \textcite{yu2017}, constructs a three-layer spiking neural network that is able to encode a sequence over several iterations and replay it during a cycle of the theta rhythm.
These models, however, still leave a large gap to high-level cognitive behaviour as modelled by mathematical models.
In addition, the \textcite{Hasselmo2012} relies on hypothetical ``arc length'' cells to disambiguate memories \parencite[cp.][]{Robins2014}.

Nevertheless, there are some connectionist models that try to reduce this gap by addressing behavioural effects with a neural network implementation.
Namely, \textcite{Burgess1992} and \textcite{Burgess1996} propose models for the articulary loop, \textcite{Norman2003} for recognition and familiarity effects, and \textcite{Botvinick2006} for immediate serial recall.
All of these models use rate based neurons as an abstraction.
While rate neuron models are a useful tool to build tractable models with a degree of biological realism, one has to be careful to not introduce biologically implausible features.
For example, the noise introduced by discrete spikes is neglected.
In particular, \textcite{Norman2003} use a $k$-winner-take-all mechanism that is hard to realize in spiking neurons as it requires a fine balance of excitation and inhibition.
Potentially even more problematic is the use of back-propagation learning in the model by \textcite{Botvinick2006}.
While learning with back-propagation is a tremendously successful technique in machine learning, it is still unclear whether biological neural networks can implement this sort of learning.
This remains true despite new techniques like feedback-alignment \parencite{lillicrap2016} and related work \parencite{bengio2015}, that might eventually provide biological plausible implementations.

To the best of my knowledge there are only two memory-related models that address these concerns about biological plausibility by using spiking neurons while at the same time connecting to behavioural data.
The first one is the ordinal serial encoding (OSE) model of serial recall by \textcite{Choo2010}.
As a primarily short-term memory model, it uses recurrently connected neurons to store the memory trace in neural activity.
This approach is also used to model a long-term memory component that can be attributed to hippocampal storage.
While this allows for a first approximation, a storage in synaptic weights would be more plausible for such a component.
The second model was specifically developed to model the storage of serial lists with hippocampus by \textcite{OliverTrujillo2014}.
It is able to reproduce neural data like replay and theta rhythm.
However, the length of stored lists is limited in similar fashion to the STM capacity in the OSE model, despite long term memory being certainly able to learn longer lists.
Learning longer lists in the model would require the chaining of individual lists with different contexts, but no mechanism for this has been given in the model.
Another questionable feature is the usage of a clock signal that is adjusted to speed up compressed replay.


\subsection{Summary}
Despite many existing models, a number of questions have not been sufficiently addressed \parencite[cp.][]{horwitz2008}.
To date no model demonstrates a satisfying degree of biological plausibility while at the same time addressing behavioural data from cognitive psychology.
Many models are not concerned with the interaction of short-term and long-term memory despite the importance for many fundamental effects on memory performance.
This includes neural processes for the coordination and control of the interplay of these memory components.
Finally, the recall process or reinstantiation of recall context is often not precisely explained even though it is an essential part of memory function.

I address these points with the context-unified encoding (CUE) model presented in this thesis.
It combines activity-based short-term memory with weight-based long-term memory and also specifies the required control and recall processes.
Biological plausibility is ensured by an implementation as a spiking neural network.
Despite the low-level implementation, it is validated against human, behavioural data.

This thesis is divided into two parts.
In the first part, all the basic methods required for the construction of such a large-scale spiking neural network model are developed and discussed.
This includes some secondary research objectives to improve neural representations to ultimately require fewer neurons for a higher simulation throughput.
In the second part, the methods from the first part are employed to build up the CUE model and compare the model predictions against human data.

\part{Methods}
\chapter{Modeling neurons}
The information processing in the brain is performed by neurons (TODO anatomical figure).
Despite a wide variety of different neuron types and behaviours, the typical behavior of most neurons can be described as follows.
These cells build up an electrical potential of about \SI{-70}{\milli\volt}, the resting potential, across their cell membrane with the ions in the intra- and extracellular fluid.
When the membrane potential is sufficiently depolarized (to about \SI{-50}{\milli\volt}), voltage gated ion channels will trigger a complete sudden and short-lived depolarization (typically about a \SI{1}{\milli\second}), a so called action potential or spike.
This action potential will travel along the neuron's axon and trigger the release of neurotransmitters at its synapses where it connects to other neuron's dendrites.
The released neurotransmitters will influence the ion channels of the post-synaptic neuron and either lower (inhibitory synapse) or raise (excitatory synapse) the membrane potential.
The deflection of the membrane potential depends on the strength of the synapse and the speed of the release and uptake of the neurotransmitter.
If the post-synaptic neuron receives sufficient input from other neurons, its membrane potential will be sufficiently deflected to initiate a new action potential.
By the pattern of connectivity that controls what activity in some neurons triggers activity in other neurons, the brain is able to perform the computations that lead to an animal's or human's behavior.

To computationally model neurons, it is necessary to chose a level of abstraction.
On the one hand, it is possible to create very detailed models with spatial extent (TODO ref) where individual ion channels and the propagation of electrical potentials is modeled.
On the other hand, one can use very abstract neuron models that basically consist of nodes summing their inputs and applying non-linearity using real valued stand-ins for firing rates (instead of discrete spikes).
This latter type of model is common in artificial neural networks and deep learning (TODO ref).

A widely used neuron model in computational neuroscience is the leaky integrate-and-fire (LIF) neuron model.
It is a point neuron model, thus not modeling any spatial extend.
It models a single membrane voltage described by the differential equation
\begin{equation}
    TODO \text{.}
\end{equation}
Whenever the membrane voltage reaches a certain threshold, the LIF neuron transmits a spike and its membrane voltage is reset for a refractory period of $\tauref$.
This type of neuron model allows to derive the firing rate for a given input current analytically as
\begin{equation}
    TODO \text{.}
\end{equation}

The LIF neuron model is a good choice for the undertaking of this thesis as it captures TODO of important neuron behavior.
It is detailed enough to relate parameters like the membrane time constant to the actual biological correspondents.
This allows to fix these parameters to values within the biological plausible range instead of having them as free parameters in the model that would require parameter matching and give the model additional degrees of freedom to match the data.
At the same time the model is simple enough to allow for reasonable performance when simulating large-scale models with these neurons.

\chapter{The Neural Engineering Framework}\label{sec:nef}

To construct a large-scale spiking neural network with a certain behaviour, some method for obtaining that behaviour is required.
In most cases this method will be a learning algorithm~\parencite[e.g.,][]{oreilly2006}.
However, this requires time-intensive training of the model and is often not viable for large models, complex behaviours, or models combining different behaviours.
In this work I opt to use the Neural Engineering Framework~\parencite[NEF;][]{eliasmith2003} which allows the direct construction of a spiking neural network from the mathematical equations describing the desired dynamics without the time-intensive training.
As such the final model does not provide an developmental account of how the neural network became organized or learned to perform its task.
But it provides a biological plausible explanation of how the developed brain might perform that task.
Furthermore, it allows for manipulations to test known experimental results in the model or obtain new predictions.

The NEF consists out of the three core principles for \emph{representation}, \emph{transformation}, and \emph{dynamics} in a neural network that I will introduce in this order.

\section{Representation}
Neurons within a (natural) neural network will have a preferred stimulus: they will fire most strongly for that stimulus and less strongly as the stimulus gets more dissimilar to the preferred stimulus. (TODO figure)
To capture this in a mathematical description, we can treat the stimulus as a vector $\vc x(t)$ that varies over time.
The preferred stimulus vector, that is the vector a neuron $i$ fires most strongly for, will be denoted with $\enc_i$.
The spiking activity $\act_i(t)$ of a neuron can then be described with
\begin{equation}
    \act_i(t) = \nl\!\sbr{\gain_i \langle\enc_i, \vc x(t)\rangle + \jbias_i}
\end{equation}
where $\gain_i$ is a neuron gain factor, $\jbias_i$ a bias input current, and $\nl$ the neuron nonlinearity.
The nonlinearity $\nl$ represents the neuron model and converts an input current into spikes.
Usually this will be the spiking, leaky integrate-and-fire model (LIF) discussed in \cref{sec:neurons} which provides a good trade-off of captured neuron behaviour, detail, and simulation effort.
But simpler neuron models (e.g., a rate-based LIF model, or rectified linear units) could be used, as well as much more complex neuron models like the compartmental model in \textcite{eliasmith2016} and \textcite{duggins2017c}.
The input current to the neuron is obtained from how well the stimulus aligns with the preferred stimulus as measured by the dot product.
As this alignment with the preferred stimulus ``encodes'' the stimulus into the neural representational space, $\enc_i$ is usually referred to as \emph{encoder} in the context of the NEF\@.
Furthermore, the gain factor $\gain_i$ and bias term $\jbias_i$ allow to adjust the neuron's tuning curve to experimentally observed firing rates.
However, it is also common to use higher maximal firing rates as to use fewer neurons in simulations to achieve an equal accuracy.
While this is not entirely adhering to biological constraints, in most cases NEF models behave the same when lowering the firing rates and increasing neuron numbers accordingly~\parencite[e.g.,][]{gosmann2015}.
As it is rare to have detailed information about the tuning curves in many part of the brain, those values are usually not directly set in the NEF\@.
Instead a representational space $\repspace$ is defined, usually as the $\dims$-dimensional $\dims$-hyperball with radius $\radius$.
Furthermore, for each neuron a maximum firing rate $\act_{\max}$ and an intercept point $p_i$ are sampled from random distributions.
These values are used to calculate the gain and bias so that the neuron starts firing at $p_i \cdot \enc_i$ when $\vc x$ varies along the encoder $\enc_i$ and that the maximum $\act_{\max,i}$ is not exceeded across the representational space $\repspace$.

Given a population of neurons, also called a neural \emph{ensemble} in NEF terms, how can the encoded vector $\vc x(t)$ be recovered?
First the activity or spike trains $\act_i(t)$ are convolved with a synaptic filter $\syn$ to obtain the induced post-synaptic voltage change.
Usually this will be a decaying exponential $\syn(t) = \exp(-t / \syntau)$, but other filters can be used to more precisely model the dynamics of the synapse~\parencite{voelker2017a} and even extensions to conductance-based synapses are possible~\parencite{stockel2017}.
From the filtered activity, the represented vector can be reconstructed with a linear, weighted decoding
\begin{equation}
    \hat{\vc x}(t) = \sum_i \dec_i \cdot \sbr{a_i * h}\!(t)
\end{equation}
with decoding weights $\dec_i$.

To get a good reconstruction of the represented value, the decoding weights should minimize the error
\begin{equation}
    E = \int_{\repspace} \norm{\vc x - \hat{\vc x}}^2 \dif \vc x \text{.}
\end{equation}
In general, this minimization cannot be solved analytically.
Thus, in the NEF the integral is approximated by randomly sampling $M$ \emph{evaluation points} $\evalp_k \in \repspace$.
Given the finite number of evaluation points, it becomes possible to solve for the decoding weights with a least-squares minimization.
The detailed derivation is given in \textcite[Ch.~2]{eliasmith2003}.
In short, one obtains the matrices
\begin{equation}
    \actmat = \sbr{\begin{array}{cccc}
            \act_1(\vc x_1) & \act_1(\vc x_2) & \cdots & \act_1(\vc x_M) \\
            \act_2(\vc x_1) & \act_2(\vc x_2) & \cdots & \act_2(\vc x_M) \\
            \vdots & \vdots & \ddots & \vdots \\
            \act_N(\vc x_1) & \act_N(\vc x_2) & \cdots & \act_N(\vc x_M) \\
    \end{array}}
    \ \text{and}\ 
    \evalpmat = \sbr{\begin{array}{c}
            \vc x_1 \\ \vc x_2 \\ \vdots \\ \vc x_M
    \end{array}}
\end{equation}
where one can use the steady-state activities in the activity matrix $\actmat$ which can be obtained analytically for LIF neurons.
Given these two matrices the decoding weights can be obtained with the regularized pseudo-inverse as
\begin{equation}
    \sbr{\begin{array}{c}
            \dec_1\Tr \\ \dec_2\Tr \\ \vdots \\ \dec_N\Tr
        \end{array}} = \del{\actmat \actmat\Tr + M \gamma^2 {\max(\actmat)}^2 \imat}^{-1} \actmat \evalpmat
\end{equation}
where $\gamma$ is the regularization scale (usually $\gamma = 0.1$).

The encoders and decoders do not only allow us to encode information into a neural ensembles and decode it back out, but also to transmit that information from one neural population to another.
By decoding from the pre-synaptic ensemble and encoding into the post-synaptic ensemble, the connection weights required between the two populations can be obtained as
\begin{equation}
    W_{ij} = \enc_i\Tr \dec_j \text{.}
\end{equation}


\section{Transformation}
To be useful, a neural network has to transform or compute functions on the represented information.
In the NEF, it is straight-forward to implement a given transformation in the connection weights between two ensembles.
To implement a function $f(\vc x)$, one replaces the matrix $\evalpmat$ with
\begin{equation}
    \evalpmat_{f(\vc x)} = \sbr{\begin{array}{c}
            f(\vc x_1) \\ f(\vc x_2) \\ \vdots \\ f(\vc x_M)
    \end{array}}
\end{equation}
when solving for decoders.
This corresponds to a minimization of the modified error $E_{f(\vc x)} = \int_{\repspace} \norm{f(\vc x) - \hat{\vc x}}^2 \dif \vc x$.

\Textcite[Ch.~7]{eliasmith2003} shows that a neural network constructed in this way is typically best at computing low-order polynomials.
Non-smooth or discontinuous functions might require a large number of neurons.
In some cases a better function approximation can be achieved by appropriately selecting parameters like intercepts or encoders or by changing the network structure to decompose a function in a different way.
An example is the calculation of products and has been discussed in \textcite{gosmann2015-1} and \cref{sec:thresholding} shows this for the thresholding of values.

\section{Dynamics}
The final principle of the NEF addresses dynamics.
In linear control theory a dynamical system is often described by state equations of the form
\begin{align}
    \od{\vc x}{t} &= \mat A \vc x(t) + \mat B \vc u(t) \\
    \vc y(t) &= \mat C \vc x(t) + \mat D \vc u(t)
\end{align}
where $\vc x(t)$ is the state vector, $\vc u(t)$ the input vector, and $\vc y(t)$ the output vector.
The system behaviour is determined by the dynamics matrix $\mat A$, input matrix $\mat B$, output matrix $\mat C$, and the feedthrough matrix $\mat D$.
TODO figure
In the NEF, we want to map a given dynamical system onto neural components (TODO figure).
The neuron dynamics are dominated by the synaptic filter \parencite[Appendix~F.1]{eliasmith2003} which becomes the transfer function and gives
\begin{equation}
    \vc x(t) = h(t) * \sbr{\mat A' \vc x(t) + \mat B' \vc u(t)} \text{.}
\end{equation}
With help of the Laplace transform one can obtain $\mat A'$ and $\mat B'$ from $\mat A$ and $\mat B$ as
\begin{align}
    \mat A' &= \syntau \mat A + \imat \\
    \mat B' &= \syntau \mat B  \text{.}
\end{align}
This implies that to implement a dynamical system with a neural ensemble, the input has to be multiplied by $\syntau$ to account for the synaptic filtering.
In addition, one needs to add a recurrent transformation implementing the function $f(\vc x) = \syntau \mat A \vc x + \vc x$.


\section{Simulating NEF networks}
To simulate NEF style networks, I use the Python library Nengo~\parencite{bekolay2014,sharma}.
It supports different backends, to run neural models on different hardware platforms.
For example, Nengo OCL targets GPUs with an OpenCL implementation.
While this allows better simulation performance, special case implementations are necessary for certain features.
In particular, this applies to the association matrix learning rule (see \cref{sec:aml}).
Moreover, I utilized the Sharcnet and Compute Canada high-performance clusters which typically provide more CPU resources than GPU resources.
Thus, I mostly used the Nengo reference (CPU) backend.
To still obtain sufficient simulation performance for the size of models constructed in this work, it was necessary to optimize memory organization of the internal data structures of the backend.
While the exact details are out of the scope of this thesis, they are published in \textcite{gosmann2017}.

\chapter{Basic NEF networks}
When constructing neural models with the NEF, there are certain networks that are often used in multiple places and constitute some basic building blocks in a sense.
In the following we will shortly discuss how to create an integrator, a gated memory buffer based on that integrator, an ensemble applying a threshold to a signal, and how to do multiplication in neurons.

\section{Integrator}
Integrators are important components in many NEF models because they allow to store values over some timespan in neural activity.
An integrator is described by the differential equation
\begin{equation}
    \od{\vc x(t)}{t} = \vc u(t)
\end{equation}
where $\vc u(t)$ is the external input to the integrator.
Applying principle 3 of the NEF tells us that the input has to be scaled by the synaptic time constant $\tausyn$.
Furthermore, a recurrent connection feeding the output of the integrator back to itself is needed.
To get a stable representations over a sufficient time window, it is best to use a long time constant like $\tausyn = \SI{0.1}{\second}$ which is the range measured for the TODO neurotransmitter.
Due to neural noise and distortion error, the represented value can drift over time.
Adding more neurons to the integrator will make it more stable.
TODO figures

\section{Gated memory buffer}
While the integrator enables us to store a value over time, it does not allow for particularly quick updating.
A quicker update can be achieved by adding a difference ensemble (TODO figure).
By scaling the difference with a factor the updating speed can be regulated.
However, too large values will lead to oscillations in the integrator.
Note that in this case feeding a null vector to the difference ensemble will clear out the memory instead of keeping the current value.
Thus, the input the integrator needs to be gated.
This can be done by inhibiting the neurons of the difference ensemble to keep the current value in the integrator.

\section{Thresholding ensembles}
Often one needs to apply a threshold to value, i.e.\ implement the function
\begin{equation}
    f(x) = \left\{ \begin{array}{ll}
            0 & x < 0 \\
            x & x \geq 0
        \end{array} \right.
    \text{,}
\end{equation}
or compute the Heaviside step function
\begin{equation}
    \Heavi(x) = \left\{ \begin{array}{ll}
            0 & x < 0 \\
            1 & x \geq 0
        \end{array} \right.
    \text{.}
\end{equation}
Both of these functions are non-differentiable at 0.
The Heaviside function is even discontinuous at that spot.
These properties make it problematic to implement this function with a standard NEF ensemble.
Nevertheless, a good approximations of these functions can be achieved by aligning the neuron's tuning curves according to the shape of these functions.

Instead of choosing encoders randomly as $-1$ and $1$, all encoders are set to $1$ and all intercepts are chosen from $x \in [0; 1]$.
Choosing the intercept distribution of this interval appropriately can further increase the accuracy.
An exponential distribution that clusters intercepts close to 0 performs best.
Note that this is even better than setting all intercepts to 0 as this gives more variation in the tuning curves.
The uniform distribution often does not produce intercepts close enough to the threshold value which leads to an increased effective threshold.

TODO figures

\section{Product}
A product of two scalar numbers $x$ and $y$ could be computed by feeding them into separate dimensions of a two-dimensional ensemble and decoding out the product.
A \SI{37}{\percent} more accurate implementation (with the same number of neurons) is, however, possible (TODO ref) by rewriting the product with squares as
\begin{equation}
    xy = \frac{1}{4}\del{x^2 + 2xy + y^2} - \frac{1}{4}\del{x^2 - 2xy + y^2} = \frac{1}{4}\del{x + y}^2 - \frac{1}{4}\del{x - y}^2 \text{.}
\end{equation}
The neural implementation of this equation is straight-forward (TODO figure).

Multiple scalar product networks can be combined to compute element-wise vector products.
By summing across those element-wise products a dot product can be computed.
Product networks are also used in the computation of circular convolution as binding operation in the Semantic Pointer Architecture (TODO ref section).

\chapter{The Semantic Pointer Architecture}
While the Neural Engineering Framework allows us to encode vectors into spiking neural network and transform them, but it does not tell us how to use those vectors to represent structured, conceptual, or symbolic information.
Different such methods could be devised, though in the context of the NEF the most widely used method is the Semantic Pointer Architecture (SPA\@; TODO ref).
The SPA is a specific instance of a Vector Symbolic Architecture (VSA\@; TODO ref).
In VSAs concepts are represented with vectors and linear and nonlinear operators are used to combine basic concepts in more complex structured representations.
Three types of operators can be considered essential in a VSA\@.

First, a measure of similarity
\begin{equation}
    \simmeasure: \mathbb{R}^{\dims} \times \mathbb{R}^{\dims} \longrightarrow \mathbb{R}
\end{equation}
for which we will use the normalized dot product
\begin{equation}
    \simmeasure(\vc x, \vc y) := \frac{\left\langle \vc x, \vc y \right\rangle}{\norm{\vc x} \cdot \norm{\vc y}}
\end{equation}
for the remainder of this thesis.
Second, a superposition operator
\begin{equation}
    \superpos: \mathbb{R}^{\dims} \times \mathbb{R}^{\dims} \longrightarrow \mathbb{R}^{\dims}
\end{equation}
that produces a vector similar to both inputs ($\vc x \sim \superpos(\vc x, \vc y) \sim \vc y$).
This is usually, and will be for the remainder of this thesis, simple addition, i.e. $\superpos(\vc x, \vc y) := \vc x + \vc y$.
Finally, a binding operator
\begin{equation}
    \bind: \mathbb{R}^{\dims_1} \times \mathbb{R}^{\dims_2} \longrightarrow \mathbb{R}^{\dims}
\end{equation}
with an approximate inverse or unbinding operation
\begin{equation}
    \bind^+: \mathbb{R}^d \times \mathbb{R}^{\dims_2} \longrightarrow \mathbb{R}^{\dims_1}
\end{equation}
that satisfies $\bind^+(\bind(\vc x, \vc y), \vc y) \approx \vc x$ and $\vc x \not \sim \bind(\vc x, \vc y) \not \sim \vc y$ (except identity FIXME better explanation/requirement).
Note, that some proposed binding operations, like the tensor product (TODO ref), change the dimensionality of the vectors on binding.
This allows for a general exact inverse $\bind^{-1}(\bind(\vc x, \vc y), y) = \vc x$ to the binding operation.
However, this will increase the vector dimensionality with each binding and leads to scaling issues in a neural system (TODO ref).
Thus, we will only consider binding operations that preserve the vector dimensionality, i.e. $\dims_1 = \dims_2 = \dims$.
This implies, that in general such a binding cannot preserve all the information in both bound vectors.
It leads to a compressed representation.
To fully recover the stored information, clean-up memories will be necessary (TODO ref section).

At this point, it is useful to introduce three more definitions.
\begin{defn}[identity vector]
    A vector $\bid_{\bind}$ with the property $\bind(\vc x, \bid_{\bind}) = \vc x$ is called \emph{identity vector} under $\bind$.
\end{defn}
\begin{defn}[absorbing element]
    A vector $\bzero_{\bind}$ with the property $\bind(\vc x, \bzero_{\bind}) = c \cdot \bzero_{\bind}$ where $c \in \mathbb{R}$ is called \emph{absorbing element} under $\bind$.
\end{defn}
Such an absorbing element effectively destroys the information in the vector $\vc x$.
For that reason, absorbing elements should be avoided when constructing representations with binding.
Note that this definition slightly differs from the usual definition of absorbing elements by allowing for a scaling factor.
\begin{defn}[unitary vector]
    A vector $\vc u$ with the property $\langle \bind(\vc x, \vc u), \bind(\vc y, \vc u) \rangle = \langle \vc x, \vc y \rangle$ is called unitary.
\end{defn}
In other words, a unitary vector preserves the dot product under binding.
This is in analogy to unitary transformation matrices that also preserve the dot product.
It also implies that binding with a unitary vector preserves the length of the bound vector.


\section{Binding operations}
TODO some text

\subsection{Circular convolution}
The binding operator classically used in the SPA is circular convolution and was suggested by TODO ref Plate for his Holographic Reduced Representations (HRRs).
\begin{defn}[circular convolution binding]
    The circular convolution binding operator is given by
    \begin{equation}
        \bind_{\circledast}(\vc x, \vc y) := \vc x \circledast \vc y\ \text{with}\ \del{\vc x \circledast \vc y}_i = \sum_{j=0}^{d - 1} x_j y_{(i - j) \bmod d}
    \end{equation}
    and has the approximate inverse (TODO ref proof)
    \begin{equation}
        \bind^+_{\circledast}(\vc x, \vc y) = \vc x \circledast \vc y^+\ \text{with}\ \vc y^+ := \del{y_0, y_{d-1}, y_{d-2}, \dotsc, y_1}\Tr \text{.}
    \end{equation}
\end{defn}

The basic properties of
\begin{align}
    &\text{distributivity:} &(\vc x_1 + \vc x_2) \circledast \vc y &= \vc x_1 \circledast \vc y + \vc x_2 \circledast \vc y \text{,}\\
    &\text{associativity:} &(\vc x \circledast \vc y) \circledast \vc z &= \vc x \circledast (\vc y \circledast \vc z) \text{,} \\
    &\text{commutativity:} &\vc x \circledast \vc y &= \vc y \circledast \vc x
\end{align}
hold for circular convolution as a binding operator.
A useful property of circular convolution for the implementation in a neural network with the NEF is, that it becomes element-wise multiplication in the Fourier space defined by
\begin{equation}
    \vc x \circledast \vc y = \fouriermat^{-1}\sbr{(\fouriermat \vc x) \circ (\fouriermat \vc y)}
\end{equation}
where $\fouriermat$ is the discrete Fourier transform (DFT) matrix.
The linear transform with the DFT matrix can be put easily into the neural connection weights and the element-wise product can be done with well-optimized product network (TODO ref).

The expression in Fourier space also allows to derive the special elements of circular convolution.
The identity vector must not change the complex Fourier coefficient in the element-wise multiplication.
Thus, its Fourier coefficients must all be $1 + 0\iu$ and the identity vector is given by
\begin{equation}
    \bid_{\circledast} = (1, 0, 0, \dotsc, 0)\Tr \text{.}
\end{equation}
Furthermore, all vectors with Fourier coefficients $c_n \in \mathbb{C}$ that are of unit length ($\abs{c_n} = 1$) will be unitary as one can easily verify.
A trivial example of a unitary vector is the identity vector $\bid_{\circledast}$.
Finally, all vectors $(z, \dotsc, z)\Tr$ with $z \in \mathbb{R}$ are absorbing elements.

\subsection{Vector-derived transformation binding}
Circular convolution can be interpreted as moving one of the operands around in the $d$-dimensional space in a way defined by the other operand.
This leads to the question, whether there are other ways to project one vector to a new location based on the other vector.
One such way is what I call vector-derived transformation binding (VTB) which to my knowledge has not been described before.
\begin{defn}[vector-derived transformation binding, VTB]
    Given a $\dims' = \dims^{\frac{1}{2}} \in \mathbb{N}_{>0}$, the vector-derived transformation binding operator $\vtb: \mathbb{R}^{\dims} \times \mathbb{R}^{\dims} \longrightarrow \mathbb{R}^{\dims}$ is defined as
    \begin{equation}
        \vtb(\vc x, \vc y) := \bar{\mat V}_{\vc y} \vc x = \begin{bmatrix}
            \mat V_{\vc y} & 0 & 0 \\
            0 & \mat V_{\vc y} & 0 \\
            0 & 0 & \ddots
        \end{bmatrix} \vc x
    \end{equation}
    with
    \begin{equation}
        \mat V_{\vc y} = \dims^{\frac{1}{4}} \begin{bmatrix}
            y_1 & y_2 & \dotso & y_{\dims'} \\
            y_{\dims' + 1} & y_{\dims' + 2} & \dotso & y_{2\dims'} \\
            \vdots & \vdots & \ddots & \vdots \\
            y_{\dims - \dims' + 1} & y_{d - \dims' + 2} & \dotso & y_{\dims}
        \end{bmatrix} \text{.}
    \end{equation}
    The approximate inverse is given by
    \begin{equation}
        \vtb^+(\vc x, \vc y) = \bar{\mat V}_{\vc y}\Tr = \begin{bmatrix}
            V_{\vc y}\Tr & 0 & 0 \\
            0 & V_{\vc y}\Tr & 0 \\
            0 & 0 & \ddots
        \end{bmatrix} \vc x \text{.}
    \end{equation}
\end{defn}
This binding method is based on the fact that in the SPA vectors are usually randomly generated and uniformly distributed with identically distributed components.
In that case each subvector (e.g., each row in $\mat V_{\vc y}$) is also uniformly distributed with identically distributed components.
Furthermore, for high-dimensional vector spaces almost all (uniformly sampled) vectors are orthogonal and semantic pointers have usually unit-length.
Thus, the matrix $\bar{\mat V}_{\vc y}$ is almost orthogonal with the implication $\bar{\mat V}_{\vc y}\Tr \bar{\mat V}_{\vc y} \approx \imat$.
Vectors $\vc y$ that give a perfectly orthogonal matrix $\mat V_{\vc y}$, will be unitary.
One special unitary vector is the identity vector.
\begin{corollary}[VTB identity vector]
    The identity vector for VTB is given by
    \begin{equation}
        \sbr{\bid_{\ped{V}}}_i = \left\{ \begin{array}{ll}
                \dims^{\frac{1}{4}} & i \in \cbr{(k - 1) \dims' + k : k \leq \dims', k \in \mathbb{N}_{>0}} \\
                0 & \text{otherwise}
        \end{array}\right. \text{.}
    \end{equation}
    \begin{proof}
        $\mat V_{\bid_{\ped{V}}} = I\ \Rightarrow\ \bar{\mat V}_{\bid_{\ped{V}}} = \imat$
    \end{proof}
\end{corollary}

\begin{corollary}[VTB distributivity]
    VTB is distributive: $\vtb(\vc x_1 + \vc x_2, \vc y) = \vtb(\vc x_1, \vc y) + \vtb(\vc x_2, \vc y)$ and $\vtb(\vc x, \vc y_1 + \vc y_2) = \vtb(\vc x, \vc y_1) + \vtb(\vc x, \vc y_2)$.
    \begin{proof}
        By applying the definitions for both directions of the distributivity:
        \begin{itemize}
            \item $\vtb(\vc x_1 + \vc x_2, \vc y) = \bar{\mat V}_{\vc y} \del{\vc x_1 + \vc x_2} = \bar{\mat V}_{\vc y} \vc x_1 + \bar{\mat V}_{\vc y} \vc x_2 = \vtb(\vc x_1, \vc y) + \vtb(\vc x_2, \vc y)$
            \item $\vtb(\vc x, \vc y_1 + \vc y_1) = \bar{\mat V}_{\vc y_1 + \vc y_2} \vc x = \del{\bar{\mat V}_{\vc y_1} + \bar{\mat V}_{\vc y_2}} \vc x = \bar{\mat V}_{\vc y_1} \vc x + \bar{\mat V}_{\vc y_2} \vc x = \vtb(\vc x, \vc y_1) + \vtb(\vc x, \vc y_2)$
    \end{itemize}
    \end{proof}
\end{corollary}
In contrast to circular convolution, VTB is neither commutative
\begin{equation}
    \vtb(\vc x, \vc y) = \bar{\mat V}_{\vc y} \vc x \neq \bar{\mat V}_{\vc x} \vc y = \vtb(\vc y, \vc x) \text{,}
\end{equation}
nor associative
\begin{equation}
    \vtb(\vc x, \vtb(\vc y, \vc z)) = \bar{\mat V}_{\bar{\mat V}_{\vc z} \vc y} \vc x \neq \bar{\mat V}_{\vc z} \bar{\mat V}_{\vc y} \vc x = \vtb(\vtb(\vc x, \vc y), \vc z) \text{.}
\end{equation}
This implies that unlike circular convolution multiple binding cannot be undone in a single step, but a separate unbinding step is required for each binding.

\subsection{Comparison of circular convolution and vector-derived transformation binding}
Both circular convolution and VTB are compressed binding operations.
Because of the lossy compression, we lose some information in each binding which makes it increasingly harder to recover the original unbound vectors.
To combat this effect (and the effect of neuron noise) clean-up memories like the one by TODO ref Stewart paper are required.

In addition to the information loss, the binding operations will change the length of the vector (if neither operand is unitary).
This can be a problem in a neural representation as neurons will saturate and might not represent the vector accurately anymore.
In particular, neural ensembles in the NEF are optimized for a certain representational space, usually a hyper-ball with a given radius $\radius$.
It is convenient to set $\radius = 1$ and try to keep the Semantic Pointer vectors at unit-length.

Figure TODO shows how the mean length of random vectors repeatedly bound with itself changes.
For the circular convolution the length increases exponentially, while VTB show a much slower increase in the length.
This is beneficial for the usage in a neural network.
Moreover, VTB preserves more information of the bound vectors as shown in Figure TODO.
After repeated binding with itself and then the same number of unbindings, the resulting VTB vector is more similar to the original vector than the circular convolution vector.

While binding vectors with themselves can sometimes be useful (e.g., for generating Semantic Pointers with a successive relationship like position indices), it is much more common to bind randomly sampled vectors.
Figure TODO shows the same experiment where a random vector was used in each binding.
In this case, the vector length will decrease to zero, even if it might increase in some of the early bindings.
Again, this decrease is much quicker for circular convolution binding than for VTB and the latter method also preserves more of the similarity across bindings.

It is conceivable that the problems with scaling of the vector length can be fixed by normalizing after each binding.
This, however, will not affect the loss of information in each binding (Figure TODO).
Also, implementing normalization in a neural network is notoriously difficult because of the involved division with an unbounded output as the divisor approaches zero.
Good approximations of normalization are only possible for a defined and finite input range.
It is worth to note that the neurons in the NEF will perform some sort of soft normalization for large values as the neuron's firing rates will saturate.
But this only affects vectors exceeding a certain length and can lead to other distortion in the representation.

Another approach to prevent the growth or decay of the vector length and even prevent the information loss, is the usage of unitary vectors.
These will keep the vector length constant and perform lossless binding due to their perfect inverse.
Note that binding two unitary vectors will give another unitary vector.
Thus, repeated binding is not a problem.
However, the scaling properties of VSAs are based on the fact that the number of almost orthogonal vectors that fits into a vector space grows exponentially with the dimensionality of that space.
Because not all vectors are unitary, this scaling property might be lost when restricted to unitary vectors.
In fact, the number of almost orthogonal unitary circular convolution vectors grows only linear (TODO proof).
For VTB the number of unitary vectors still grows exponentially (TODO proof), but these are picked out of a space that effectively is only $\sqrt{\dims}$-dimensional.
It might be best to use unitary vectors only for those Semantic Pointers that will be repeatedly used in bindings.
But it is also worth keeping in mind that achieving the theoretical limit of almost orthogonal vectors in a space is hard and unsolved (TODO accurate?) mathematical problem related to sphere packing (TODO ref).
Thus, the practical scaling of the number of useable vectors might be closer to linear for both unitary and non-unitary vectors.

So far VTB looks like the better choice for a binding operation.
But it is not without downsides.
In contrast to circular convolution, it is not associative and commutative.
While the desirability of commutativity depends on the employed representation scheme, the non-associativity implies that each binding has to be undone in an individual step, while circular convolution allows to undo a chain of bindings in a single step if the vector representing that chain is available.
Thus, circular convolution can allow to recover information more quickly.
Ultimately, this will be a question what binding operation the brain applies (if the SPA is at all related ot what the brain does).
Potentially, the binding operations lead to different timing predictions as unbinding with the VTB will take more time.
Deriving such predictions and testing them experimentally is, however, out of the scope of this thesis.

Finally, we have to consider the neural implementation of these binding operations.
Both essentially require a set of multiplication networks.
For the circular convolution, the DFT (and inverse DFT) can be implemented in feed-forward connection weights that do not affect the number of neurons required.
For each of the input vectors $\dims$ Fourier complex coefficients will be produced, but as the inputs are real-valued, half of these will be the complex conjugate of the other half.
Thus, only $\dims / 2$ coefficients have to be considered.
Each coefficient is a complex number multiplied with one coefficient of the other vector.
That results in for real-valued multiplies per coefficient.
In total, $2\dims$ multiplications will be required for a circular convolution.
For VTB, there are $d^{1/2}$ multiplications of $d^{1/2} \times d^{1/2}$ matrices with a vector, resulting in a total of $d^{3/2}$ multiplications.
Thus, the VTB requires more neural resources as a larger number of multiplication networks is required.
It should be noted, that for either binding method the binding with a fixed vector can be implemented purely in the connection weights as it reduces to a simple matrix multiplication in either case.

Despite VTB having many advantages over circular convolution, I decided to use circular convolution in the memory model.
The main reason is that support for circular convolution is already implemented in Nengo and the model does not use a lot of binding operations.
Nevertheless, it would be interesting to switch the model over to VTB in the future.


\section{Structured representations}

Representations, the Semantic Pointers, in the SPA act similar to pointers in computer science.
They can be dereferenced to access information not directly contained in that representation.
But opposed to computer science pointers, they are also semantic by capturing semantic relations with their distance in vector space.



mention Spaun and other models using SPA

\chapter{Optimized high-dimensional representation in spiking neurons}
The implementation of a Semantic Pointer Architecture in a spiking neural network requires the representation of high-dimensional vectors.
While the standard NEF already provides us with a method to do this, it does not tell us the best way to do so.
A good representation will try to minimize the error or noise in the representation.
Alternatively, if the error is sufficiently small, it allows to reduce the number of neurons which reduces the simulation run time.
I previously proposed an optimization method for the representation (TODO ref), which improved the accuracy of SPA operations by up to 25 times.
Here, I will describe a more general applicable method that matches or exceeds the performance of the method among some other advantages.

\section{Types of error in neural representations}
In the NEF the total representational error is given by
\begin{equation}
    \errtotal^2 = \left\langle \errtotal^2(\vc x) \right\rangle_{\!\vc x \in \repspace} = \left\langle \norm{\vc x - \hat{\vc x}(t)}^2 \right\rangle_{\!t,\,\vc x \in \repspace} \text{.}
\end{equation}
As detailed in TODO ref NEF book the total error is constituted out of the error caused by spiking noise $\errnoise$ and the error due to the static distortion $\errdist$ from the non-perfect decoding:
\begin{align}
    \errtotal^2(\vc x) &= \errnoise^2(\vc x) + \errdist^2(\vc x) \\
    \errnoise(\vc x) &= \left\langle \norm{\hat{\vc x}(t) - \langle \hat{\vc x}(t) \rangle_{\!t}}^2 \right\rangle_{\!t} \\
    \errdist(\vc x) &= \norm{\vc x - \langle \hat{\vc x}(t) \rangle_{\!t}}^2 \text{.}
\end{align}
The relation of the error terms is explained by the partitioning of the sum of squares in ordinary least squares model (which is used to solve for decoders in the NEF).
Note that the noise error will depend on the decoding synapse.
As $\tausyn \longrightarrow \infty$, the noise error will approach zero ($\errnoise \infty 0$).
Because the synapse limits how fast the neural representation can be updated, we get a trade-off of the noise in the system and how fast it reacts to new inputs.

Due to the neuron nonlinearities finding analytical solutions for the error terms is likely not possible (except for constrained special cases).
However, we can estimate the error terms from computational experiments.
To do so, we sample $\vc x \in \repspace$ or use a regular grid of $\vc x$.
Each $\vc x$ is then presented for some duration $\Delta t_{\ped{ss}}$ to reach the steady state and then $\hat{\vc x}(t)$ is measured for some duration sample duration $\Delta t_{\ped{sample}}$.
Appropriate durations will depend on the decoding synapse (longer synapses require more time to reach the steady state) and firing rate (a longer sampling duration is required for accurate estimates with low firing rates).

As the dimensionality of the higher-dimensional space increases, it becomes increasingly difficult to cover the whole space with samples from $\repspace$.
Most of the time, though, we can treat the space as an isotropic hyper-ball, i.e.\ it does not matter along which direction we move through the space.
This requires that the NEF ensemble's encoders are uniformly sampled from the hyper-sphere surface which is usually the case (but there are some exceptions like certain implementations of a product network, TODO ref).
Without loss of generality, we assume the representational radius of the hyper-ball to be $r = 1$ (as it is basically just a scaling factor).
The isotropy property allows us to cut through the center of the hyper-ball with a one-dimensional line.
Measuring the error $\err(x) = \err(\vc x)$ at $m$ regular spaced points $\vc x_i = (x_i, 0, \dotsc, 0)\Tr$ with $x_i = i * \Delta x - \Delta x/2 - 1, \Delta x = 2/m$ along such a line, the mean error for the hyper-ball can be estimated as
\begin{align}
    \err &= \frac{\sa_{\dims}}{2\ballvol_{\dims}} \sum_{i=1}^{m} \err(x_i) \cdot \Delta x \cdot r(x_i) \\
    \sa_{\dims} &= \frac{2 \pi^{\dims/2}}{\gammafn(\dims/2)} \\
    \ballvol_{\dims} &= \frac{\pi^{\dims/2}}{\gammafn\!\del{\frac{\dims}{2} + 1}} \\
    r(x) &= \frac{1}{q} \sum_{i=1}^{q} \abs{x + i \frac{\Delta x}{q + 1}}^{\dims - 1}
\end{align}
where $\sa_{\dims}$ is the $\dims$-dimensional solid angle, $\ballvol_{\dims}$ the volume of a $\dims$-ball with radius $\radius = 1$, and $r(x)$ estimates the radius to the power of $\dims-1$ for an $x$ with $q$ evaluation points.
This later estimation of the radius across the $\Delta x$ interval is necessary to not under- or overestimate the integral by a large amount.
This were to happen if only the radius at the exact evaluation point would be used. (TODO why is E divided by two)


\section{Properties of the error in neural representations}
When looking at the representation of a spiking neural network, the noise error is the main factor to consider.
It will go down by $\bO(1/\sqrt{n})$ where $n$ is the number of neurons, whereas the distortion error will decrease by $\bO(1/n)$ and is thus dominated by the noise error (TODO figure, ref NEF book).
In contrast, for rate neurons $\errnoise = 0$ and only the distortion error is relevant.
Furthermore, with the Nengo defaults the noise error in the NEF the increase in the noise error with dimensions $\dims$ will be in $\bO(d)$ (TODO figure).

When looking at the error along a line through the hyper-ball (TODO figure), it becomes apparent that the distortion is mostly flat, but increases near the surface.
This becomes more pronounced as the dimensionality increases.
The noise error will be slightly larger in the center of the ball than towards the surface with higher dimensionalities (it is a flat line for $\dims = 1$).
This is caused by the uniform sampling of evaluation points from the hyper-ball (Figure TODO).
When looking at the convex hull of the sample points, this hull will always be smaller than the hyper-ball (even if some evaluation points are exactly on the surface).
Thus, parts of the hyper-ball near the surface are not covered by the evaluation points and will not be considered in the least squares optimization of the decoders.
As the number of dimensions increases, this will become a bigger problem as the volume for a hyper-ball goes to zero as $\dims \longrightarrow \infty$ (all of the ball will be surface).
To show that this distortion is indeed caused by the partial covering, we can increase the radius of the hyper-ball for sampling the evaluation points slightly to cover more of the unit-ball (Figure TODO).
While this makes the distortion more even, it unfortunately also increases noise level and baseline distortion because evaluation points now have a larger spacing.

Vectors in the SPA are often of unit-length and thus a good, low-distortion representation of the hyper-ball surface is desirable.
Unfortunately, I am not aware of any method to improve the current state.
To completely cover the ball in a convex hull of evaluation points, it is necessary to place some evaluation points outside of the ball which will cover space and optimize for space outside of the representational space.
This will lead to a trade-off of flatness of the distortion and baseline of the distortion.


\section{Effect of the intercept distribution on noise and distortion}


\part{The Context-Unified Encoding memory model}
\chapter{The Ordinal Serial Encoding Model}
The Ordinal Serial Encoding (OSE) model \parencite{Choo2010} is an NEF and SPA based model of serial recall.
It was able to reproduce various effects found in human recall data such as the primacy effect, recency effect, and transposition gradients in serial and delayed forward recall.
Within the context-unified encoding (CUE) memory model it provides the basis for the short-term memory component.

In the OSE, $m$ presented items $\vc v_i$ are bound to fixed position vectors $\vc p_i$ and stored in two memory traces
\begin{align}
    \osestm &= \sum_{i=1}^m \osestmdecay^{m - i} \bind(\vc v_i, \vc p_i) \\
    \oseepis &= \sum_{i=1}^m \oseepisscale^{m - i} \bind(\vc v_i, \vc p_i)
\end{align}
with decay factor $\osestmdecay < 1$ and scaling factor $\oseepisscale > 1$.
The memory traces $\osestm$ and $\oseepis$ represent the short-term and episodic memory store, respectively.
The binding operation $\bind$ used here is circular convolution, but it could be worth exploring the effect of other binding operations in the future.
The recall of an item is given by unbinding the corresponding position vector as
\begin{equation}
    \vc v_i \approx \bind^+\big(\osestm + \oseepis, \vc p_i\big) \text{.}
\end{equation}
These encoding equations produce the primacy and recency effect due to the differential effect of the decay and scaling factors $\osestmdecay$ and $\oseepisscale$.

For the neural implementation, each memory trace can be stored in an integrator with some additional processes for updating and unbinding the recalled item.
For the integration within the CUE memory model, the episodic memory trace is replaced by a version based on the temporal context model presented in the next chapter.
This introduces a more plausible episodic memory, storing experiences in actual synaptic weight changes rather than in the activities of a neural population.
In addition, the recall process needs to be adjusted to integrate information from the exchanged episodic memory component (\cref{sec:recall}).


\section{Neural STM implementation}
In the CUE model, the short-term memory buffer of the OSE model is implemented as depicted in \cref{fig:ose-flip-flop}.
The network gets an item and position Semantic Pointer as input which are bound together.
The bound result is added into the memory trace stored in \pop{combined} as long as it does not receive the \nin{input\_store} signal.
Once the \nin{input\_store} signal is received, the contents from \pop{combined} are transferred to the $\osestm$ populations.
Items are decoded from the memory via the approximate inverse circular convolution.
The decay factor of $\osestmdecay = 0.9775$ is taken from the original OSE implementation \parencite{Choo2010} and is implemented on the connection from $\osestm$ to \pop{combined}.
\begin{figure}
    \centering
    \begin{tikzpicture}[nef]
        \graph [no placement] {
            item/"item $\vc v$" [x=-0.5, y=0, ext];
            pos/"position $\vc p$" [x=-0.5, y=-1, ext];
            bind/"$\circledast$" [x=2, y=-0.5, net];
            diff1/"" [x=4, y=-0.5, ea];
            combined/"" [x=7, y=-0.5, ea];
            combinedlbl/"{\lato combined}" [x=7, y=0.1];
            diff2/"" [x=7, y=-3, ea];
            mem/"$\osestm$" [x=4, y=-3, ea];
            recall/"$\circledast^{-1}$" [x=2, y=-3, net];
            out/"recalled $\hat{\vc{v}}$" [x=-0.5, y=-3, ext];
            store/"input\_store" [x=-0.5, y=-2, ext];
            invstore/"" [x=6, y=-1.5, pnode];

            item -> [out=0, in=170] bind;
            pos -> [out=0, in=190] bind;
            bind -> diff1 -> combined -> diff2 -> mem -> ["\osestmdecay" {auto, yshift=3mm, xshift=1mm}] diff1;
            combined -> [bend right, "$-1$"] diff1;
            mem -> [bend right, "$-1$" below] diff2;
            mem -> recall;
            pos -> [out=0, in=90] recall;
            recall -> out;

            store -> [out=0, in=225, inhibit] diff1;
            store -> [out=0, in=200, "$-1$" {below, very near end}] invstore -> [inhibit] diff2;
            bias/"$1$" [x=5, y=-1.25, ext] -> invstore;
        };
    \end{tikzpicture}
    \caption[Implementation of the OSE short-term memory trace $\osestm$ with the NEF.]{Implementation of the OSE short-term memory trace $\osestm$ with the NEF\@. See text for additional details.}\label{fig:ose-flip-flop}
\end{figure}


\section{Neural position counting}\label{sec:posnet}
For the OSE it is necessary to keep track of the ordinal position of the current item.
The CUE model extends the OSE to do this also in neurons.
\Cref{fig:posnet} shows the network implementing this functionality.
All ensembles in this networks are implementing a threshold at zero so that represented values are always positive.
The \pop{state} ensemble array has one ensemble for each possible position and only the ensemble for the current position is active.
This is ensured by providing a small negative bias ($-0.2$) to all ensembles to prevent spontaneous activity.
Furthermore, a recurrent connection with $\syntau = \SI{0.1}{\second}$ decoding a constant of $1.2$ keeps the current position in a state of stable activity.
\begin{figure} 
    \centering
    \begin{tikzpicture}[nef]
        \graph [no placement] {
            inputinc/"increment" [x=-2, y=0, ext] -> ["\small $\syntau = \SI{5}{\milli\second}$" below]
            red/"" [x=2, y=0, rect] ->
            reg/"" [x=4, y=0, rect];

            inputinc -> [bend left, "\small $-1,\ \syntau = \SI{50}{\milli\second}$"] red;
            adv/"" [x=6, y=-1, ea, inner sep=0.25ex];
            advrect/"" [x=6, y=-1, rect];
            advlbl/"{\lato advance threshold}" [anchor=west, x=6.4, y=-1];

            state/"" [x=3.5, y=-3, ea, inner sep=0.25ex];
            staterect/"" [x=3.5, y=-3, rect];
            statelbl/"{\lato state}" [x=2.9, y=-3.3];

            inhibit/"" [x=6, y=-5, ea, inner sep=0.25ex];
            inhibitrect/"" [x=6, y=-5, rect];
            inhibitlbl/"{\lato inhibit threshold}" [anchor=west, x=6.4, y=-5];

            reg -> ["\small $0.8 \Heavi(x) \imat$" auto] adv;
            adv -> [bend left, "\small $\mat T_2,\ \syntau = \SI{0.1}{\second}$" {below, align=left, very near start, xshift=1.3cm}] state;
            state -> [bend left, "\small $\mat T_1$" above] adv;
            state -> output [x=7, y=-3, ext];
            state -> [bend left] inhibit;
            inhibit -> [bend left, "\small $\mat T_3 \Heavi(\vc x)$" {below, xshift=-5mm, near end}] state;
            bias1/"$-0.6$" [x=3.5, y=-5, ext] -> inhibit;
            bias2/"$-0.2$" [x=2, y=-3, ext] -> state;
            state -> [out=90, in=160, distance=2cm, "\small $1.2 \Heavi(\vc x)$" above] state;

            gating/"gate signal" [x=4, y=1, ext] -> reg;
        };
    \end{tikzpicture}
    \caption[Implementation of position counting with the NEF\@.]{Implementation of position counting with the NEF\@. See text for details.}\label{fig:posnet}
\end{figure}

To advance to the next position a signal with a rising edge has to be provided to the \nin{increment} input.
To detect the rising edge, a differentiator ensemble is used that receives its input via two connections where one connection has a fast synaptic time constant ($\syntau = \SI{5}{\milli\second}$) and the other connection has a slow synaptic time constant ($\syntau = \SI{50}{\milli\second}$) and a transform of $-1$ (\cref{sec:differentiator}).
The output is fed through a gate ensemble that can be inhibited to prevent position increments.

Then the Heaviside step function is decoded from the \pop{gate signal} and fed into the \pop{advance threshold} ensemble array scaled by a factor of \num{0.8}.
The output of \pop{state} is also fed into \pop{advance threshold} with the transform
\begin{equation}
    \mat T_1 = \sbr{\begin{array}{cccc}
        0 & -1 & -1 & \cdots \\
        -1 & 0 & -1 & \cdots \\
        -1 & -1 & 0 & \cdots \\
        \vdots & \vdots & \vdots & \ddots
    \end{array}}
\end{equation}
which inhibits all ensembles except the one corresponding to the current position.
This ensemble only becomes active when a rising edge for the \pop{increment} input is detected.
The \pop{advance threshold} ensemble projects back to the \pop{state} ensembles with a transform of
\begin{equation}
    \mat T_2 = \sbr{\begin{array}{ccccc}
        0 & 0 & \cdots & 0 & 0 \\
        2 & 0 & \cdots & 0 & 0 \\
        0 & 2 & \cdots & 0 & 0 \\
        \vdots & \vdots & \ddots & \vdots & \vdots \\
        0 & 0 & \cdots & 2 & 0
    \end{array}}
\end{equation}
to excite the population representing the next position.

When the next position gets active, the old position needs to be inhibited at some point to prevent two positions from being active at the same time.
This is done via the \pop{inhibit threshold} ensemble array.
It receives a bias input of \num{-0.6} and input from the decoded constant from \pop{state}.
Once the threshold (decoded as Heaviside step function) is exceeded, the previous and next item are inhibited with the transform given by
\begin{equation}
    \mat T_3 = -\sbr{\begin{array}{ccccc}
            0 & 2 & 0 & \cdots & 0 \\
            0 & 0 & 2 & \cdots & 0 \\
            \vdots & \vdots & \vdots & \ddots & \vdots \\
            0 & 0 & 0 & \cdots & 2 \\
            2 & 0 & 0 & \cdots & 0
    \end{array}} - \sbr{\begin{array}{ccccc}
        0 & 0 & \cdots & 2 & 0 \\
        0 & 0 & \cdots & 0 & 2 \\
        2 & 0 & \cdots & 0 & 0 \\
        0 & 2 & \cdots & 0 & 0 \\
        \vdots & \vdots & \ddots & \vdots & \vdots
    \end{array}} \text{.}
\end{equation}
An example of how these component interact to advance the represented position is given in \cref{fig:pos-example}.
\begin{figure}
    \centering
    \includegraphics{figures/pos-example}
    \caption[Position increment in the position counting network.]{Position increment in the position counting network. The top-most plot shows the input signal and output of the differentiator. The following plots from top to bottom show: Heaviside output of the network, representation in the \pop{state} ensembles, transformed input to the \pop{state} ensembles from the \pop{advance threshold} ensembles, and transformed input from the \pop{inhibit threshold} ensembles.}\label{fig:pos-example}
\end{figure}

\chapter{The Temporal Context Model}\label{sec:tcm}
The temporal context model (TCM) was proposed by \textcite{Howard2002} as a model of free recall.
It matched data of immediate, delayed, and continuous distractor recall tasks.
As the distractor task used in the delayed and continuous distractor condition is designed to prevent active rehearsal, this model is likely to address more long-term, synaptic storage as opposed to the short-term OSE model.
Similar to a number of other memory models, the TCM assumes a time varying context signal that items are associated with.
But unlike those other models, this context is based on the items themselves rather than being randomly generated.

In particular, items in the TCM are represented as orthogonal vectors $\tcmitem_i$ and the context signal is also a vector $\ctx$.
If we relax the orthogonality constraint on the items to almost (instead of perfectly) orthogonal, we can use Semantic Pointers for these vectors.
To associate items and contexts, two association matrices are used.
The $\mtf$ matrix represents the associations from a context to an item and is constructed as an outer product matrix as
\begin{equation}
    \mtf = \sum_i \tcmitem_i \ctx_i\Tr \text{.}
\end{equation}
Note that the $\mtf$ matrix can be easily updated by adding another item/context outer product.
The $\mft$ matrix is used to retrieve a context vector $\ctxin_i = \mft \tcmitem_i$ to update the current context according to the \emph{evolution equation} (also see \cref{fig:tcm}a)
\begin{equation}
    \ctx_i = \theta_i \ctx_{i-1} + \tcmbeta \ctxin_i\label{eqn:ctx-update}
\end{equation}
where $\tcmbeta$ is a free parameter controlling how fast the context drifts and $0 < \theta_i \leq 1$ is determined to ensure unit length of $\ctx_i$ in each timestep as
\begin{equation}
    \theta_i = \sqrt{1 + \tcmbeta^2 \sbr{\left\langle\ctx_{i-1}, \ctxin_i\right\rangle^2 - 1}} - \tcmbeta \left\langle\ctx_{i-1}, \ctxin_i\right\rangle\text{.}
\end{equation}
In the CUE model, however, $\theta_i$ is fixed to the asymptotic value for $\langle\ctx_{i-1}, \ctxin_i\rangle \rightarrow 0$
\begin{equation}
    \theta_i = \sqrt{1 - \tcmbeta^2}\text{.}
\end{equation}
In the TCM this corresponds to the assumption that item $i$ has not been presented for a sufficiently long time which results in a retrieved context $\ctxin_i$ that is almost orthogonal to the current context.
This change is further motivated by still producing a good match to the data and simplifying the neural implementation as no dynamic scaling of a vector is required.
Such scaling would require a product network for each vector dimension in the NEF\@.
\Cref{eqn:ctx-update} introduces the asymmetric bias to forward recall into the model (\cref{fig:ctxsim}).
While the similarity of contexts $\langle\ctx_i, \ctx_j\rangle$ is symmetric for the lag $j - i$, $\ctxin_i$ is only included in context vectors $\ctx_j$ with $j \geq i$.
\begin{figure}
    \hfill
    \subcaptionbox{}{\begin{tikzpicture}[every path/.style={-Latex}]
        \graph [grow down=1.5cm, branch right=1.5cm] {
            f0/"$\tcmitem_{i-2}$" -> ["$\mft$" anchor=east] cin0/"$\ctxin_{i-2}$" -> ["$\tcmbeta$" anchor=east] c0/"$\ctx_{i-2}$";
            f1/"$\tcmitem_{i-1}$" -> cin1/"$\ctxin_{i-1}$" -> c1/"$\ctx_{i-1}$";
            f2/"$\tcmitem_{i}$" -> cin2/"$\ctxin_{i}$" -> c2/"$\ctx_{i}$";
            start/"$\dots$" [x=-6cm, y=-3cm] -> ["$\theta$" below] c0 -> c1 -> c2 -> end/"$\dots$" [y=-1.5cm];
        };
    \end{tikzpicture}}  % chktex 31
    \hfill
    \subcaptionbox{}{\begin{tikzpicture}[every path/.style={-Latex}]
        \graph [grow down=1.5cm, branch right=1.5cm] {
            start/"$\dots$";
            c0/"$\ctx_i$" -> ["$\mtf$" anchor=east] f0/"$\tcmitemin_i$" -> ["$\mft$" anchor=east] cin0/"$\ctxin_i$";
            c1/"$\ctx_{i+1}$" -> f1/"$\tcmitemin_{i+1}$" -> cin1/"$\ctxin_{i+1}$";
            c2/"$\ctx_{i+2}$" -> f2/"$\tcmitemin_{i+2}$" -> cin2/"$\ctxin_{i+2}$";
            cin0 -> ["$\tcmbeta$" {near end, yshift=-3mm}] c1;
            cin1 -> c2;
            start -> c0 -> ["$\theta$"] c1 -> c2 -> end/"$\dots$";
            cin2 -> end;
        };
    \end{tikzpicture}}  % chktex 31
    \hfill
    \caption{Evolution of the context in the TCM during (a) item presentation and (b) recall.}\label{fig:tcm}
\end{figure}
\begin{figure}
    \centering
    \includegraphics{figures/ctxsim}
    \caption{Similarity of the context to itself $\langle \ctx_i, \ctx_j \rangle$ and to the retrieved context $\langle \ctxin_i, \ctx_j \rangle$ for different lags $j-i$.}\label{fig:ctxsim}
\end{figure}

Finally, the associations from items to context $\mtf$ need to be updated, so that a recalled item can be used to update and partially restore a previous context to retrieve further items.
In the original TCM, this update is given by
\begin{align}
    \mft_{i+1} &= \mft_i \tilde{\mat{P}}_{\tcmitem_i} + a_i \mft_i \mat{P}_{\tcmitem_i} + b_i \ctx_i \tcmitem_i\Tr \\
    a_i &= \gamma b_i \\
    b_i &= \frac{1}{\gamma^2 + 2 \gamma \left\langle\ctxin, \ctx_i\right\rangle + 1}
\end{align}
with projection operators $\mat{P}_{\vc v} = \vc v \vc v\Tr\!/ \norm{\vc v}^2$, $\tilde{\mat{P}}_{\vc v} = \imat - \mat{P}_{\vc v}$, and a free parameter $\gamma$ specifying the relative contribution of previously associated context $\ctxin_i$ and new context $\ctx_i$.
Again, to facilitate the neural implementation, the exact weighting of $\ctxin_i$ and $\ctx_i$ is relaxed while still achieving a good match to data.
Instead of splitting $\mft_i$ into components parallel and orthogonal to $\tcmitem_i$, $\ctx_i\tcmitem_i\Tr$ is added directly into the matrix with a fixed parameter $b$,
\begin{equation}
    \mft_{i+1} = \mft_i + b \ctx_i\tcmitem_i\Tr\text{.}
\end{equation}

Given a context $\ctx$, a mixture of associated items can be recalled as $\tcmitemin = \mtf \ctx$.
To retrieve a single item some form of cleanup has to performed.
Once such a single item has been recalled, the item can be used to recall the associated context as $\mft \tcmitemin$ which in turn can be used to update the current context according to \cref{eqn:ctx-update}.
The updated context allows recalling further items (\cref{fig:tcm}b).

Different cleanup strategies for the recalled item vector can be used.
In the original TCM model, a set of activities $a_i = \tcmitem_i\Tr \tcmitemin$ was obtained and used to make a probabilistic decision according to Luce's choice rule.
The probability of retrieving item $\tcmitem_i$ is given as
\begin{equation}
    P\bigl(\tcmitem_i \bigl|\,\tcmitemin\bigr) = \frac{\exp\!\del{\frac{2a_i}{\tau}}}{\sum_j \exp\!\del{\frac{2a_j}{\tau}}}
\end{equation}
with a parameter $\tau$ that specifies the sensitivity to the activities.

The version of the TCM model presented by \textcite{Sederberg2008}, uses a winner-take-all process based on the leaky, competing accumulator model \parencite{Usher2001}.
In this model, for each possible item a leaky integrator integrates evidence over time.
At the same time, the integrators inhibit each other laterally.
The dynamics can be described by
\begin{equation}
    \vc x_s = \vc x_{s - 1} + \frac{1}{\tau} \del{\vc u - \kappa \vc x_{s - 1} - \lambda \mat L \vc x_{s - 1}} + \vc\eta
\end{equation}
where $\vc u$ is the scaled vector of inputs determined from $\mtf \ctx$ with the current context, $\kappa$ the leak rate, $\lambda$ the lateral inhibition, ${[\mat L]}_{ij} = 1 - \krond_{ij}$ the lateral inhibition matrix, and $\vc\eta$ normal distributed random noise.
This is a more detailed description of how the brain might actually decide for a single item.
However, it is problematic to incorporate within a larger scale neural model under noisy conditions as detailed in \cref{sec:recall}.
For this reason, a different winner-take-all process described in that chapter is used.


\section{Neural context update}\label{sec:ctx-update}
A neural implementation of the TCM needs to implement the updating of the $\mft$ and $\mtf$ matrices discussed in \cref{sec:aml}, the recall of items discussed in \cref{sec:recall}, and updating of the context given by \cref{eqn:ctx-update}, discussed in the remainder of this chapter.
Despite being a simple equation, a number of different implementation approaches are potentially viable.
However, only one of these methods was successful in matching the human data when incorporated into the complete model.
It is still instructive to compare these different approaches and consider why they fail.

\subsection{Boundend integrator}
\Cref{eqn:ctx-update} assumes discrete steps, but for a neural implementation a continuous formulation is more natural and given by
\begin{equation}
    \od{\ctx}{t} = \big(\bar{\theta} - 1\big) \ctx + \bar{\tcmbeta} \ctxin\,\text{.}
\end{equation}
This equation is easily implemented with a neural integrator for a constant $\bar{\theta}$ and $\bar{\tcmbeta}$.
However, there is no limit on the integration of $\ctxin$ anymore so that the proportion of $\ctxin$ added into $\ctx$ can exceed $\tcmbeta$.
To add at most $\tcmbeta \ctxin$ to the context $\ctx$, we can gate the input to the integrator and add a network computing the dot product between $\ctx$ and $\ctxin$.
After thresholding the dot product at $\tcmbeta$, it can be used to suppress the input by inhibiting the gate ensembles (see \cref{fig:ctx-bounded-integrator}).
\begin{figure}
    \centering
    \begin{tikzpicture}[nef]
        \graph [branch down=2.4cm] {
            in/\ctxin [ext] -!- {
                gate/ [ea] -> ["$\bar{\tcmbeta}$"] integrator/\ctx [ea] -> out/ [ext],
                threshold/ [rect] -!- downscale/$\ctx_{\downarrow}$ [ea] -!- length/ [rect],
                dot [net]
            },
            in -> gate,
            in -> dot -> threshold -> [inhibit, "$\Heavi(x - \tcmbeta)$" {rotate=90}] gate,
            integrator -> dot,
            integrator -> [recurrent, "$\bar{\theta}$" above] integrator,
            integrator -> [bend right, out=340] downscale -> [bend right, "$\zeta$" {xshift=-2mm}] integrator,
            integrator -> ["$1 - \norm{\ctx}$" {anchor=south west}] length -> [inhibit] downscale
        };
    \end{tikzpicture}
    \caption{Bounded integrator network.}\label{fig:ctx-bounded-integrator}
\end{figure}
Furthermore, in the original TCM $\ctx$ was kept at unit length while the integration has no such bound.
To keep the unit length, we can project $\ctx$ to another population $\ctx_{\downarrow}$ which projects back to the integrator with a transform of $\zeta = -0.1$.
Picking a $\zeta$ closer to zero allows the $\vc c$ vector exceed unit length by a larger amount while the integrator receives input and will increase the time required to settle back to unit length, whereas a large magnitude of $\zeta$ can lead to oscillatory behaviour.
The $\ctx_{\downarrow}$ population needs to be controlled to only provide the inhibitory input to the integrator as long as $\lVert\ctx\rVert > 1$.
This is achieved by decoding the length of $\vc c$ from the integrator and thresholding it at $1$.
As long as the threshold is not exceeded $\ctx_{\downarrow}$ is inhibited.

To investigate the behaviour of the network, it was fed with new context vectors $\ctxin$ at rate of one vector per second.
These vectors were either orthogonal or had a cosine similarity of \num{0.6} between successive pairs.
\Cref{fig:bounded-integrator} shows the mean similarity of the context vector to itself with given time lag.
\begin{figure}
    \centering
    \includegraphics{figures/context-analysis/bounded-integrator}
    \caption[Decay in context similarity with the bounded integrator network.]{
        Decay in context similarity with the bounded integrator network given almost orthogonal inputs and inputs with a similarity of approximately \num{0.6}.
        The desired similarity is given by the gray line. Shaded regions indicate \SI{95}{\percent} confidence intervals.}\label{fig:bounded-integrator}
\end{figure}
For (almost) orthogonal vectors $\ctxin$, the similarity between the context vectors is close under the target.
For non-orthogonal vectors with $\big\langle\ctxin_i, \ctxin_{i+1}\big\rangle \approx 0.6$, however, the similarity of the context vectors is by far larger than the target.
This is caused by the input already being similar to the context and thus stopping the update too early.
Note that it is not sufficient to simply adjust $\tcmbeta$ as depending on the similarity of the inputs it needs to either be incremented or decremented.

\subsection{Alternating update of two memories}
With a single integrator, we have to rely on the dot product between the input vector and current context as a measure of $\tcmbeta$.
This dot product is biased in different directions if the input vectors have differing similarities.
To circumvent this, we need to use two gated memory populations that are updated in alternating fashion.
Then the output of the old context and input vector can be combined according to $\theta \ctx + \tcmbeta \ctxin$ and fed into to the memory buffer for the current context.
The completion of that memory update can be detected by the dot product of the updated context and the current context crossing a threshold of one.
Such a network is shown in \cref{fig:ctx-alt-update}.
\begin{figure}
    \centering
    \begin{tikzpicture}[nef, x=2cm, y=2cm]
        \graph [no placement] {
            in/\ctxin [ext, at={(0,0)}] -> ["$\tcmbeta$"] new/ [pnode, at={(1,0)}] -> cgate/ [ea, at={(2, 1)}] -> current/\ctx [ea, at={(3, 1)}] -> oldgate/ [ea, at={(3, -1)}] -> old/$\ctx'$ [ea, at={(2, -1)}] -> ["$\theta$"] new,
            current -> [bend right, "$-1$"] cgate,
            new -> [out=0, in=225] dot [net, at={(4, 1)}], current -> dot [net],
            dot -> rectification/ [rect, at={(4.5, 0.5)}] -> ["$\Heavi(x)$" anchor=west] heavi/ [pnode, at={(4.5, -0.5)}] -> [inhibit, bend left] cgate,
            heavi -> [inhibit] invert/ [ens, at={(4, -1)}] -> [inhibit] oldgate,
            bias/"$1$" [ext, at={(4.5, -1)}] -> invert,
            old -> [bend right, "$-1$" below] oldgate
        };
    \end{tikzpicture}
    \caption{Alternating update of memory buffers.}\label{fig:ctx-alt-update}
\end{figure}

Unfortunately, this still does not work for non-orthogonal input vectors (\cref{fig:amb}).
In that case the dot product of the updated context and current context are already quite high and the updated context is not completely loaded into the current memory buffer.
\begin{figure}
    \centering
    \includegraphics{figures/context-analysis/amb}
    \caption[Decay in context similarity with the alternating update of two memories.]{
        Decay in context similarity with the alternating update of two memories given almost orthogonal inputs and inputs with a similarity of approximately \num{0.6}.
        The desired similarity is given by the gray line. Shaded regions indicate \SI{95}{\percent} confidence intervals.}\label{fig:amb}
\end{figure}


\subsection{Externally controlled alternating memory buffers}
All approaches to determine required context updates based on vector similarity will fail because the similarity of $\ctxin_i$ and $\ctx_{i-1}$ is not known beforehand and can vary widely depending on what contexts are recalled.
Thus, for a properly working context update in the TCM model, the update process has to be controlled by an external control signal (see \cref{sec:control}).
This control signal indicates when the context signal needs to be updated with the provided input and when it has to be kept stable.
If we take the alternating memory buffer network, but control the updating externally (\cref{fig:ctx-ext-ctrl}), it works for both orthogonal and similar input vectors (\cref{fig:ext-amb}).
\begin{figure}
    \centering
    \begin{tikzpicture}[nef, x=2cm, y=2cm]
        \graph [no placement] {
            in/\ctxin [ext, at={(0,0.5)}] -> ["$\tcmbeta$"] new/ [pnode, at={(1,0.5)}] -> cgate/ [ea, at={(2, 1)}] -> current/\ctx [ea, at={(3,1)}] -> oldgate/ [ea, at={(3, -1)}] -> old/$\ctx'$ [ea, at={(2, -1)}] -> ["$\theta$" {very near start, below}] new,
            current -> [bend right, "$-1$"] cgate,
            keep/"keep context" [ext, at={(0,-.5)}] -> ctrl/ [pnode, at={(1,-.5)}] -> [inhibit] cgate,
            ctrl -> ["$-1$" near end] invert/ [pnode, at={(2.5, -0.5)}] -> [inhibit] oldgate,
            bias/$1$ [ext, at={(2.5, 0)}] -> invert,
            old -> [bend right, "$-1$" below] oldgate
        };
    \end{tikzpicture}
    \caption{Alternating update of memory buffers with external control.}\label{fig:ctx-ext-ctrl}
\end{figure}
\begin{figure}
    \centering
    \includegraphics{figures/context-analysis/ext-amb}
    \caption[Decay in context similarity with the alternating update of two memories and external control.]{
        Decay in context similarity with the alternating update of two memories and external control given almost orthogonal inputs and inputs with a similarity of approximately \num{0.6}.
        The desired similarity is given by the gray line. Shaded regions indicate \SI{95}{\percent} confidence intervals.}\label{fig:ext-amb}
\end{figure}

This leads to two predictions.
First, the update of the context signal is not directly regulated by the input, but externally controlled.
Second, there are neural populations that start representing the current context in succession: first the updated context $\ctx$ is constructed in one memory population before it is transmitted to a secondary population $\ctx'$ to be available as old context for the next update.

\chapter{Context update}

The context update network has to approximate Equation~TODO which, as a reminder, is restated here:
\begin{equation}
    \ctx_i = \rho_i \ctx_{i-1} + \tcmbeta \ctxin_i\,\text{.} \label{eqn:ctx-update}
\end{equation}
Different methods of approximating this equation can be thought of and in the following I will describe four methods of which only one was successful in matching the data.
Even though most of these methods have been unsuccessful it is instructive to see why these methods failed to match the data as this demonstrates which features of the mathematical TCM formulation are relevant and which are non-relevant side-effects of a particular formulation.

\section{Boundend integrator}
Equation~\ref{eqn:ctx-update} assumes discrete steps, but for a neural implementation a continuous formulation is more natural and given by
\begin{equation}
    \od{\ctx}{t} = (\bar{\rho} - 1) \ctx + \bar{\tcmbeta} \ctxin\,\text{.}
\end{equation}
This equation is easily implemented with a neural integrator for a constant $\bar{rho}$ and $\bar{\tcmbeta}$.
However, there is no limit on the integration of $\ctxin$ anymore.
To add at most $\tcmbeta \ctxin$ to the context $\ctx$, we can gate the input to the integrator and add a network computing the dot product between $\ctx$ and $\ctxin$.
After thresholding it at $\tcmbeta$, it can be used to suppress the input by inhibiting the gate (see \cref{fig:ctx-bounded-integrator}).
Furthermore, $\bar{\rho}$ needs to be adjusted to keep the unit length of $\ctx$.
To do so, we can project $\ctx$ to another population $\ctx_{\downarrow}$ which projects back to the integrator with a transform of $\gamma = -0.1$.
Picking a $\gamma$ closer to zero will allow the $\vc c$ vector exceed unit length by a larger amount while the integrator receives input and will increase the time required to settle back to unit length, whereas a large magnitude of $\gamma$ can lead to oscillatory behaviour.
The $\ctx_{\downarrow}$ population needs to be controlled to only provide the inhibitory input to the integrator as long as $\norm{\ctx} > 1$.
This is achieved by decoding the length of $\vc c$ from the integrator and thresholding it at $1$.
As long as the threshold is not exceeded $\ctx_{\downarrow}$ will be inhibited.
\begin{figure}
    \centering
    \begin{tikzpicture}[nef]
        \graph {
            in/\ctxin [ext] -!- {
                gate/ [ea] -> ["$\bar{\tcmbeta}$"] integrator/\ctx [ea] -> out/ [ext],
                threshold/ [rect] -!- downscale/$\ctx_{\downarrow}$ [ea] -!- length/ [rect],
                dot [net]
            },
            in -> gate,
            in -> dot -> threshold -> [inhibit, "$\Heavi(x - \bar{\tcmbeta})$" {rotate=90}] gate,
            integrator -> dot,
            integrator -> [recurrent, "$\bar{\rho}$" above] integrator,
            integrator -> [bend right] downscale -> [bend right, "$\gamma$" {below, rotate=270}] integrator,
            integrator -> ["$1 - \norm{\ctx}$" {anchor=south west}] length -> [inhibit] downscale
        };
    \end{tikzpicture}
    \caption{Bounded integrator network.}\label{fig:ctx-bounded-integrator}
\end{figure}

This network I fed it with new context vectors $\ctxin$ at a rate of one vector per second and record the represented context $\ctx$.
I used three metrics to test the basic functionality. First, the norm of the context vector $\norm{ctx}$.
Second the effective $\tcmbeta$ which is the similarity of the represented context $\ctx$ and the new context vector $\ctxin$.
For orthogonal $\ctxin$ vectors, the effective $\tcmbeta$ should rise to $\tcmbeta$, but note that for non orthogonal vectors an effective $\tcmbeta$ of more than $\tcmbeta$ is expected.
In this latter case the new context vectors $\ctxin$ is still supposed to be added with a strength of $\tcmbeta$, but because the context $\ctx$ will already be similar to $\ctxin$, the updated context vectors should have a higher similarity to $\ctxin$ than $\tcmbeta$.
Third and most importantly, it is useful to look at the decay of the context similarity over time for each updated context vector.

\Cref{fig:bounded-integrator-orthogonal} shows these metrics for the bounded integrator network. With the orthogonal input contexts it seems to be working properly.
The norm of the context vectors stays at one, the effective $\tcmbeta$ rises to $\tcmbeta$ for each new input vector, and the context similarity decay is roughly what is expected despite a few traces decaying too quickly.
However, the network fails, if the input context vectors are not orthogonal (\cref{fig:boundint}).
In this case, the context similarity does not decay nearly as quickly as it should.
This can be attributed to stopping the updating once the effective $\tcmbeta$ reaches $\tcmbeta$ even though in the case of similar input vectors this is not sufficient.
\begin{figure}
    \centering
    \includegraphics[draft]{bounded-integrator-orthogonal}
    \caption{
        Properties of the context vectors produces by the bounded integrator context network.
        It was provided with a new input context vector every second.
        Upper left: $\ell^2$-norm of the context.
        Lower left: Similarity of the context to the new input context.
        Right: The blue lines show the similarity of the context after each update to future context vectors.
        The red shading shows the expected similarities for different values of $\tcmbeta$ with the darkest shading corresponding to the target of $\tcmbeta = 0.6$.}\label{fig:bounded-integrator-orthogonal}
\end{figure}
\begin{figure}
    \centering
    \includegraphics[draft]{boundint}
    \caption{
        Properties of the context vectors produces by the bounded integrator context network.
        It was provided with a new input context vector every second.
        Upper left: $\ell^2$-norm of the context.
        Lower left: Similarity of the context to the new input context.
        Right: The blue lines show the similarity of the context after each update to future context vectors.
        The red shading shows the expected similarities for different values of $\tcmbeta$ with the darkest shading corresponding to the target of $\tcmbeta = 0.6$.}\label{fig:boundint}
\end{figure}

\section{Alternating update of two memories}
With a single integrator we have to rely on the dot product between the input vector and current context as a measure of $\tcmbeta$.
To circumvent this we need to use to gated memory populations that are updated in alternating fashion.
Then the output of the old context and input vector can be combined according to $\rho \ctx + \tcmbeta \ctxin$ and fed into to the memory buffer for the current context.
The completion of that memory update can be detected by the dot product of the updated context and the current context crossing 1.
\begin{figure}
    \centering
    \begin{tikzpicture}[nef, x=2cm, y=2cm]
        \graph [no placement] {
            in/\ctxin [ext, at={(0,0)}] -> ["$\tcmbeta$"] new/ [pnode, at={(1,0)}] -> cgate/ [ea, at={(2, 1)}] -> current/\ctx [ea, at={(3, 1)}] -> oldgate/ [ea, at={(3, -1)}] -> old/$\ctx'$ [ea, at={(2, -1)}] -> ["$\rho$"] new,
            current -> [bend right, "$-1$"] cgate,
            new -> [out=0, in=225] dot [net, at={(4, 1)}], current -> dot [net],
            dot -> rectification/ [rect, at={(4.5, 0.5)}] -> ["$\Heavi(x)$" anchor=west] heavi/ [pnode, at={(4.5, -0.5)}] -> [inhibit, bend left] cgate,
            heavi -> [inhibit] invert/ [ens, at={(4, -1)}] -> [inhibit] oldgate,
            bias/1 [ext, at={(4.5, -1)}] -> invert,
            old -> [bend right, "$-1$" below] oldgate
        };
    \end{tikzpicture}
    \caption{Alternating update of memory buffers.}\label{fig:ctx-bounded-integrator}
\end{figure}

Unfortunately, this still does not work for input vectors with some similarity.
In that case the dot product of the updated context and current context will already be quite high and the updated context is not completely loaded into the current memory buffer.
\begin{figure}
    \centering
    \includegraphics[draft]{alternating-memory-buffers}
    \caption{
        Properties of the context vectors produces by the alternating update of two memories context network.
        It was provided with a new input context vector every second with a similarity of approximately $\sqrt{1. - \beta^2}$ between consecutive vectors.
        Upper left: $\ell^2$-norm of the context.
        Lower left: Similarity of the context to the new input context.
        Right: The blue lines show the similarity of the context after each update to future context vectors.
        The red shading shows the expected similarities for different values of $\tcmbeta$ with the darkest shading corresponding to the target of $\tcmbeta = 0.6$.}\label{fig:alternating-memory-buffers}
\end{figure}
neural resources


\section{Externally controlled alternating memory buffers}
All approaches to determine required context updates based on vector similarity will fail because the similarity of $\ctxin_i$ and $\ctx_{i-1}$ is not known beforehand and can vary widely depending on what contexts are recalled.
Thus, for a properly working context update in the TCM model, the update process has to be controlled by an external control signal (TODO reference other chapter).
If we take the alternating memory buffer network, it works for both orthogonal and similar input vectors.
\begin{figure}
    \centering
    \begin{tikzpicture}[nef, x=2cm, y=2cm]
        \graph [no placement] {
            in/\ctxin [ext, at={(0,0.5)}] -> ["$\tcmbeta$"] new/ [pnode, at={(1,0.5)}] -> cgate/ [ea, at={(2, 1)}] -> current/\ctx [ea, at={(3,1)}] -> oldgate/ [ea, at={(3, -1)}] -> old/$\ctx'$ [ea, at={(2, -1)}] -> ["$\rho$" {very near start, below}] new,
            current -> [bend right, "$-1$"] cgate,
            TODO [ext, at={(0,-.5)}] -> ctrl/ [pnode, at={(1,-.5)}] -> [inhibit] cgate,
            ctrl -> ["$-1$" near end] invert/ [pnode, at={(2.5, -0.5)}] -> [inhibit] oldgate,
            bias/1 [ext, at={(2.5, 0)}] -> invert,
            old -> [bend right, "$-1$" below] oldgate
        };
    \end{tikzpicture}
    \caption{Alternating update of memory buffers.}\label{fig:ctx-bounded-integrator}
\end{figure}

This leads to a number of predictions.
First, the update of the context signal is not directly regulated by the input, but externally controlled.
Second, there are neural populations that will start representing the current context in succession.

\chapter{Association Matrix Learning}\label{sec:aml}

The TCM requires two association matrices, $\mtf$ and $\mft$, to be updated.
To translate this into neurons, an appropriate learning rule, the association matrix learning rule (AML), has to be derived.
The TCM gives the update of such an association matrix as
\begin{align}
    \mat M_{i+1} &= \mat M_i + \Delta \mat M_i \\
    \Delta \mat M_i &= \vc v_i \vc u_i\Tr
\end{align}
for adding an association from $\vc u_i$ to $\vc v_i$.
The association matrix after $n$ updates can be expressed as
\begin{equation}
    \mat M_n = \mat M_0 + \sum_{i=1}^{n} \Delta \mat M_i = \mat M_0 + \sum_{i=1}^n \vc v_i \vc u_i\Tr \text{.}
\end{equation}
This allows us to express the neural connection weights after learning $n$ associations as
\begin{equation}
    \weights = \menc \mat M_n \mdec = \menc \mat M_0 \mdec + \menc \sum_{i=1}^n \vc v_i \vc u_i\Tr \mdec
\end{equation}
where $\menc$ is the post-synaptic encoder matrix and $\mdec$ are the pre-synaptic decoders of the identity function.
This equation gives us some important information on how the learning of such association matrices can be implemented.
First, preexisting weights can be implemented as a transform on a normal neural connection that is kept constant.
Second, all the weight changes can be collapsed into decoder changes.
Thus, we need the AML to implement the decoder change given by
\begin{align}
    \tilde{\mdec}_{i+1} &= \tilde{\mdec}_i + \Delta \tilde{\mdec}_i \\
    \Delta \tilde{\mdec}_i &= \vc v_i \vc u_i\Tr \mdec
\end{align}
where $\tilde{\mdec}$ is the matrix of learned decoders.

To implement this within a neural network, the discrete equation has to be converted into continuous form:
\begin{equation}
    \od{\tilde{\mat D}}{t} = \eta \vc v(t) \vc u(t)\!\Tr \mat D \label{eqn:aml}
\end{equation}
with learning rate $\eta$.
Note that while in the discrete formulation all associations are added in with the same strength, in the continuous formulation the associative strength depends on the learning rate and presentation time.
This equation can be directly implemented with the NEF and thus realized with spiking neurons.
That alone, however, does not ensure the biological plausibility as any mathematical formulation of synaptic weight changes could be implemented with the NEF\@.
There are also multiple ways to implement the equation with the NEF that have different implications about a potential biological realization.
In the following these different ways are discussed.


\section{Explicit calculation of weight change}
Both the cue $\vc u(t)$ and target $\vc v(t)$ are available as neural signals.
That allows to implement the calculation of the outer product $\vc v(t) \vc u(t)\!\Tr$ in a set of neural ensembles.
The multiplication of each pair of scalars can be accurately implemented in spiking neurons as demonstrated by \textcite{gosmann2015-1}.
Those ensembles can than be connected up to modulate the synaptic strengths from the pre to the post populations (see \cref{fig:aml-explicit}) by forming the transpose of the outer product, applying a transform of $\mdec\Tr\!$, and transposing back.
\begin{figure}
    \centering
    \begin{tikzpicture}[nef]
        \graph[no placement] {
            u/$\vc u$ [ext, at={(0, 0)}],
            v/$\vc v$ [ext, at={(2, 0)}],
            uv/$\vc v \vc u\Tr$ [net, at={(1, -1)}],
            u -> uv, v -> uv,
            pre [ens, at={(-1, -3)}, minimum size=30] -- mod/"" [minimum size=0, inner sep=0, at={(1, -3)}] -> post [ens, at={(3, -3)}, minimum size=30],
            uv -> [modulatory, "$\mdec\Tr$" right] mod,
        };
    \end{tikzpicture}
    \caption{Explicit neural calculation of the AML weight change}\label{fig:aml-explicit}
\end{figure}

It might be surprising that the change in the connection weights between the pre and post ensemble does not depend on their own activity, but is controlled externally.
While this is different from many other common learning rules, there is evidence of such learning in the brain and specifically the hippocampus \parencite{huilme2014,rebola2017,uchida2012}.
Furthermore, this learning rule requires some preexisting structure and connection weights to calculate the signal for the weight modulation.
But as most NEF models do not give a developmental account of how such structures come about, I put this question aside and leave it at something that has to be answered in the future and concerns any type of NEF model, not just this learning rule.
However, there is one thing that can be criticized as being biologically implausible: the connections from the outer product calculation depend on the decoders of the pre population.
It is not clear how the pre-population could transmit this information to this other place.

A similar problem of biological plausibility occurs in classical neural networks and deep learning with back propagation where the weights for the transmission of the error signal need to be symmetric to the forward weights.
Recently, this concern of biological plausibility in deep learning has been slightly alleviated that the weight symmetry is not strictly required.
It is possible for the network to adjust it forward weights to account for existing, non-symmetric backward weights, a process known as feedback alignment \parencite{lillicrap2016}.
Unfortunately, it is not clear whether this can be applied to the situation here.


\section{Explicit calculation of weight change without weight symmetry}
By reformulating $\vc v(t) \vc u(t)\Tr \mdec$ as $\vc v(t)[\mdec\Tr \vc u(t)]\Tr$ it is possible to implement the learning rule without the need for a connection to be based on decoders of a neural ensemble it is not related to.
Again there is a set of neural ensembles calculating an outer product (see \cref{fig:aml-explicit-no-sym}).
But in contrast to the previous approach, if the pre-ensemble is also used as the cue input $\vc u(t)$, the transform $\mdec\Tr$ applied to $\vc u(t)$ is based on that ensemble's encoders decoding $\vc u(t)$.
The transform could even be rolled into the decoder matrix
\begin{equation}
    \mdec_{*} = \mdec\Tr \mdec \text{.}
\end{equation}
As we generally assume in the context of the NEF that there is some mechanism in place to establish the decoders to compute arbitrary given functions, this might already be considered a satisfactory answer to the biological plausibility.
\begin{figure}
    \centering
    \begin{tikzpicture}[nef]
        \graph[no placement] {
            v/$\vc v$ [ext, at={(2, 0)}],
            uv/$\vc v (\mdec\Tr \vc u)\!\Tr$ [net, at={(1, -1)}],
            v -> uv,
            pre [ens, at={(-1, -3)}, minimum size=30] -- mod/"" [minimum size=0, inner sep=0, at={(1, -3)}] -> post [ens, at={(3, -3)}, minimum size=30],
            pre -> ["$\mdec\Tr$"] uv,
            uv -> [modulatory] mod,
            u/$\vc u$ [ext, at={(-3, -3)}] -> pre,
        };
    \end{tikzpicture}
    \caption{Explicit neural calculation of the AML weight change avoiding weight symmetry}\label{fig:aml-explicit-no-sym}
\end{figure}

However, there it is not possible to state the function to obtain the decoders $\mdec_*$ because the function itself depends on the encoding by the neurons.
Moreover, $\mdec_*$ is a symmetric matrix which is a constrained not respected by the normal decoder optimization process.
But in the context of the NEF, it can be assumed that the neural network can decode the identity function, i.e.\ there is a connection with decoders $\mdec$.
This can be used to learn a connection with decoders $\mdec_*$.

Given the vector of neural activities $\vc \act$ at time $t$, we have $\vc x = \mdec \vc \act$ and can state
\begin{equation}
    \vc x\Tr \vc x = \vc \act\Tr \mdec\Tr \mdec \vc\act \overset{!}{=} \vc\act\Tr\mdec_*\vc\act \text{.}
\end{equation}
This gives us an error expression as
\begin{equation}
    \err(\vc\act) = \left\lvert\vc x\Tr \vc x - \vc\act\Tr \mdec_* \vc\act\right\rvert^2 \overset{!}{=} 0
\end{equation}
with the gradient defined as
\begin{equation}
    \pd{\err^2(\vc\act)}{\mdec_*} = 2 \del{\vc\act\Tr \mdec_* \vc\act - \vc x\Tr \vc x} \del{\vc\act \vc\act\Tr} \text{.}
\end{equation}
The gradient can be used for a weight update rule
\begin{equation}
    \od{\mdec_*}{t} = -\eta_* \pd{\err^2(\vc\act)}{\mdec_*}
\end{equation}
in a (spiking) neural network to perform stochastic gradient descent.

To demonstrate that this learning rule allows to learn the desired symmetric matrix, it was applied to a connection where the pre-synaptic ensemble was fed with randomly varying vector signal.
The vector was generated from bandwidth limited Gaussian white noise (upper limit \SI{40}{\hertz}) for each component, normalized, and then multiplied by a bandwidth limited white noise scalar.
The learning rate was set to $\eta_* = \SI{1e-13}{\second^{-1}}$.
We can see that the Frobenius norm of the difference between the learned matrix $\mdec_*$ and the desired matrix $\mdec\Tr\mdec$ continuously decreases (\cref{fig:aml:grad-err}).
The decoded output for some dimensions quickly aligns with the target output (\cref{fig:aml:grad-decode}), but does not happen for all dimensions.
This might be due to the random input not sufficiently covering the representational space.
\begin{figure}
    \centering
    \includegraphics{figures/aml-grad-err}
    \caption{Error $\lVert \mdec\Tr\mdec - \mdec_*\rVert$}\label{fig:aml:grad-err}
\end{figure}
\begin{figure}
    \centering
    \includegraphics{figures/aml-decode}
    \caption[Example of two outputs decoded with the symmetric weights being learned]{Example of two outputs decoded with the symmetric weights being learned. The output dimensions on the left quickly aligns with the target, while this does not happen for the output dimension on the right within the shown time frame.}\label{fig:aml:grad-decode}
\end{figure}

Again, the pure derivation of a learning rule does not ensure its biological plausibility.
Thus, let us consider the individual terms in the gradient.
The term $\vc\act\Tr \mdec_* \vc\act$ is using the current decoders to decode the ensemble's activity in no way different than usually done in the NEF\@.
The decoded value is then correlated with the ensemble's activity.
The plausibility of this is somewhat unclear, but to my knowledge there is no data excluding this possibility.
In particular, the decoded value and activities could be projected to another neural ensembles (see \cref{fig:aml-grad-desc}) that calculates the inner product.
The same, a projection to neural ensemble calculating the inner product, could happen with the decoded value $\vc x$.
The existence of these standard decoders is generally assumed in the NEF\@.
Alternatively, $\vc x\Tr \vc x$ could directly be decoded (as a square of the individual components all projecting into the same dimension).
\Cref{fig:aml:neural-grad-err} shows the development of the error when learning $\mdec_*$ with such a neural gradient computations.
The observed decrease demonstrates the basic viability of this approach, but the learning is slower due to the spiking noise.
\begin{figure}
    \centering
    \begin{tikzpicture}[nef]
        \graph[no placement] {
            pre [ens, at={(0, 0)}, minimum size=30] -- mod/"" [inner sep=0, minimum size=0, at={(2.5, 0)}] -> post [ens, at={(4, 0)}, minimum size=30],
            pre -> [bend left] act/"$\vc\act\Tr \mdec_* \vc\act$" [net, at={(0, 3)}],
            pre -> [bend right, "$\mdec_*$" xshift=-3mm] act,
            pre -> targetact/"$\vc x\Tr \vc x$" [net, at={(1.5, 1.5)}],
            act -> combine/"" [pnode, at={(3, 3)}],
            targetact -> ["$-1$"] combine,
            combine -> [modulatory, out=270, in=90] mod,
        };
    \end{tikzpicture}
    \caption{Neural computation of the error signal necessary for learning symmetric decoders with gradient descent}\label{fig:aml-grad-desc}
\end{figure}
\begin{figure}
    \centering
    \includegraphics{figures/aml-neural-grad-err}
    \caption{Error $\lVert \mdec\Tr\mdec - \mdec_*\rVert$ with neural gradient calculation}\label{fig:aml:neural-grad-err}
\end{figure}


The whole term $\vc\act\Tr \mdec_* \vc\act - \vc x\Tr \vc x$ is a scalar that influence all synapses.
This could be realized by a broadly acting neuro-modulator or by extensive connectivity of the neural ensembles calculating this error term.
Neither option is implausible.
The role of this gradient term seems to play a role weight normalization as it acts equally on all weights and the magnitude depends on the magnitude of the values of $\mdec_*$.
Also, without it all weight changes would always be negative.
Note that in many learning rules normalization factors are often criticized as implausible for requiring knowledge of the whole weight matrix at the level of a single synapse.
This critique does not apply to this learning rule.
The decoding matrix is only used to decode from the activities which require only local knowledge of the weights.

Finally, the term $\vc\act \vc\act\Tr$ correlates neuron activities and appears somewhat Hebbian-like.
Hebbian learning is one of the best studied ways biological systems learn.
In this form of learning pre- and post-synaptic neurons become connected when they fire in short succession.
In difference to this usual account, here two pre-synaptic neurons will connect to the same post-synaptic neuron if they fire together.
Again, the biological plausibility of this is somewhat uncertain, but cannot be outright rejected.


\section{Implicit error calculation}
The previous two implementations of the AML use minimal assumptions about the computational power of synapses.
All that is needed are additive weight changes proportional to some error signal.
However, it has been proposed that synapses could not simply transmit information, but also perform computations \parencite{abbott2004}.
In particular, \textcite{andreasstockel2018} showed that conductance-based synapses can be used in the NEF to compute nonlinear functions like mulitplication.
That should allow to roll the explicit outer product required in the previous two approaches into the synapses itself.
This requires the considerably less neurons as no neural populations are required for each product.

In this thesis, I am using an implementation that corresponds to this implicit error calculation, but implements the required synaptic computation in pure math rather than implementing it with actual synaptic models.
This is mainly to reduce simulation times and is not meant as a statement that this is likely to correspond to the brain's implementation as there is not enough evidence to make such a strong claim.


\section{Normative interpretation}
While the AML cannot be outright rejected as biologically implausible, there are definitely some open questions concerning it.
Nevertheless, it should not be evaluated purely on this fact.
Many models in cognitive science, psychology, and computational neuroscience assume the encoding of associations in an outer product matrix \parencite[e.g.,][]{kajic2017,nowak2001}.
Ultimately, the validity of all those models depends on the possibility that the brain can learn such a matrix.
As such, the AML makes it explicit which operations have to be implemented to enable such learning.
If these turn out as either not being implemented in the brain or as not being implementable at all, it would follow that the brain has to use some other mechanism to represent associations.
Thus, the AML has merit as a normative theory, describing what the brain is ought to do, and directing research to open questions.

That being said, the AML does not make any assumption about the pre-synaptic neurons.
With such assumptions, some of the restrictions of the AML might be lessened.
For example, assuming orthogonal encoders, the decoder matrix $\mdec$ will be orthogonal too.
That in turn means that $\mdec_* = \mdec\Tr \mdec = \imat$ becomes the identity matrix which is independent of the exact decoders.
This simplifies connectivity and removes the need to learn the symmetric matrix $\mdec_*$ which alleviates some of the concerns of biological plausibility.

The dentate gyrus in the hippocampus is often assumed to pattern separation which is a form of orthogonalization.
Hence, it might be possible that the hippocampus learns associations with a form of the AML where $\mdec_*$ ends up being the identity matrix and thus simplifies the learning rule itself.


\section{Properties of the AML}
So far the considerations about the AML were purely theoretical.
\Cref{fig:aml} demonstrates that an implementation of the AML can indeed learn associations in a spiking neural network.
Five different cue-target pairs were presented for one second each before testing the recall with the same cues, but no target vectors.
The initially presented target vectors are almost perfectly recalled.
Note that no catastrophic forgetting occurred and each association was learned with a single presentation of the cue-target pair (one-shot learning).
\begin{figure}
    \centering
    \includegraphics{figures/aml}
    \caption[Learning and recall testing of five cue-target pairs with the AML]{Learning and recall testing of five cue-target pairs with the AML\@. Each colored trace is the dot product with one of the vectors used.}\label{fig:aml}
\end{figure}

It is worth highlighting, that this learning rule allows one-shot learning, i.e.\ a single presentation of an association allows to recall it perfectly even after several other associations have been learned.
Most other learning rules exhibit destructive interference between the items in this scenario.
As an example, Figure TODO shows the same experiment using the Prescribed Error Sensitivity (PES) learning rule (TODO ref) which is commonly used in NEF models (TODO refs).
Here all associations except for the last get destroyed.
It is still possible to learn associations with PES, but it requires to present each item multiple times, i.e.\ one-shot learning cannot be done to learn associations in a reliable way.
To learn a new association, old associations need to be presented again.

This ability for one-shot learning with the AML is due to the $\mdec_*$ matrix which accounts for the interference between neurons.
In fact, the AML and PES are the same except for that matrix.
The PES learning rule can be expressed as
\begin{equation}
    \od{\tilde{\mdec}}{t} = \eta \vc v(t) \vc\act_{\vc u}\Tr(t) \text{,}
\end{equation}
while the AML learning rule can be written as
\begin{equation}
    \od{\tilde{\mdec}}{t} = \eta \vc v(t) \vc u(t)\Tr \mdec = \eta \vc v(t) \vc\act_{\vc u}\Tr(t) \mdec\Tr \mdec
\end{equation}
which is the same, except for $\mdec_* = \mdec\Tr \mdec$.

The one-short learning ability comes with some limitations, though.
The learning rule is restricted to learn transformations that can be expressed as outer product matrices, while learning rules like PES can learn arbitrary transform matrices.
Moreover, the learning rule is essentially trying to learn a look-up table mapping items to other items.
Thus, one should not expect the AML to generalize.


\section{Weight normalization}
Note that the AML allows weights to grow without bound.
By introducing a factor of $1 - \vc v(t)\Tr \hat{\vc v}(t)$ this can be prevented, but similar to other weight normalizations it introduces the need for each weight to have access to the global population activity and weights as $\hat{\vc v}(t) = \tilde{\mdec} \vc\act_{\vc u}(t)$.
Such global dependencies are ofter criticized for not being biological plausible.
As such, I decided to take a slightly different approach with an equivalent effect.
Instead of including the dot product $\vc v(t)\Tr \hat{\vc v}(t)$ in the learning rule, it can be computed by another neural population and the result can be used to inhibit the population providing $\vc v$.
Once fully inhibited $\vc\act_{\vc v}(t)$ will be all-zero and thus prevent further weight changes.


Things to mention:
\begin{itemize}
    \item Different learning rules for different things seem plausible. For example, motor learning requires many repetitions (no one-shot learning), but seems to somewhat generalize (and is continuous). So this might be done with something PES like. Fact learning (a form of associations) can be one-shot, but generalizing from multiple facts is a conscious process/does not happen automatically. This might be done with something AML like.
    \item Maybe that explains the need for consolidation? First, one shot learning that doesn't allow generalization, then in consolidation commonalities get extracted? Replay in sleep/resting might produce the interleaving/multiple presentations required to learn associations with PES\@.
    \item Gallistel argues there is no general purpose learning rule, which is in line with the more pupose specific learning rules in this chapter
\end{itemize}

\chapter{Recalling items}\label{sec:recall}
In the original TCM model \parencite{Howard2002} the activations $a_i$ of items in the memory is mapped to a recall probability by a softmax function
\begin{equation}
    P(\tcmitem_i | \ctx) = \frac{\exp(2a_i/\tau)}{\sum_j \exp(2a_j/\tau)}
\end{equation}
with a free parameter $\tau$ controlling for the sensitivity.
While this does well in capturing the recall probabilities, it does not provide much insight in how this recall process might be realized neurally.
In an extension of the TCM model \parencite{Sederberg2008} a winner-take-all (WTA) process based on the widely-used leaky, competing accumulator (LCA) model by \textcite{Usher2001} was used.
This works well if the output can be evaluated in a mathematical analysis.
However, within the CUE model other parts of the model need to recognize when a single recall is completed to update the context and reset the recall system.
As I show, this is difficult to do robustly with the LCA model, but more easily accomplished with an alternate WTA mechanism termed the independent accumulator (IA) model.
A comparison of these two networks has also been previously published as \textcite{jangosmann2017}.


\section{Leaky, competing accumulator model}
Given $D$ choices, the leaky, competing accumulator (LCA) model proposed by \textcite{Usher2001} describes the dynamics of $D$ scalar state variables $x_i(t), 1 \leq i \leq D$ as
\begin{equation}
    \od{x_i}{t} = \frac{1}{\tau} \del{u_i(t) -\kappa x_i - \lambda \sum_{j \neq i} x_j}, \quad x_i \geq 0 \label{eqn:lca}
\end{equation}
where $u_i(t)$ are the external inputs, $\kappa$ is the leak rate, $\lambda$ the lateral inhibition, and $\tau$ the integration time-constant.
Each state variable $x_i$ is kept non-negative by setting negative values to zero.
Intuitively, each state variable integrates its input with leak term of $-\kappa x_i$ and provides lateral inhibition to all other state variables.
Given one input $u_i > u_j$ for all $j \neq i$, the state variable $x_i$ will converge to $u_i$ while all other state variables $x_j,\ j \neq i$ will converge to $0$ if $\kappa = \lambda = 1$ (\cref{sec:apdx-wta}).
In the following analysis, $\kappa = \lambda = 1$ will be fixed.
Other choices of $\lambda$ affect the effective integration time-constant $\tau$ and gain on the input, while changing $\beta$ will result in undesired behaviour where the history of inputs influences the current choice.

By means of principle 3 of the NEF, the prescribed dynamics can be exactly implemented in the NEF\@.
Here, one ensemble per state variable is used (\cref{fig:lca}) and the gains and biases of the neurons are distributed as described in \cref{sec:thresholding} to ensure the rectification of the state variables.
\begin{figure}
    \begin{captionbeside}{Leaky, competing accumulator (LCA) network\label{fig:lca}}
        \begin{tikzpicture}[nef, every path/.style={-{Latex}}]
            \matrix [column sep=20, row sep=18] {
                \node (rho1) {$\rho_1$}; & \node (x1) [ens] {$x_1$}; &[20] \node (out1) {}; \\
                \node (rho2) {$\rho_2$}; & \node (x2) [ens] {$x_2$}; & \node (out2) {}; \\
                & \node {$\vdots$}; & \\
                \node (rhoD) {$\rho_D$}; & \node (xD) [ens] {$x_D$}; & \node (outD) {}; \\
            };

            \foreach \i in {1, 2, D} {
                \draw [loop above, min distance=18, in=110, out=70] (x\i) to (x\i);
                \draw (rho\i) to (x\i);
                \draw (x\i) to (out\i);
            }
            \draw (x1) [inhibit, in=20, out=340, distance=15] to (x2);
            \draw (x1) [inhibit, in=20, out=340, distance=30] to (xD);
            \draw (xD) [inhibit, out=35, in=325, distance=30] to (x1);
            \draw (xD) [inhibit, out=35, in=340, distance=15] to (x2);
            \draw (x2) [inhibit, out=40, in=310, distance=8] to (x1);
            \draw (x2) [inhibit, out=320, in=50, distance=8] to (xD);
        \end{tikzpicture}
    \end{captionbeside}
\end{figure}


\section{Independent accumulator model}\label{sec:ia}
The dynamics of the independent accumulator (IA) model are given by
\begin{align}
    \od{x_i}{t} &= \frac{1}{\tau_1} u_i(t) + \frac{1}{\tau_2} \del{\bar{x}_i - \bar{\lambda} \sum_{j \neq i} \bar{x}_j}, \quad x_i \geq 0 \\
    \bar{x}_i &= \Heavi(x_i - \vartheta)
\end{align}
where $u_i(t)$ again gives the external input, $\tau_1$ and $\tau_2$ are feedforward and feedback time-constants, $\bar{\lambda}$ is an inhibition constant, $\Heavi$ is the Heaviside step function, and $\vartheta$ is a threshold.
The Heaviside non-linearity is the crucial difference to the LCA model as we can reduce this equation to \cref{eqn:lca} by setting $\tau = \tau_1$, $k = -\tau_1/\tau_2$, and $\lambda = \bar{\lambda} \tau_1/\tau_2$ when not considering the Heaviside function.
Through the Heaviside function, the accumulators will act as perfect (opposed to leaky) and independent integrators.
Only when the threshold $\vartheta$ is reached, the winning choice will be stabilized by feedback while all non-winning choices will be inhibited.

These dynamics can again be neurally implemented by means of the NEF (\cref{fig:ia}).
While for the LCA model a single layer was sufficient, it is best to use two layers to implement the IA model.
The first layer consists of independent accumulators representing the state variables $x_i$.
This layer projects to the second layer that performs the computation of $\bar{x}_i$ (i.e.\ the Heaviside function).
The second layer projects back to first layer and can be used to read out the output of the network.
\begin{figure}
    \begin{captionbeside}[Independent accumulator (IA) network]{The independent accumulator (IA) network. The second layer is used to compute the function $\bar{x}_i = \Heavi(x_i - \vartheta)$.\label{fig:ia}}
        \begin{tikzpicture}[nef, every path/.style={-{Latex}}]

            \matrix [column sep=20, row sep=18] {
                \node (rho1) {$\rho_1$}; & \node (x1) [ens] {$x_1$}; &[20] \node 
                (bar-x1) [ens] {$\bar{x}_1$}; & \node (out1) {}; \\
                \node (rho2) {$\rho_2$}; & \node (x2) [ens] {$x_2$}; & \node (bar-x2) 
                [ens] {$\bar{x}_2$}; & \node (out2) {}; \\
                & \node {$\vdots$}; & \node {$\vdots$}; & \\
                \node (rhoD) {$\rho_D$}; & \node (xD) [ens] {$x_D$}; & \node (bar-xD) 
                [ens] {$\bar{x}_D$}; & \node (outD) {}; \\
            };

            \foreach \i in {1, 2, D} {
                \draw [loop above, min distance=18, in=110, out=70] (x\i) to (x\i);
                \draw (rho\i) to (x\i);
                \draw (x\i) [{Latex}-{Latex}] to  (bar-x\i);
                \draw (bar-x\i) to (out\i);
            }
            \foreach \i/\j in {1/2, 1/D, 2/D} {
                \draw (bar-x\i) [inhibit, in=20, out=200]  to (x\j);
            }
            \foreach \i/\j in {2/1, D/1, D/2} {
                \draw (bar-x\i) [inhibit, in=-20, out=160]  to (x\j);
            }

            \node (l1) [above=0.4cm of x1, ext] {Layer 1};
            \node [above=0.4cm of bar-x1, ext] {Layer 2};
            \node [left=0.1cm of l1, ext] {Inputs};
        \end{tikzpicture}
    \end{captionbeside}
\end{figure}

\section{Comparisons of the WTA networks}
The pure analytical description does not tell us which network is better suited for certain tasks.
To compare these networks, I simulate them with an input of $u_i = u - s\del{1 - \krond_{1i}} + \eta_i$, where $u$ is the magnitude of the largest input, $s > 0$ is the target separation, $\krond$ is the Kronecker delta, and $\eta_i$ is Gaussian white noise with a standard deviation of $\sigma$.  (without loss of generality as we can reorder the indices).
The first input is always set to be the target and all other inputs are set to be lower by $s$.
The rationale behind this is that it should present the hardest case because every choice is close to the largest input.
As $s \rightarrow 0$, the problem will get more difficult as the separation to the target shrinks.
Note, that $u$ determines the general baseline of inputs in addition to the value of the largest input.
Furthermore, it is important to consider the influence of noise $\eta_i$ on the robustness of the decision process.

As the input lists to the CUE model will be about ten to twelve items, I compare the networks for $D=10$ choices.
To make a fair comparison \num{200}~LIF neurons are used per choice in either model.
In the IA model this means, that the number of neurons is split between the first and second layer.
Here I use \num{150} neurons in the first, and \num{50} neurons in the second layer.
As the first layer is performing the evidence accumulation it needs to more accurate and requires more neural resources.
Due to the lower number of neurons in the output layer, the decoded output has a higher variance.
This, however, is not as relevant as variation of the highest output.
If all other choices do not produce an output, the winning choice can still be clearly identified despite the variance.

Given this basic setup of the comparison, a number of metrics give insight in the performance of the networks.
First, we want the network to reach a \emph{clear decision} which I define as exactly one output being above a threshold of \num{0.15} during the time interval from \SIrange{1}{2}{\second} simulation time, while all other outputs remain below the threshold.
The threshold of \num{0.15} was chosen as it is usually sufficient to tell a zero and non-zero signal apart despite spiking noise (except for very low neuron numbers or firing rates).
This metric requires that the output does not change during the interval as this might cause problems in downstream networks that operate on the output.
Also, with a change in output it would be unclear which output should be taken as the actual decision.
One thing this metric does not take into account is whether the output corresponds to the highest input.
This, whether a decision is \emph{correct}, is the second metric.
But note, that in some situations within a larger scale network a wrong, but clear decision, can be preferable to a decision that tends to be correct, but is unstable.

Two more metrics are used for all trials that reached a clear decision.
The amount of time taken to fulfill the conditions for a clear decision are considered as the \emph{decision time}.
It measures how fast the network is obtaining the decision.
Finally, the networks can produce transient outputs unrelated to the final decision.
A downstream network might consider this output as an actual decision and thus those transient outputs should be as small as possible.
The \emph{highest output of a losing choice} during the whole simulation is taken as a metric for this.


\subsection{Results}
\Cref{fig:ia-clear} shows the proportion of trials that the LCA network reaches a clear decision for different input parameters.
The input magnitude $u$ must be large enough to reliably exceed the \num{0.15} detection threshold under noise.
For $u = 0.2$ and even low amounts of noise the network fails to reach a clear decision.
However, it seems that a too large input magnitude decreases performance as well when we compare the performance for $u=0.6$ and $u=1$.
Increasing the noise standard deviation $\sigma$ or decreasing the target separation $s$, both decrease the performance as this makes the problem harder.
Interestingly, the IA network does not fail to reach a clear decision in any of these conditions.
\begin{figure}
    \centering
    \includegraphics{figures/ia-clear}
    \caption[Proportion of trials with a clear decision for the LCA network]{Proportion of trials with a clear decision for the LCA network. The data for correct trials is exactly the same. Each plot shows a different input magnitude $u$ and each curve is for a different target separation $s$. The optimum is marked with the gray horizontal line and coincides with the performance of the IA network for clear decisions. Error bars show \SI{95}{\percent} confidence intervals.}\label{fig:ia-clear}
\end{figure}

Moving on to correct decisions, the LCA network always produced the correct output given it produced a clear decision.
The IA network, may produce incorrect outputs despite a clear decision (\cref{fig:ia-correct}).
Again, as the problem gets harder either by decreased target separation or increased noise, the network performance on this metric decreases.
Note that the feedforward integration time-constant can be increased to integrate evidence over a prolonged time period  to increase performance on this metric (right-most panel).
\begin{figure}
    \centering
    \includegraphics{figures/ia-correct}
    \caption[Proportion of correct trials for the IA network]{Proportion of correct trials for the IA network. Each plot shows a different combination of input magnitude $u$ and integration time constant $\tau_1$. Each curve shows a different target separation $s$. The optimum is marked with the gray horizontal line. Error bars show \SI{95}{\percent} confidence intervals.}\label{fig:ia-correct}
\end{figure}

This, however, will increase the decision times.
These tend to be already slower for the IA network than for the LCA network (\cref{fig:ia-time}).
Additional noise can shorten the decision times in the IA network as it increases the likelihood of an integrator randomly accumulating enough evidence to cross the threshold.
For the LCA network the noise level only has a minor influence on the decision time and was averaged over in the plot.
In other words, noise in the LCA network influences whether a decision can be reached, but not how long it takes to reach a decision if one is reached.
\begin{figure}
    \centering
    \includegraphics{figures/ia-time}
    \caption[Mean WTA decisions times]{Mean decision times for the (a) LCA and (b) IA network. Data for the LCA network is shown as depending on the input magnitude $u$ and is averaged over all noise levels $\sigma$ because noise had a minimal effect on decision times. Data fro the IA network is shown as depending on the noise standard deviation $\sigma$ for two input magnitudes of $u = 1$ (dashed lines) and $u = 0.2$ (solid lines). Target separation is indicated by color. Error bars show \SI{95}{\percent} confidence intervals.}\label{fig:ia-time}
\end{figure}

Finally, both networks might produce a transient response that gets worse with increased noise (\cref{fig:ia-transient}).
This is inherent in the LCA network because the state variables are directly used as output.
In the IA network, this transient response is caused when two accumulators cross the threshold in close temporal proximity before the inhibition from the first one can silence the remaining accumulators.
By increasing the feedforward time-constant, this transient response gets reduced as the evidence integration gets slowed down, which temporally stretches out when accumulators cross the threshold.
\begin{figure}
    \centering
    \includegraphics{figures/ia-transient}
    \caption[Transient WTA responses]{Transient responses (i.e.\ highest output of a non-winning choice) for the WTA networks as depending on the noise standard deviation $\sigma$. Data from the IA network is shown for two integration constants. Each curves shows a different target separation $s$. Error bars show \SI{95}{\percent} confidence intervals.}\label{fig:ia-transient}
\end{figure}


\subsection{Discussion}
Neither network performs better on all metrics.
Thus, each is best suited for a different purpose.
In situations where a continuous adjustment of a decision is necessary, the LCA network is most likely the better choice.
Under noisy conditions it is not necessarily able to produce a stable, clear output, but if a clear output is obtained, it is always correct.
If the input changes, the output adjusts according to the network's time constant and allows for continuous updates.
Unfortunately, this also makes the network susceptible to noise.

The IA network is better for a discrete series of decisions.
It does not produce a continuous output, but waits until the evidence integration threshold is crossed, at which point the network needs to be inhibited to be reset and obtain another decision.
While this produces sequential and discrete decisions, it has the advantage that evidence can be accumulated over a longer time frame to average out noise.
Depending on the choice of the integration time-constant, the IA network either be not as quick or as reliable in identifying the correct winner as the LCA network.
However, it will eventually produces a clear decision that a downstream network can act on (as long as at least one input is strictly positive).
This can be important in a larger scale model where stalling a decision indefinitely can result in a breakdown of model behaviour.
Or in other words, sometimes it is better to act on a wrong decision than to not act all.
For example, in memory recall it might be better to produce a wrong output and continue to recall the next item than indefinitely be trying to recall an item that cannot be recalled.
It is also worth pointing out that the IA network allows for dynamic control of the decision speed by adjusting the $\vartheta$ threshold through a bias input to the second layer.

As mentioned, it is also important to consider the transient outputs of the network within a larger scale model.
Such transient responses are inherent in the LCA model as the state variable is directly used as an output.
One could pass the output through a thresholding layer, but the right choice of threshold is not clear, as the output magnitude depends on the input magnitude.
If the threshold is too low, a transient response would be produced even with the thresholding layer.
If it is too high, small inputs might not produce an output at all.
While the IA can also produce transient responses, these can be reduced to almost zero by a proper selection of the integration time-constant according to the input magnitude and target separation.

By increasing $\tau_1 \rightarrow \infty$, the IA discrimination ability can be increased without bound.
This is sometimes used as argument to criticize these sort of models \parencite[e.g.,][]{Usher2001}, because there is no sensible stopping criterion.
However, this does not consider that there might be a cost to taking more time for a decision.
If that cost is included, then there is a trade-off between accurate decisions and the cost incurred by taking more time to decide.
Furthermore, this argument also assumes perfect integration accuracy, which an actual neural system cannot have due to neural noise and limited neural resources.

To summarize these findings, for the recall of items in memory experiments the IA network is better suited.
Such recall requires discrete decisions and a stable input to downstream motor systems producing the response.
Such a stable input cannot be guaranteed with the LCA network and it proves a challenge to detect when a recall is completed.
\Textcite{Sederberg2008} used the LCA network for modelling the recall, but it is important to highlight that upon reaching the decision threshold, the state variables were immediately set back to zero.
This is easy to do in a pure math model, but when transitioning to a full neural model detecting the decision and resetting the state variables is much more difficult, as it cannot be done instantaneously.

Finally, let us consider biological plausibility briefly.
Both networks where simulated in spiking neurons, which ensures a certain degree of biological plausibility.
Also, accumulation of evidence to a threshold is a well known finding for neurons in LIP \parencite{gold2007,smith2004}.
Often this is assumed to be a gradual integration, but when looking at individual trials instead of the trial-average, a distinct step response becomes evident \parencite{latimer2015}.
This matches the output of the second layer of the IA network.
However, both networks have an integration layer with gradually increasing firing rates, which implies that such neurons should exist too.
Ultimately, it is possible, that the brain employs both networks for different tasks as they have different strengths and weaknesses.


\section{Recall network}\label{sec:recall-net}
The recall network in the CUE model is based on the independent accumulator network.
Each potentially recallable item is regarded as one choice.
An additional \emph{null choice} is added to represent a failed recall.
This is a stand-in for additional items that might be present in the recall network, but have not occurred in the learned list.
It is also a way of providing a time-limit on trying to recall a particular item, and prevents pure noise resulting in a successful recall of one of the learned list items.
The additional choice is fed a constant signal of $\minev$.
Additive Gaussian white noise is applied to each input with zero mean and a standard deviation of $\recnoise$ to account for additional processes that might interfere with the recall process.

Furthermore, the IA network is embedded into further components (\cref{fig:recall}).
First, items are represented as Semantic Pointers, but the IA network needs a separate utility value for each potential choice.
Thus, the incoming connection uses the matrix of the Semantic Pointers of all possible list items as a transform, effectively calculating a dot product between the input signal and each potential item.
These utility values are then rectified to only consider positive evidence.
By subtracting \num{0.1} from the input utilities provided by the OSE output, integration of pure noise from a failed short-term memory recall is prevented.
\begin{figure}
    \centering
    \begin{tikzpicture}[nef]
        \graph [no placement] {
            noise [ext, x=-.5, y=1.5] -> xi/"$x_i$" [ea, x=0, y=0] <-> xib/"$\bar{x}_i$" [ea, x=1.5, y=0];
            oserectshadow/"" [ea, x=-1.5, y=0.5, inner sep=0.25ex];
            tcmrectshadow/"" [ea, x=-1.5, y=-0.5, inner sep=0.25ex];
            OSE [ext, x=-3, y=0.5] -> ["$\mat V$"] oserect/"" [rect, x=-1.5, y=0.5] -> xi;
            b1/"$-0.1$" [ext, x=-2, y=1.5] -> oserect;
            TCM [ext, x=-3, y=-0.5] -> ["$\mat V$"] tcmrect/"" [rect, x=-1.5, y=-0.5] -> xi;

            i1/"" [ens, x=3, y=1];
            xib -> ["$\mat V\Tr$"] g1/"" [ea, x=4, y=0] -> m1/"" [ea, x=5.5, y=0] -> recalled [ext, x=8, y=0];
            m1 -> [bend right, "$-1$"] g1;
            m1 -> [recurrent] m1;

            b2/"$1$" [ext, x=3, y=2] -> i1;

            xib -> ["$x_{\ped{null}}$" {very near end, xshift=3mm}] m2/"" [ens, x=2.5, y=-2.25] -> r2/"" [rect, x=1.5, y=-2.25] -> [inhibit, "$\Heavi(x - 0.3)$" {fill=white, inner sep=0cm, yshift=-2mm}] xi;
            r2 -> [recurrent] r2;

            m1 -> g3/"" [ea, x=5.5, y=-2] -> m3/"" [ea, x=5.5, y=-3.5] -> [bend right] dot [net, x=8, y=-2.5] -> gs3/"" [rect, x=6.5, y=-2.5] -> [inhibit] g3;
            m1 -> [bend left] dot;
            m3 -> [recurrent] m3;
            b3/"$-0.5$" [ext, x=6.5, y=-1.5] -> gs3;

            r4shadow/"" [ea, x=0, y=-4.5, inner sep=0.25ex];
            m3 -> g4/"" [ea, x=3.5, y=-5] -> m4/"" [ea, x=2, y=-5] -> r4/"" [rect, x=0, y=-4.5] -> ["$-6$" xshift=-3mm, bend left] xi;
            m4 -> [recurrent] m4;
            m4 -> ["$-1$", bend left] g4;

            extproc/"ext.\ processing done" [ext, x=3.5, y=-6] -> [inhibit] g4;
        };
        \draw [inhibit] (xib) |- node [above] {\small $\sum$} (i1);
        \draw [-{Latex}] (i1) -| (g1);

        \node (ia) [fit=(xi) (xib), draw, dashed, gray, rounded corners=0.5em, inner sep=0.5em] {};
        \node [above=0cm of ia.north, anchor=south, gray, font={\lato}] {IA};
        
        \node (gm1) [fit={(g1) (m1) (5.5, 0.75)}, draw, dashed, gray, rounded corners=0.5em, inner sep=0.5em] {};
        \node [above=0cm of gm1.north, anchor=south, gray, font={\lato}] {output};

        \node (freset) [fit={(m2) (r2) (2.5, -1.5)}, draw, dashed, gray, rounded corners=0.5em, inner sep=0.5em] {};
        \node [below=0cm of freset.south, anchor=north, gray, font={\lato}] {failed recall reset};

        \node (stage1) [fit=(g3) (m3) (dot) (gs3) (g3) (b3), draw, dashed, gray, rounded corners=0.5em, inner sep=0.5em] {};
        \node [below=0cm of stage1.south, anchor=north, gray, font={\lato}] {recall inhibition stage 1};

        \node (stage2) [fit={(m4) (g4) (2, -4.25)}, draw, dashed, gray, rounded corners=0.5em, inner sep=0.5em] {};
        \node [below=0cm of stage2.north, anchor=south, gray, font={\lato}] {recall inhibition stage 2};
    \end{tikzpicture}    
    \caption[Recall network]{Recall network. See text for details.}\label{fig:recall}
\end{figure}

The IA network does not produce an output during the evidence accumulation phase.
It is, however, helpful to have a persistent output of the last recalled item (note that this is still different from the output of the LCA network).
% TODO why is it helpful? give a quick reason in one sentence
To achieve this, the output of the IA network is routed through a gated memory that is only updated when the IA network produces an output.
As the gated memory stores a Semantic Pointer, a transform matrix needs to be applied to the IA output to project the choice back into the Semantic Pointer space.
The updating is controlled by inhibiting the memory gate by default and disinhibiting it (by inhbiting the inhibitory population) when the IA network produces output.

Repetition errors are rarely made in recall experiments.
Thus, it is necessary to inhibit already recalled items.
However, this inhibition should not happen immediately as otherwise the recall output would be inhibited too quickly for the downstream network to act upon.
Thus, a two-stage process is used.
First, an initial working memory population gets immediately updated by feeding in the currently recalled item.
This adds the recalled Semantic Pointer into the representation of items to inhibit.
From that representation and the current recalled item, a dot product is calculated and thresholded at \num{0.5}.
Once the threshold is crossed, the gate to the memory population is inhibited as the new item should be added into the representation, but not completely overwrite it.
The output of that memory population is fed to a gated memory that provides the inhibition to recall in the IA\@.
The gate for this memory is controlled externally and disinhibited once the downstream processing has completed.
Note that the output of that memory population is projected back into utility values and thresholded to prevent positive evidence from the inhibition memory.
% TODO verb missing in previous sentence

Finally, the recall network needs to be reset if a recall fails to allow it to continue trying to recall the item for the next position in serial recall.
Here the problem is that if the IA output for a failed recall is directly used to inhibit the IA network to reset it, this also immediately inhibits the output used for the inhibition.
This will not reliably reset the network as the reset signal is disabling itself.
Thus, the signal will be fed to an integrator until a threshold is reached and then the signal of the integrator is used to provide an inhibitory pulse to the IA network.
The slower decay of the integrator ensures a sufficient pulse length to restart the recall process.

\chapter{The complete model}

Now we have all the essential components to construct the complete context-unified encoding (CUE) model.
\Cref{fig:general-routing} gives an overview of the information flow between the different components.
The Semantic Pointers $\vc v$ for the presented items are the input to the model and the recalled items $\hat{\vc v}$ of the \pop{item recall} network are the model output.
The part of the model corresponding to the TCM consists mainly of the $\mft$, $\mtf$, and \pop{context} networks, whereas \pop{OSE} and \pop{position} correspond to the OSE\@.
The \pop{position} network (\cref{sec:posnet}) stores a Semantic Pointer $\vc p$ indicating the current list position.
The position is advanced by a control signal discussed in the next section.
Both the current list item and position are input to the $\mft$ and \pop{OSE} networks.
\begin{figure}
    \centering
    \begin{tikzpicture}[nef]
        \graph [no placement] {
            item/"list item $\vc{v}$" [ext, x=0, y=0];
            ose/OSE [net, x=3, y=-1.5];
            position/"position $\vc{p}$" [net, x=3, y=-3];
            recall/"item recall $\hat{\vc{v}}$" [net, x=0, y=-5];
            precall/"position recall $\hat{\vc{p}}$" [net, x=0, y=-6];
            mfc/"$\mft$" [net, x=-3, y=-1];
            mcf/"$\mtf$" [net, x=-3, y=-4];
            ctx/context [net, x=-3, y=-2.5];

            ctx -> [modulatory, bend right, dashed, gray] mfc;
            item -> [modulatory, out=-120, in=0, dashed, gray] mcf;
            position -> [modulatory, out=180, in=0, dashed, gray] mcf;
            item -> [in=0, out=-120] mfc;
            item -> [out=-60, in=170] ose;
            position -> [in=0, out=180] mfc;
            position -> [out=180, in=190, distance=20] ose;
            mfc -> [bend right] ctx;
            mfc -> [out=220, in=140] mcf;
            ctx -> [out=270, in=90] mcf;
            mcf -> [out=270, in=180] recall;
            mcf -> [out=270, in=180] precall;
            precall -> [out=0, in=190] position;
            ose -> [out=180, in=90] recall;
            recall -> [out=162, in=-15] mfc;
            recall -> output [ext, x=3, y=-5];
        };
    \end{tikzpicture}
    \caption[General information flow in the CUE model.]{General information flow in the CUE model. The routing of information depending on the task, and task phase is not shown in this figures. Thus, not all shown connections are active at all times.}\label{fig:general-routing}
\end{figure}

Within the \pop{OSE} network, the list item and position Semantic Pointers are bound together and added into the representation of the current list in a neural integrator.
The inputted position is also used to unbind an item from the list representation and feed it to the \pop{item recall} network.

In the $\mft$ network the superposition of the input item and position (instead of the binding) is created.
This superposition is used to recall the context previously associated with the item and position and to update the current context in the \pop{context} network accordingly.
Furthermore, the current context is fed back to $\mft$ as a modulatory signal to learn the association from the current item and position input to the current context.
Via the $\mtf$ network, the context recalls the associated Semantic Pointer and transmits it to the \pop{item recall} and \pop{position recall} networks.
The current item and position are a modulatory input to the $\mtf$ network to create the association from the current context to these Semantic Pointers.

During the recall phase, recalled items and positions are routed back to the $\mft$ network to recall further items.
The recalled position, also sets the current position in the \pop{position} network to recall that position's item from the \pop{OSE} short-term memory.
The recall networks also store recently recalled items in neural integrators to prevent repetition errors that happen rarely in human experiments.

% TODO more detailed Gephi visualization? Neuron numbers etc.?


\section{Control}\label{sec:control}
While the general structure of the model is important, the desired model behaviour can only be achieved by controlling the flow of information appropriately.
This control happens on multiple levels.
On the highest levels, the effective connectivity is modified by the general task performed (e.g., an immediate versus a serial recall task) and the task phase (e.g., presentation versus recall phase).
Below that, certain information routing is done for each item until it has been stored in memory or for each recall.
On the lowest levels, some control and routing happens within the individual networks of the CUE model as described in the corresponding sections.
For example, the $\mft$ and $\mtf$ networks inhibit the modulatory signal once a certain association strength has been reached.

\Cref{fig:pres-routing} shows the information flow during the presentation phase.
Parts of the recall networks and the input to them are inhibited.
During the recall phase, the routing of information depends in part on the type of recall task as shown in \cref{fig:recall-routing}.
For serial recall, the transmission of the recall network outputs is inhibited because the recalled item is not supposed to be a cue for recalling the next item.
Instead, the output of the \pop{position} network is used as a cue.
Also, for this cue to be most effective, the output of the $\mft$ network is routed directly to the $\mtf$ network.
During free recall, instead, the output of $\mft$ is used to update the context as usual and the updated context acts as input to the $\mtf$ network.
\begin{figure}
    \centering
    \begin{tikzpicture}[nef]
        \graph [no placement] {
            item/"list item $\vc{v}$" [ext, x=0, y=0];
            ose/OSE [net, x=3, y=-1.5];
            position/"position $\vc{p}$" [net, x=3, y=-3];
            recall/"item recall $\hat{\vc{v}}$" [net, x=0, y=-5, dashed];
            precall/"position recall $\hat{\vc{p}}$" [net, x=0, y=-6, dashed];
            mfc/"$\mft$" [net, x=-3, y=-1];
            mcf/"$\mtf$" [net, x=-3, y=-4];
            ctx/context [net, x=-3, y=-2.5];

            ctx -> [modulatory, bend right, dashed, gray] mfc;
            item -> [modulatory, out=-120, in=0, dashed, gray] mcf;
            position -> [modulatory, out=180, in=0, dashed, gray] mcf;
            item -> [in=0, out=-120] mfc;
            item -> [out=-60, in=170] ose;
            position -> [in=0, out=180] mfc;
            position -> [out=180, in=190, distance=20] ose;
            mfc -> [bend right] ctx;
            ctx -> [out=270, in=90] mcf;
            precall -> [out=0, in=190] position;
            recall -> [out=162, in=-15] mfc;
        };
    \end{tikzpicture}
    \caption{Information flow during the presentation phase.}\label{fig:pres-routing}
\end{figure}
\begin{figure}
    \subcaptionbox{Serial recall}{
        \begin{tikzpicture}[nef]
            \graph [no placement] {
                ose/OSE [net, x=2, y=-1.5];
                position/"position $\vc{p}$" [net, x=2, y=-3];
                recall/"item recall $\hat{\vc{v}}$" [net, x=0, y=-5];
                precall/"position recall $\hat{\vc{p}}$" [net, x=0, y=-6];
                mfc/"$\mft$" [net, x=-2.5, y=-1];
                mcf/"$\mtf$" [net, x=-2.5, y=-4];
                ctx/context [net, x=-2.5, y=-2.5];

                position -> [in=0, out=180] mfc;
                position -> [out=180, in=190, distance=20] ose;
                mfc -> ctx;
                mfc -> [out=220, in=140] mcf;
                mcf -> [out=270, in=180] recall;
                mcf -> [out=270, in=180] precall;
                ose -> [out=180, in=90] recall;
                recall -> output [ext, x=2.5, y=-5];
            };
        \end{tikzpicture}
    }
    \hfill
    \subcaptionbox{Free recall}{
        \begin{tikzpicture}[nef]
            \graph [no placement] {
                ose/OSE [net, x=2.5, y=-1.5];
                position/"position $\vc{p}$" [net, x=2.5, y=-3];
                recall/"item recall $\hat{\vc{v}}$" [net, x=0, y=-5];
                precall/"position recall $\hat{\vc{p}}$" [net, x=0, y=-6];
                mfc/"$\mft$" [net, x=-2.5, y=-1];
                mcf/"$\mtf$" [net, x=-2.5, y=-4];
                ctx/context [net, x=-2.5, y=-2.5];

                position -> [in=0, out=180] mfc;
                position -> [out=180, in=190, distance=20] ose;
                mfc -> ctx;
                ctx -> [out=270, in=90] mcf;
                mcf -> [out=270, in=180] recall;
                mcf -> [out=270, in=180] precall;
                precall -> [out=0, in=190] position;
                ose -> [out=180, in=90] recall;
                recall -> [out=162, in=-15] mfc;
                recall -> output [ext, x=2.5, y=-5];
            };
        \end{tikzpicture}
    }
    \caption{Information flow during the recall phase.}\label{fig:recall-routing}
\end{figure}

The main control problem for each item is to regulate the context update because (as discussed in \cref{sec:ctx-update}) this cannot be done based on the context signal alone, but requires an external control signal.
Before the context can be updated, the new input signal \ctxin\ needs to be present at the network, input which requires a delay after a new list item has been presented.
Accordingly, as soon as a new item is presented, the context update should be stopped until that input signal is propagated and the current context has been propagated to the buffer for the old context in the context network.
To achieve this delay, the Semantic Pointer of the new item is fed into an integrator with a slow synaptic time constant of $\tau = \SI{0.1}{\second}$ (\cref{fig:simth-ctrl}).
Between the input Semantic Pointer and the output of the integrator a dot product is calculated that slowly increases towards one.
The threshold obtained with a thresholding ensemble gives the required control signal for the presentation phase.
During the recall phase, the logic is inverted.
That enables an immediate update of the current context based on the position provided by the \pop{position} network and last recalled item.
Once an item has been recalled, it is used like a newly presented item and fed to the integrator and dot product.
This provides the signal to transfer the current context to the secondary buffer after a delay.
The thresholded dot product signal is also used to control three other aspects of the model:
\begin{itemize}
    \item It is required to enable the learning of associations in the $\mft$ and $\mtf$ matrices to prevent the creation of associations before the context has been updated.
    \item It is used to provide the control signal to transfer the updated OSE memory to the secondary memory buffer to allow for the next update.
    \item The inverted signal is used to gate the transmission of the recalled item in the recall network to the memory of recalled items preventing repetition errors.
\end{itemize}
\begin{figure}
    \centering
    \begin{tikzpicture}[nef]
        \graph [no placement] {
            presented [ext, x=-.25, y=1.5];
            recalled [ext, x=-.25, y=0];
            int/"" [ea, x=2.5, y=1.5];
            dot [net, x=2.5, y=0];
            simth/"" [rect, x=4, y=0];
            simthpos/"" [rect, x=5.5, y=0.75];
            simthneg/"" [rect, x=5.5, y=-0.75];
            presphase/"presentation phase" [ext, x=5.5, y=2];
            recallphase/"recall phase" [ext, x=5.5, y=-2];
            posinc/"increment position" [ext, x=8, y=0.75, anchor=west];
            ctxupdate/"update context" [ext, x=8, y=-0.75, anchor=west];
            pos/"" [inner sep=0cm, minimum size=0cm, x=6.5, y=0.75];
            neg/"" [inner sep=0cm, minimum size=0cm, x=6.5, y=-0.75];
            osestore/"OSE store" [ext, x=5, y=-3, anchor=west];
            procdone/"recall net processing done" [ext, x=5, y=-3.5, anchor=west];

            presented -> ["$\tau = \SI{0.1}{\second}$" above] int;
            presented -> [out=0, in=180] dot;
            recalled -> [out = 0, in=180] int;
            recalled -> dot;

            int -> [recurrent] int;
            int -> dot;

            dot -> simth;
            simth -> [out=0, in=180] simthpos;
            simth -> [out=0, in=180] simthneg;

            presphase -> [inhibit] simthpos;
            recallphase -> [inhibit] simthneg;

            simthpos -- pos -> posinc;
            pos -> [out=0, in=180] ctxupdate;
            simthneg -- ["$-1$"] neg -> [out=0, in=180] posinc;
            neg -> ctxupdate;

            simth -> [out=270, in=180] osestore;
            simth -> [out=270, in=180] procdone;
            bias1/"$-1$" [ext, x=3.5, y=-3.5] -> procdone;
        };
    \end{tikzpicture}
    \caption{Generation of control signals from the currently presented or recalled item.}\label{fig:simth-ctrl}
\end{figure}

Special control and information routing is also necessary during the distractor tasks or when no list item stimulus is present.
Because the distractors are irrelevant to the task, they are assumed to be not encoded in the hippocampal long-term memory.
Thus, during the distractor phases the error signals for the association learning are inhibited to prevent the distractors from being learned.
Moreover, the distractors are not part of the learned list and thus the advancement of the position counter is inhibited.
Instead of the position network output, a different Semantic Pointer indicating an irrelevant position is routed to networks otherwise receiving the position Semantic Pointer. 

Switching of task phases also requires some reconfiguration of the network state.
The start of the recall phase is detected with thresholded differentiators for serial and free recall (one of them is inhibited).
For serial recall the position network is reset to the first position by exciting the neural ensemble for the first position and inhibiting all others.
At the same time the Semantic Pointer for the first position is fed to the \mft\ network to start of the recall while the position network is still transitioning to representing the first position.
Because some subjects may use a serial recall strategy even in free recall, models are configured with probability $\psi = 0.1$ to perform this serial recall routing even in free recall (except in the delayed recall condition).
In free recall, the position network is inhibited at the start of recall to base the first recall purely on the currently active context signal.
If later during the free recall process a position vector is recalled, it may still set the position network to represent that position.

Finally, the recall networks might fail to recall an item (or position) if the input evidence is too low or too noisy.
While these networks will restart the recall process internally, some global actions are necessary for the failed recall of items.
The context network is provided with a signal to update the current context to then use the updated context for the next recall attempt.
Also, for serial recall, the position network is provided with a signal to increment the current position to attempt recalling the next serial position in the list.

\chapter{Results}
To validate the CUE model, I matched it against human experimental data of serial and free recall experiments.
The same model architecture was used in all of these simulations with only small changes is some parameter values across experimental conditions.
The simulations were designed to replicate the experimental paradigms as closely as possible.
In particular, the list length, item presentation times, delay times, and recall times where matched.

To model the effect of distractor tasks during delay phases, non-list items where presented a rate of $\drate$ items per second.
These were allowed to influence the STM component, but learning in the LTM component was disabled as these items were irrelevant to the main memorization task.
This is similar to how \textcite{Howard2002} modeled the distractor interval, though in their case they did not define the distractor rate, but the effective length of the distractor interval.
As I was aiming to match the experimental paradigms as closely as possible, changing the length of the distractor interval was not an option.

As long as not otherwise noted, 100 simulations with different seeds were run per experimental condition.
This number is sufficient to get clear results with reasonably small confidence intervals, but still small enough that the simulation and parameter matching is feasible on a high-performance computing cluster.


TODO parameter values


\section{Serial recall}
In serial recall participants are asked to recall items in the same order as they were presented.
In an experiment presented by \textcite{Jahnke1968} list of ten items was presented at the rate of one item per second and recalled immediately.
Figure TODO shows the serial position curve for the experimental and model data and the distribution of transposition errors averaged over all serial positions.
In both cases the serial position curve show a clear primacy and recency effect.
While \textcite{Jahnke1968} did not provide the transposition data, the model matches qualitatively the transposition gradients reported elsewhere (TODO ref).
In general, few transposition errors are made, but for those that occur transpositions of nearby items are more likely than transpositions of distant items.

TODO primacy by LTM, recency by STM\@?


\section{Free recall}
While the order of recall is predetermined in serial recall, in free recall list items may be recalled in any order.
Here, I provide the model match to the data from \textcite{Howard1999} which also has been used in original fits of the TCM model in \textcite{Howard2002} and \textcite{Sederberg2008}.
Three experimental conditions were attempted to match: immediate recall, delayed recall, and continuous distractor recall.

In the immediate free recall condition, list items were presented at a rate of one item every second.
After the presentation phase a recall phase of \SI{45}{\second} followed immediately.
This protocol was changed to a presentation rate of one item every \SI{1.2}{\second} and a recall phase of \SI{60}{\second} in the delayed and continuous distractor conditions.
In both of these conditions, the presentation and recall phase were separated by a \SI{16}{\second} distractor task.
In addition, in the continuous distractor conditions such a \SI{16}{\second} distractor phase would be inserted in between every pair of items.
A list length of 12 items was used in all conditions.

A number of resulting metrics from the model and experimental data is shown in Figure TODO\@.
First, the distribution of the total number of successful recalls is shown.
To my knowledge, this data has not been analyzed for the original TCM model, even though it is arguably the most fundamental comparison.
In all conditions, the 95\% confidence intervals of the mean, standard deviation, and kurtosis overlap with the exception of the kurtosis in the immediate recall condition.
Thus, no significant difference for the most essential moments could be shown which indicates that the model approximates the experimental distributions well, even though an equality of the distributions cannot be inferred.
TODO KL-divergence?

Second, the serial position curves are given.
The strong recency effect in immediate recall is attenuated in delayed recall, but reappears to some degree in continuous distractor recall.
Interestingly, the recency effect in immediate recall give the curve an S-like shape that is missing in continuous distractor recall.

Third, these effects also show in the probability of first recall.
In immediate recall, the first recall is with high probability from the end of the list, whereas in delayed recall the probability is much more uniform.
In continuous distractor recall, the probability to start the recall at the end of the list is partially restored.

Finally, the conditional response probability (CRP) gives the probability of how much lag between the positions of to recalled items is.
For example, the asymmetry in immediate recall shows the bias to do forward recall and the peak around zero that nearby items tend to be recalled together.
Both of these effects become attenuated in delayed and continuous distractor recall.
In delayed free recall, the model predicts a stronger forward bias than the experimental data shows.


\section{Scopolamine}
A spiking neural network model allows to investigate the effects of drugs more readily than a pure mathematical model.
I demonstrate this here with the acetylcholine antagonist scopolamine.
Administered before presentation phase in an immediate recall experiment, it will be detrimental to the recall performance~\parencite{ghoneim1975}.
However, scopolamine administered inbetween the presentation and recall phase does not influence performance.
This indicates that scopolamine prevents encoding of new memories in LTM, but does not prevent recall of already encoded memories.
More precisely, scopolamine has been shown to attenuate long-term potentiation in hippocampus~\parencite{leung2003,ito1988,hirotsu1989}.

To model the effect of scopolamine on LTP, the AML learning rate for learning the $\mtf$ and $\mft$ matrices can be adjusted.
By reducing it to TODO, the experimental results by \textcite{ghoneim1975} can be reproduced (see Figure TODO).
The simulation protocols were again modeled to replicate the experimental settings with 16 items per list, and a presentation time of two seconds per item.


\section{Hebb repetition effect}


\section{Spiking behaviour}

\chapter{Discussion}
In this thesis, I presented the context-unified encoding (CUE) model.
To my knowledge, it is the first spiking neural network model of human memory that integrates activity-based short-term memory and weight-based long-term memory.
The same model matches a variety of behavioural data from serial and free recall experiments, but in contrast to previous models provides a neural mechanistic explanation.

The CUE model exhibits many of the hallmark findings in memory research.
It shows the primacy and recency effect in immediate serial and free recall.
These effects get attenuated in delayed free recall, but in continuous distractor free recall the recency effect reappears.
Furthermore, the model was found to make very view transposition errors in serial recall and if it does so nearby items will be transposed.
In the free recall conditions, the model tends to start with items at the end of the list, recall nearby items together, and favour recall in forward direction.
Introducing delays and distractors attenuates these effects.
All of this matches the findings from experiments with human subjects.

Not only the qualitative effects are reproduced, but also the quantitative match to the data is good.
Only very few significant differences close to the number of differences expected by chance were found.
One of these differences is worth to be considered in more detail: the model predicts a too strong forward bias in delayed free recall with both the lag \num{1} and \num{2} values of the CRP curve being significantly above the experimentally found values.
Interestingly, this is also highlighted as the least well matched aspect in the original TCM \parencite{Howard2002}.
While in that publication the TCM prediction is closer to the experimental data, the TCM prediction from a more recent paper \parencite{Sederberg2008} is closer to the CUE model prediction.
This makes it likely that the difference is not based on pure statistical chance, but that both the TCM and CUE model do not capture an essential aspect of memory, potentially related to the evolution of the context signal, that leads to reduced forward bias in delayed free recall.
It remains for future work, to precisely identify the reason for this mismatch and extend the model.

Opposed to pure math models, the implementation as a spiking neural network allows to compare and validate the model against data from neural recordings in addition to the behavioural in the future.
In \cref{sec:aml-neural}, it was already demonstrated that the isolated mechanism of association learning is able to explain neural data.
Unfortunately, neural data recorded from human in memory experiments is still scarce because invasive recordings can only be done when such recordings are required for medical reasons.
Nevertheless, implementing models with spiking neurons is worthwhile for several other reasons, despite the more complicated model construction and increased simulations times.
Drug effects, like scopolamine, can be more readily modelled as was done with CUE model.
Also a higher degree of biological plausibility is achieved as one is forced to consider, for example, spiking noise and synaptic time constants.
This prevents common assumptions like arbitrary precision or perfectly orthogonal vectors made in many math models.

The spiking neural implementation also helps constraining many parameter values.
Synaptic time constants, membrane time constants, and similar cellular physical quantities can be set to biological plausible values constrained by experimental findings.
These are fixed parameters that have not been adjusted for matching the data.
Similarly, as in the NEF most connection weights are directly determined by least-squares minimization to implement a given function determined by the prescribed model architecture, the connection weights are fixed as well.
This leaves the model with very few free parameters.

To match the immediate serial recall, only two parameters were adjusted: the bias of the null choice $\minev$ and the input noise standard deviation $\recnoise$ in recall.
Both account for the fact that the recall network was restricted to recalling the items used within the memory experiment, while in reality a number of other items might interfere with the recall process.
For free recall experiments, one additional parameter $\psi$ is added that determines the probability to use the serial recall strategy even for free recall.
(For serial recall, a fixed values of $\psi = 1$ is implied as no free recall is allowed.)
Furthermore, in experiments with delay periods, a distractor rate $\drate$ needs to be set.
Lastly, to simulate the effect of scopolamine the AML learning rate $\eta$ was adjusted.
However, in non-scopolamine conditions, it was treated as a fixed parameter and set to a value high enough to learn associations until the threshold for inhibition was achieved within the presentation duration.
Even higher values would not have any effect as long as it does not largely exceed the inverse of the synaptic time delay of the inhibition.

While few free parameters are desirable with respect to model parsimony, they should also be assigned similar values to model related experimental conditions.
This is mostly the case for the CUE model.
The bias of the null choice in recall ranges from \numrange{0.03}{0.04} and values get monotonously smaller as the task difficulty increases with additional delays.
This corresponds to plausible longer recall attempts in more difficult experimental conditions.
Only a small difference is also observed in the distractor rates (\numrange{0.03}{0.04}) and the probability of using a serial recall strategy (zero for delayed recall and \num{0.1} in all other free recall conditions).
However, the noise standard deviation $\recnoise$ in recall differs by a factor of more than \num{1.5} without a clear relation to the experimental condition.
It is hard to hypothesize potential reasons for this difference as the parameter is accounting for things not explicitly modeled in recall.

It is also of interest how robust the model is against parameter changes.
I have not done a formal analysis of this because the model simulation times are prohibitive.
However, this also means that only a small set of parameter values without a lot of fine tuning has been tested (less than \num{200} combinations summed over all experimental conditions).
Given that finding the right parameters with few simulations is less likely if the model were highly sensitive to the parameter choice, a sufficient robustness to the exact choice of parameter values can be expected.

The CUE model is based on prior model of memory, but improves on them in important ways.
With regard to the OSE model two main advancements can be stated.
First, The episodic memory buffer has been replaced with a much more plausible long-term memory mechanism that relies on synaptic-weight changes rather than reverberating neural activity.
Second, the CUE model also implements the mechanism providing the position tags fully in spiking neurons.

Implementing a long-term memory component based on the TCM in a spiking neural network provides a strong support for the biological plausibility of the TCM that previously was missing.
Certain simplifications of the TCM equations in this process to facilitate this implementation highlight which aspects of the TCM are essential and which do not contribute to the explanation of the data.
In particular, it also shows that certain assumptions, like perfectly orthogonal vectors, useful in the mathematical analysis, are not essential.
In addition, the modified TCM has been extended with a short-term memory component in the CUE model.
While the TCM has been posited as a single-store model, this has been criticized \parencite{Davelaar2008}.
The CUE model demonstrates that treating the TCM as part of a multi-store model is not unreasonable, provides good matches to the free recall data and in addition allows to match serial recall data.
Finally, the recall process in the TCM was not modeled in a particular biological plausible way and has been replaced with a more plausible spiking neural mechanism (\cref{sec:recall}).

In the broader context of memory models, the CUE model is unique as providing a low-level spiking network implementation, but matching high-level behavioural data.
This includes the recall process that is not explicitly modeled in many other models.
Furthermore, due to the item based context, there is no reinstantiation problem found in most context-based memory models (except the TCM and CUE model).


[AML]
central part, meaning, insights?

\begin{itemize}
    %\item match multitude of data
    %\item delayed CRP not worse the original paper
    %\item Robustness against parameter change (picked with few simulation, thus robust)
    \item basal ganglia involvement?
    %\item changes with regard to original TCM\@: finite vector dimensionality, not perfectly orthogonal; simpler equations
    %\item partially better match than TCM\@?
    \item Role of AML
    %\item neural position counting
    \item Control is the hard part
    %\item From math model to spiking neural network model viable and useful 
    %(why?)
    %\item Consideration of spiking noise
    \item exact implementation of experimental protocol
    %\item discuss merits of model in terms of desired properties of models
    %\item Explains context reinstantiation
    %\item giving account of response generation missing from other models
    \item fuzzy temp memory sensitive to noise and thus no alternative for context signal
    \item Things developed on the way of creating large scale model: spaopt, better product, binding operations, wta network, optimizer
    \item Integration with Spaun, use actual distractor task
\end{itemize}

\section{Anatomical mapping}

Given that the CUE model is neural, it is of interest to consider how parts of the model map to brain areas.
TODO proposed a mapping of the TCM model to brain areas that applies to a large degree also to the TCM-based part of the CUE model.
There are however some details to be reconsidered and the OSE-based short-term part has obviously not been discussed.

The medial temporal lobe (MTL) is known to be essential for free recall.
Damage to the MTL is detrimental to free recall performance (TODO ref Graf, Squire, Mandler 1984).
Thus, we can assume the TCM related parts of the model to reside in the MTL (TODO figure).

More precisely, the context storage network can be mapped  onto parahippocampal areas, in particular the entorhinal cortex (EC).
Its properties are consistent with the storage of non-spatial memories for tens of seconds (TODO ref delayed match to sample papers).
Also, the electrophysiological properties of the EC support integration (TODO ref Egorow, Hamary, Fransen, Hasselmo, Alonso 2002).
Note, that the context network does contain integration ensembles that maintain the context signal over the timespan of seconds.

While the EC firing is modulated to some degree by the position, this coding is more noisy than in the hippocampus (TODO ref Quirk et al 1992).
This indicates that EC codes for additional information.
TODO LM Frank et al 2000 have shown that superficial EC employs retrospective coding, that it differentiates visits of the same position by the history leading up to that visit.
This is consistent with a context signal encoding the history of items leading up to the current item.

The other major component to map to brain structures are the learning and retrieval of associations in the $\mft$ and $\mtf$ matrices.
TODO ref stated that the $\mft$ matrix might not be implemented by a single anatomical region due to its complicated structure.
However, the updating equation for $\mft$ in the CUE model has been simplified which makes the correspondence to a single region more plausible.
The learning of new associations in these matrices is attributed to the hippocampus by TODO\@.
TODO more details.
Though, they do not consider the retrieval to be dependent on the hippocampus because TODO\@.

The CUE model provides a more detailed description of the updating of the association matrices due to the neural implementation by means of the AML\@.
In the learning network we get recurrent connectivity between TODO\@.
This is analogous to recurrent connectivity in the CA3 region of hippocampus.
TODO additional input via different pathways.
Moreover, the AML highlighted the need to incorporate the decoder matrix $\mdec\Tr$ into the connections.
While I have shown it might be plausible that this connectivity is learned, the matrix could be reduced to the identity if the input ensemble were to provide orthogonalized inputs.
The dentate gyrus, providing input to CA3 (TODO is this right?) is commonly assumed to perform such orthogonalization given its large neuron count, sparse firing, and neurogenesis (TODO refs).
Thus, while the CUE model does not explicitly model the dentate gyrus (which is a significant research problem in itself), the learning rule used at least provides a principled reason for its existence.

Further evidence for this neuroanatomical mapping can be obtained from the connectivity between hippocampus and EC\@.
The superficial EC provides input to hippocampus, but does not receive direct input for hippocampus (TODO ref).
In contrast to that, the deep layers of EC receive input from hippocampus and might be relevant for recall, especially the recall of pre-experimental contex.
This is consistent with the connectivity in the model where the context network projects to the association matrix learning network attributed to hippocampus.
The learning network for $\mft$ also projects back to an ensemble recalling the prior context before it gets combined in a different ensemble.


TODO did they actually say anything about context-to-item matrix?

more going on in MTL, eg place cells

\chapter{Conclusion}

\addchap{Acknowledgements}
IK
Sharcnet/compute canada + their support team
Nengo development team and opportunities

\printbibliography[heading=bibintoc,title=References]

\appendix
\addpart{Appendices}
\chapter{Derivation of the cosine similarity distribution}
\begin{proof}
The most straight-forward way to derive the cosine similarity distribution is to use a result by \textcite{cai2013}.
Theorem~1 states that the probability density of the angles $\theta$ between independent $\dims$-dimensional vectors is given by
\begin{equation}
    h(\theta) = K \cdot \del{\sin \theta}^{\dims - 2},\quad \theta \in \sbr{0, \uppi}
\end{equation}
with
\begin{equation}
    K = \frac{1}{\sqrt{\uppi}} \frac{\Gamma\!\del{\frac{\dims}{2}}}{\Gamma\!\del{\frac{\dims - 1}{2}}} = \frac{\Gamma\!\del{\frac{\dims}{2}}}{\Gamma\!\del{\frac{1}{2}} \Gamma\!\del{\frac{\dims - 1}{2}}} = \frac{1}{B\!\del{\frac{1}{2}, \frac{\dims -1}{2}}} \text{.}
\end{equation}
To obtain the cosine similarity distribution a change of variables with $\theta = \arccos x$ has to be performed:
\begin{align}
    \pcs(x; \dims) &= \abs{\od{}{x} \del{\arccos x}} \cdot h(\arccos x) \\
    &= \frac{1}{\sqrt{1 - x^2}} \cdot K \cdot \del{\sin \arccos x}^{\dims - 2} \\
    &= \frac{1}{\sqrt{1 - x^2}} \cdot K \cdot \del{\sqrt{1 - x^2}}^{\dims - 2} \\
    &= K \cdot \del{1 - x^2}^{\del{\dims - 3}/2} \text{.}
\end{align}
This matches \cref{eqn:pcs}.
\end{proof}

TODO derive from square root beta


\end{document}
