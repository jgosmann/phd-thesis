\chapter{Introduction}

Memory in its different forms is an important aspect of human, but also animal cognition.
For example, it allows animals to return to previously visited water and food locations (TODO refs).
Prior experiences also allow to act more optimally in similar situations or avoid dangerous situations.
In this sense memory allows for adaptation on a faster time scale than genetic selection.
This is especially important in unstable and changing environments.
In humans, memory is also important to form social relationships, a shared culture, and even a functioning society.
Moreover, it contributes to our individual sense of self.

TODO cite Gallistel book on how addressable memory allows for more compact implementation as opposed to pure state machine

Accordingly, there is a long history of memory research.
TODO more details.

Nevertheless, there are still many open questions.
One important challenge that has to be solved by a memory system is the so-called \emph{stability-plasticity dilemma} (TODO ref Abaraham and A Robins 2005).
On the one hand, there is a need to quickly form new memories, sometimes even with a single exposure known as \emph{one-shot learning}.
On the other hand, such high plasticity can easily lead to overwriting of old memories rendering memory system useless.
TODO refo Buzsaki and others proposed multi-stage memory models where different memory system are in place for different timescales with different levels of plasticity.
However, this still leaves open TODO questions.

It also is a strong indication that different memory systems interact.
Though, it much of experimental research and modeling different memory systems have been treated as isolated.
This simplifies the analysis on a certain level, but to get a general understanding of memory as a whole the results need to integrated at some point.
Furthermore, many models of memory or learning focus on either small scale neural changes without any direct connection to behaviour or on the other extreme of describing behaviour with mathematical equations, but no solid grounding in biological plausibility.
So it is not only important to integrate our understanding of memory systems, but also to bridge the gap from neural mechanisms to behaviour.
This thesis will try to advance our understanding in this direction.

While this will still be comparably rough sketch and still fairly limited in memory systems covered, there are promising long-term prospects from a better understanding of human memory.
Many forms of memory loss are currently untreatable.
However, diseases associated with aging like Alzheimer's significantly impair the function of memory and are getting more common as our life-span increases due to other medical and nutritional advances.
(TODO stroke, hippocampal lesions?)
A better understanding of memory might allow us to devise better treatments or even stop and reverse the memory detoriation.
A potential route to this are memory implants that have already been demonstrated in rats by TODO Berger et all 2011.
For this sort of implant, an understanding of how memories are encoded is at least helpful if not crucial.

These are certainly strong motivators for research into memory, but it is also worthwhile to advance our general understanding of how the human brain works.
A question that has puzzled researchers for a long time and probably still will for a long time to come despite great advances.
An important step in testing our current understanding of the brain was the Spaun model (TODO ref).
A spiking neural network of XX neurons, thus grounded in biology, that can perform 8 different tasks.
Like a real brain it gets sensory input (low resolution black and white symbols) and produces a behavioural motor output with a simulated arm.
The number of tasks it can perform demonstrates that it is not a specialized model for a single task, but can switch between different tasks like a real brain.
Obviously, Spaun is still much simpler (and has much less neurons) like a real brain.
So there is no denying that it is still a long way to a full understanding of the whole brain.
But it incorporates a number of qualitative key aspects.
However, one key aspect is still missing.
While it can perform working memory dependent tasks and has simple reinforcement learning, it is missing a declarative and episodic long-term memory.
The work in this thesis is also a step towards providing Spaun with such a memory by integrating it with a short-term memory component similar to the one used in Spaun.


\section{Behavioural characterization of memory}
Memory is commonly characterized along two orthogonal dimensions:
by timescale and by type of information stored.
On the timescale axis one usually discriminates between short-term memory (STM), and long-term memory or short-term, long-term, and very long-term memory (vLTM).
However, there is no single agreed upon definition which timescales are encompassed in each of these terms.
Sometimes LTM is used to refer to vLTM, while sometimes LTM is in-between STM and vLTM\@.
As a rough orientation STM is usually taken to be on the sub-second timescale, up to a few seconds.
Very long-term memory is on timescales of weeks and up (TODO verify).
LTM, depending on the definition, covers timescales from minutes and up.
Moreover, the term working memory (WM) is often used to refer to active representations that allow direct mental manipulation.

In this thesis, I will use the term STM for timescales in the order of seconds and LTM for longer timescales, but separate from vLTM\@.
Moreover, I will assume STM to be based on neural-activity and thus this definition is somewhat in line with the definition of WM\@.

When classifying memory by type of information, a representation as a tree structure can be helpful.
On the highest level, we have the distinction between implicit and explicit memory.
Implicit memory does not allow for conscious access.
A typical example is procedural or motor memory like how to ride a bike.
Explicit memory allows for conscious access and is further subdivided into declarative and episodic memory.
The declarative memory allows us to store and reproduce facts like the birthday of a friend, whereas the episodic memory provides a recollection of life experiences.

Here, I will focus on declarative memory as this type of memory has been well studied in many memory experiments.


\section{Experimental findings in memory research}
Most memory experiments require the participants to memorize lists of words and recall them afterwards.
The words are typically chosen from the Toronto noun pool and are matched for semantic similarity and occurrence frequency (TODO verify, ref).
Usually an experiment will fall in one of two categories depending on how the subjects are required to recall the list items: in free recall experiments, the subjects are free to recall the items in any order; in serial recall experiments, the subjects are required to recall the items in order.

The hallmark finding in memory research, especially for serial recall experiments, are the primacy and recency effect.
Subjects tend to remember the start of list (primacy) and the end of list (recency) better (TODO fig).
Furthermore, if subjects in a serial recall experiment recall an item at the wrong position, a so-called \emph{transposition}, it is more likely to be an item that was close in the list than an item several positions spaced out.

In free recall experiments, participants usually start out by recalling the last items first.
This can be measured by the probability of first recall (TODO fig).
There is also the tendency, to recall items in forward direction and items close to the last recalled item.
The conditional response probability (CRP) measures this (TODO fig)
It gives the probability of how many items forward or backward (know as the lag) the next recall will jump.

In memory experiments the recall phase can directly follow the study phase, known as an immediate recall experiment, or a delay, potentially with a distractor task, can separate the two, known as delayed recall.
Sometimes a distraction task will be performed in between each list item which is known as continuous distractor recall.


\section{Neuroanatomy of memory}
Cerebellum for motor
Hippocampus

\section{Memory models}

Further points (some maybe for conclusion, not introduction?):
\begin{itemize}
    \item Large scale modeling has new challenges.
    \item How do things interact?
    \item How are they coordinated?
    \item Increases to simulation speed (better neural representation, optimizer)
\end{itemize}
