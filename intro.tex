\chapter{Introduction}

\begin{itemize}
    \item Importance of memory
        \begin{itemize}
            \item fast adaptation (faster then genetics)
            \item use previous experiences to act better in novel environments
            \item relationships and shared culture
            \item sense of self
        \end{itemize}
    \item Prospects of better understanding of memory
        \begin{itemize}
            \item Treatment for memory loss
            \item due to hippocampal lesions
            \item due to Alzheimer's (longer life spans!)
            \item memory implants (demonstrated in rats by Berger et al. 2011)
        \end{itemize}
    \item Open questions
        \begin{itemize}
            \item Stability-plasticity dilemma (Abaraham and A. Robins 2005)
            \item One-shot learning
            \item Proposed mechanism: multi-stage memory models (Buzsaki 1989), but not whole story?
            \item At least: memory systems interact
            \item Bridge gap from neural mechanisms to behaviour
        \end{itemize}
    \item General justification for research, but more specific: What is missing from Spaun?
        \begin{itemize}
            \item Sensory input
            \item Behavioural output
            \item Working memory dependent tasks
            \item Even some learning
            \item But not factual long term memory
        \end{itemize}
\end{itemize}

\section{Behavioural characterization of memory}
by timescale: STM, LTM, vLTM
by type of information

\section{Experimental findings in memory research}

\section{Neuroanatomy of memory}
Cerebellum for motor
Hippocampus

\section{Memory models}

Further points (some maybe for conclusion, not introduction?):
\begin{itemize}
    \item Large scale modeling has new challenges.
    \item How do things interact?
    \item How are they coordinated?
    \item Increases to simulation speed (better neural representation, optimizer)
\end{itemize}
