\chapter{Comparisons of the uniform and cosine similarity intercept distributions}\label{apdx:hdrep}
\Cref{tbl:csdist} summarizes the change in error over a wide range of parameters when switching from a uniform intercept distribution to the $\csdist(\dims + 2)$ distribution.
In general, the cosine similarity distribution performs better.
It performs equal to the uniform distribution for $\dims = 1$ in which case $\csdist(\dims + 2)$ reduces to a uniform distribution and rectified linear (rate) neurons with a regularization of $\reg = 0.1$.
The cosine similarity distribution performs slightly worse for rate neurons (LIF rate and rectified linear) when the regularization is adjusted to $\reg = 0.01$ to account for the non-existent spiking noise.
The only other case with slightly worse performance is when computing pairwise products $y_i = x_{2i - 2} x_{2i - 1},\ 1 \leq i \leq i/2$, but note that no further optimization for this sort of function has been done and the high dimensionality makes this a hard function to compute.
On the simpler squaring, the cosine similarity distribution performs better.
\begin{table}
    \begin{addmargin*}[0mm]{-26pt}
        \caption[Comparison of uniformly and $\csdist(\dims + 2)$ distributed intercepts.]{Change in representational error in the NEF when switching from uniformly distributed intercepts to $\csdist(\dims + 2)$ distributed intercepts for different dimensionalities $\dims$, neuron numbers $n$, synaptic time constants $\syntau$, decoded functions, regularization $\reg$, and neuron types.
    A negative change in error (highlighted red) means that the cosine similarity distribution performed better.
    Statistical significance, determined with bootstrapping, is marked with **** for $p < 0.0001$ and * for $p < 0.05$.}\label{tbl:csdist}
        \scriptsize\centering
        \sisetup{
            table-figures-integer=1,
            table-figures-decimal=4,
            table-sign-mantissa,
            table-number-alignment=left,
        }
        \begin{tabular}{S[table-number-alignment=right,table-figures-integer=2,table-figures-decimal=0]S[table-number-alignment=right,table-figures-integer=4,table-figures-decimal=0]S[table-figures-decimal=3]lS[table-figures-decimal=3]lS[table-space-text-post={****},round-precision=4,round-mode=places,scientific-notation=fixed,fixed-exponent=0,negative-color=BrickRed]S[table-space-text-post={****},round-precision=4,round-mode=places,scientific-notation=fixed,fixed-exponent=0,negative-color=BrickRed]S[table-space-text-post={****},round-precision=4,round-mode=places,scientific-notation=fixed,fixed-exponent=0,negative-color=BrickRed]}
\toprule
   &    &       &         &       &              & \multicolumn{3}{c}{Change in} \\
   &    &       &         &       &              &   $\langle\errdist\rangle$ &   $\langle\errnoise\rangle$ &  $\langle\errtotal\rangle$ \\
$\dims$ & $n / \dims$ & $\syntau / \si{\second}$ & Function & $\lambda$ & Neuron type &                            &                             &                            \\
\midrule
1  & 50 & 0.005 & $\vc x$ & 0.100 & LIF &     1.2849080544736352e-05 &       0.0009289477152641251 &      0.0009813160649235486 \\
2  &    &       &         &       &              &      0.0006935373487160466 &   -0.005011356652797429**** &  -0.004319006519981905**** \\
4  &    &       &         &       &              &      -0.002040085331075192 &   -0.013890338543531083**** &  -0.013721351649611621**** \\
8  &    &       &         &       &              &      0.0036561681879286392 &    -0.03181686125505534**** &  -0.026775116022725975**** \\
64 & 10 &       &         &       &              &   -0.12385726519109116**** &    -0.21313656378019652**** &   -0.24577344411013363**** \\
   & 25 &       &         &       &              &   -0.04380034323030774**** &    -0.17432274728926517**** &    -0.1778286359045759**** \\
   & 50 &       &         & 0.005 &              &   0.034684333550832655**** &      -0.283590365358218**** &   -0.27902555939727636**** \\
   &    &       &         & 0.010 & Adaptive LIF &     0.0821060272370703**** &    -0.22492585022544237**** &   -0.17895684535932888**** \\
   &    &       &         &       & LIF &   0.026205643385408477**** &     -0.2190924433586482**** &   -0.21418489823193215**** \\
   &    &       &         &       & LIF Rate &   0.026216644979428466**** &     1.3495558654430138e-15* &   0.026216644979428137**** \\
   &    &       &         &       & Rectified Linear &    0.03065106265473031**** &       5.094502290754735e-16 &    0.03065106265472966**** \\
   &    &       &         & 0.100 & Adaptive LIF &    0.03113217839300403**** &    -0.13152209374136542**** &    -0.0814949509662396**** \\
   &    &       &         &       & LIF &   -0.01998629808658868**** &    -0.13129882048505984**** &   -0.13036370447062218**** \\
   &    &       &         &       & LIF Rate &  -0.019953903694741773**** &       2.354290724182059e-16 &   -0.01995390369474198**** \\
   &    &       &         &       & Rectified Linear &      0.0021980363520863327 &      3.7418367648159223e-16 &       0.002198036352086402 \\
   &    &       &         & 0.200 & LIF &    -0.0872786554326745**** &    -0.10038627112634677**** &   -0.13198979228029828**** \\
   &    &       & $\vc x^2$ & 0.100 &              &   -0.10299033323251117**** &    0.022920261679068646**** &   -0.10163286355777879**** \\
   &    &       & $x_{2i-2} x_{2i - 1}$ &       &              &   0.008360279430439767**** &    0.011477902843039196**** &   0.014235555683535045**** \\
   &    & 0.100 & $\vc x$ &       &              &  -0.019953778698958965**** &  -0.0066234817377022505**** &  -0.020718455984526263**** \\
\bottomrule
\end{tabular}

    \end{addmargin*}
\end{table}
