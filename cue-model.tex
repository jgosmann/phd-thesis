\chapter{The complete model}

Now we have all the essential components to construct the complete context-unified encoding (CUE) model.
\Cref{fig:general-routing} gives an overview of the information flow between the different components.
The Semantic Pointers $\spv v$ for the presented items are the input to the model and the recalled items $\hat{\spv v}$ of the \pop{item recall} network are the model output.
The part of the model corresponding to the TCM consists mainly of the $\mft$, $\mtf$, and \pop{context} networks, whereas \pop{OSE} and \pop{position} correspond to the OSE\@.
The \pop{position} network (TODO ref section) stores a Semantic Pointer $\spv p$ indicating the current list position.
The position is advanced by a control signal discussed in Section TODO\@.
Both the current list item and position are input to the $\mft$ and \pop{OSE} networks.
\begin{figure}
    \centering
    \begin{tikzpicture}[nef]
        \graph [no placement] {
            item/"list item $\bm{v}$" [ext, x=0, y=0];
            ose/OSE [net, x=3, y=-1.5];
            position/"position $\bm{p}$" [net, x=3, y=-3];
            recall/"item recall $\hat{\bm{v}}$" [net, x=0, y=-5];
            precall/"position recall $\hat{\bm{p}}$" [net, x=0, y=-6];
            mfc/"$\mft$" [net, x=-3, y=-1];
            mcf/"$\mtf$" [net, x=-3, y=-4];
            ctx/context [net, x=-3, y=-2.5];

            ctx -> [modulatory, bend right, dashed, gray] mfc;
            item -> [modulatory, out=-120, in=0, dashed, gray] mcf;
            position -> [modulatory, out=180, in=0, dashed, gray] mcf;
            item -> [in=0, out=-120] mfc;
            item -> [out=-60, in=170] ose;
            position -> [in=0, out=180] mfc;
            position -> [out=180, in=190, distance=20] ose;
            mfc -> [bend right] ctx;
            mfc -> [out=220, in=140] mcf;
            ctx -> [out=270, in=90] mcf;
            mcf -> [out=270, in=180] recall;
            mcf -> [out=270, in=180] precall;
            precall -> [out=0, in=190] position;
            ose -> [out=180, in=90] recall;
            recall -> [out=162, in=-15] mfc;
        };
    \end{tikzpicture}
    \caption{TODO}\label{fig:general-routing}
\end{figure}

Within the \pop{OSE} network (TODO section), the list item and position Semantic Pointers are bound together and added into the representation of the current list in a neural integrator.
The inputted position is also used to unbind an item from the list representation and feed it to the \pop{item recall} network.

In the $\mft$ network the superposition of the input item and position (instead of the binding) is created.
This superposition is used to recall the context previously associated with the item and position and to update the current context in the \pop{context} network accordingly.
Furthermore, the current context is fed back to $\mft$ as modulatory signal to learn the association from the current item and position input to the current context.
Via the $\mtf$ network, the context recalls the associated Semantic Pointer and transmits it to the \pop{item recall} and \pop{position recall} networks.
The current item and position are a modulatory input the $\mtf$ network to create the association from the current context to these Semantic Pointers.

During the recall phase, recalled items and position are routed back to the $\mft$ network to recall further items.
The recalled position also sets the current position in the \pop{position} network to recall that position from the \pop{OSE} short-term memory.
The recall networks also store recently recalled items in neural integrators to prevent repetition errors that happen rarely in human experiments.


TODO more detailed Gephi visualization? Neuron numbers etc.?


\section{Control}
While the general structure of the model is important, the desired model behaviour can only be achieved by controlling the flow of information appropriately.
This control happens on multiple levels.
On the highest levels, the effective connectivity is modified by the general task performed (e.g., an immediate versus a serial recall task) and the task phase (e.g., presentation versus recall phase).
Below that, certain information routing is done for each item until it has been stored in memory or for each recall.
On the lowest levels, some control and routing happens within the individual networks of the CUE model as described in the corresponding sections.
For example, the $\mft$ and $\mtf$ will inhibit the modulatory signal once a certain association strength has been reached.

\Cref{fig:pres-routing} shows the information flow during the presentation phase.
Parts of the recall networks and the input to them will be inhibited.
During the recall phase the routing of information depends in part on the type of recall task as shown in \cref{fig:recall-routing}.
For serial recall, the transmission of the recall network outputs will be inhibited because the recalled item is not supposed to be a cue for recalling the next item.
Instead, the output of the \pop{position} network is used as a cue.
Also, for this cue to be most effective the output of the $\mft$ network is routed directly to the $\mtf$ network.
During free recall, instead, the output of $\mft$ is used to update the context as usual and the updated context acts as input to the $\mtf$ network.
\begin{figure}
    \centering
    \begin{tikzpicture}[nef]
        \graph [no placement] {
            item/"list item $\bm{v}$" [ext, x=0, y=0];
            ose/OSE [net, x=3, y=-1.5];
            position/"position $\bm{p}$" [net, x=3, y=-3];
            recall/"item recall $\hat{\bm{v}}$" [net, x=0, y=-5, dashed];
            precall/"position recall $\hat{\bm{p}}$" [net, x=0, y=-6, dashed];
            mfc/"$\mft$" [net, x=-3, y=-1];
            mcf/"$\mtf$" [net, x=-3, y=-4];
            ctx/context [net, x=-3, y=-2.5];

            ctx -> [modulatory, bend right, dashed, gray] mfc;
            item -> [modulatory, out=-120, in=0, dashed, gray] mcf;
            position -> [modulatory, out=180, in=0, dashed, gray] mcf;
            item -> [in=0, out=-120] mfc;
            item -> [out=-60, in=170] ose;
            position -> [in=0, out=180] mfc;
            position -> [out=180, in=190, distance=20] ose;
            mfc -> [bend right] ctx;
            ctx -> [out=270, in=90] mcf;
            precall -> [out=0, in=190] position;
            recall -> [out=162, in=-15] mfc;
        };
    \end{tikzpicture}
    \caption{TODO}\label{fig:pres-routing}
\end{figure}
\begin{figure}
    \centering
    \begin{tikzpicture}[nef]
        \graph [no placement] {
            ose/OSE [net, x=3, y=-1.5];
            position/"position $\bm{p}$" [net, x=3, y=-3];
            recall/"item recall $\hat{\bm{v}}$" [net, x=0, y=-5];
            precall/"position recall $\hat{\bm{p}}$" [net, x=0, y=-6];
            mfc/"$\mft$" [net, x=-3, y=-1];
            mcf/"$\mtf$" [net, x=-3, y=-4];
            ctx/context [net, x=-3, y=-2.5];

            position -> [in=0, out=180] mfc;
            position -> [out=180, in=190, distance=20] ose;
            mfc -> ctx;
            mfc -> [out=220, in=140] mcf;
            mcf -> [out=270, in=180] recall;
            mcf -> [out=270, in=180] precall;
            ose -> [out=180, in=90] recall;
        };
    \end{tikzpicture}
    \begin{tikzpicture}[nef]
        \graph [no placement] {
            ose/OSE [net, x=3, y=-1.5];
            position/"position $\bm{p}$" [net, x=3, y=-3];
            recall/"item recall $\hat{\bm{v}}$" [net, x=0, y=-5];
            precall/"position recall $\hat{\bm{p}}$" [net, x=0, y=-6];
            mfc/"$\mft$" [net, x=-3, y=-1];
            mcf/"$\mtf$" [net, x=-3, y=-4];
            ctx/context [net, x=-3, y=-2.5];

            position -> [in=0, out=180] mfc;
            position -> [out=180, in=190, distance=20] ose;
            mfc -> ctx;
            ctx -> [out=270, in=90] mcf;
            mcf -> [out=270, in=180] recall;
            mcf -> [out=270, in=180] precall;
            precall -> [out=0, in=190] position;
            ose -> [out=180, in=90] recall;
            recall -> [out=162, in=-15] mfc;
        };
    \end{tikzpicture}
    \caption{TODO}\label{fig:recall-routing}
\end{figure}



